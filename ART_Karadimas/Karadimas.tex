\begin{artengenv}{Panagiotis Karadimas}
	{Explanation, representation and information}
	{Explanation, representation and information}
	{Explanation, representation and information}
	{National and Kapodistrian University of Athens;\\
	Hellenic Air Force Academy}
	{The ontic conception of explanation is predicated on the proposition that ``explanation is a~relation between real objects in the world'' and hence, according to this approach, scientific explanation cannot take place absent such a~premise. Despite the fact that critics have emphasized several drawbacks of the ontic conception, as for example its inability to address the so-called ``abstract explanations'', the debate is not settled and the ontic view can claim to capture cases of explanation that are non-abstract, such as causal relations between events. However, by eliminating the distinction between abstract and non-abstract explanations, it follows that ontic and epistemic proposals can no longer contend to capture different cases of explanation and either all are captured by the ontic view or all are captured by the epistemic view. On closer inspection, it turns out that the ontic view deals with events that fall outside the scientists' scope of observation and that it does not accommodate common instances of explanation such as explanations from false propositions and hence it cannot establish itself as the dominant philosophical stance with respect to explanation. On the contrary, the epistemic conception does account for almost all episodes of explanation and can be described as a~relation between representations, whereby the explanans transmit information to the explanandum and that this information can come, dependent on context, in the form of any of the available theories of explanation (law-like, unificatory, causal and non-causal). The range of application of the ontic view thus is severely restricted to trivial cases of explanation that come through direct observation of the events involved in an explanation and explanation is to be mostly conceived epistemically.
	}
	{scientific explanation, representation, optimization process, ontic conception of explanation, epistemic conception of explanation.}



\section{Introduction}
\lettrine[loversize=0.13,lines=2,lraise=-0.03,nindent=0em,findent=0.2pt]%
{T}{}he main ontic thesis is that explanations are not representations (texts, diagrams et.al.) and that they are in fact relations between real objects in the world
%\label{ref:RNDc8rmBrQnVK}(Glennan, 2005; Craver, 2007).
\parencites[][]{glennan_modeling_2005}[][]{craver_explaining_2007}. %
 Hence what scientific explanation amounts to, the argument goes, is causally relating the objects in question and showcasing how they can exhibit the causal patterns of the world 
%\label{ref:RNDpHelcf0jAc}(Salmon, 1984; 1989; 1998).
\parencites[][]{salmon_scientific_1984}[][]{salmon_four_1989}[][]{salmon_causality_1998}. %
 Even though this philosophical stance is not always clearly formulated 
%\label{ref:RNDUIPFpRZqxF}(Wright, 2015)
\parencite[][]{wright_ontic_2015} %
 and furthermore it has been shown that much of the causal-mechanistic explanations that supposedly vindicate the ontic view, can be reconstructed without adhering to such an ontic conception of explanation 
%\label{ref:RNDi1WVeDytMe}(Wright, 2012),
\parencite[][]{wright_mechanistic_2012}, %
 a~great deal of the current literature does rely on the main tenets of the ontic conception. And although the relevant arguments may differ slightly as for the lengths to which each account is willing to go when calling for ontological commitments, all of them eventually posit that explanation is a~relation between real objects in the world 
%\label{ref:RNDCJPfHfw1rn}(Machamer, Darden and Craver, 2000; Craver, 2005; Craver and Bechtel, 2006; Winning, 2020).
\parencites[][]{machamer_thinking_2000}[][]{craver_beyond_2005}[][]{sarkar_mechaninsm_2006}[][]{winning_mechanistic_2020}. %
 Wright and van Eck have offered the strongest---at least to the best of my knowledge---objection to the ontic conception. Their argument in part entails the view that the ontic proposal does not account for ``generalized explanations'', namely explanations that are given by representations which abstract away from details of the world and that such instances of explanation can be conceived only epistemically 
%\label{ref:RNDiYzs6YDu0E}(Wright and Van Eck, 2018, p.1019).
\parencite[][p.1019]{wright_ontic_2018}.%


Abstracting away from details of the real world that are considered unimportant to an explanation, is an issue that has attracted the attention of scholars and, one way or another, it amounts to the assumption that scientists often intentionally omit irrelevant details of the world when pursuing explanations. While Wright and van Eck are right to mention that such explanations as the so-called ``abstract'' ones cast doubt on the ontic approach to explanation, this philosophical stance seems to allow for some reconciliation of the ontic and the epistemic approaches to explanation. That is, it may be argued that ``abstract'' explanations and ontic explanations explain in different circumstances or that the epistemic view is mostly normative while the ontic is the one that in fact gives explanations, and so, they are not mutually exclusive philosophical and epistemic points of view. Sheredos for example seems to argue along these lines
%\label{ref:RNDH3RtQ0TwsI}(Sheredos, 2016; 2019).
\parencites[][]{sheredos_re-reconciling_2016}[][]{sheredos_re-re-reconciling_2019}.%


However, it does seem to me that instances of explanation that are now considered to be ``abstract'' are not in fact a~special case of explanation, or a~distinct category thereof, but rather constitute part of the standard way scientific explanations are offered. To appreciate this, we first need to re-conceptualize ``abstract explanations'' and to consider them no more that way, but rather, as one of the ways ``optimization process'' occurs. During optimization, scientists rule out parameters that convey information that can render an explanation irrelevant. These can be trivial details of the world or even theoretical postulates that are irrelevant in context. Optimization processes are, as Strevens has long argued, a~central ingredient in scientific explanations and given the contextual nature of why-questions, no explanation can be given without an optimization process for if that happens then the information that is transmitted can be irrelevant.

In view of such a~novel description of ``abstract explanations'' as mere optimization processes, we can go one step further and consider them in relation to scientific representations. There seems to be a~central feature of scientific representations that is not part of the optimization process, even though it may lead to informationally equivalent outcomes. That is, representations often miss aspects of the observable portion of the world and so they are constructed as giving an empirical picture of it but \textit{not} a~real one. This differs from optimization processes (and hence from abstract explanations) for it is not that scientists choose to exclude the irrelevant elements from the discussion (as it happens during optimization), it is that a~set of observable events of the representation's target system do not appear in the representation for reasons not directly linked to the explanation, but which are mainly related to the very nature of scientific representations, namely that they, most of the times, represent aspects of the system they target to and not every element of it. Despite the fact that the absence of observable events is documented, these representations are nevertheless used as explanations. It is therefore necessary to be mindful of this subtlety, for if we do not do so, then explanations that come from representations that miss elements of the observable world can be described as ``abstractions'' even though they are inherently constructed that way in the first place, regardless of whether they are applied to explanation or not. Of course, when a~set of representations appears as a~possible answer to a~why-question, then it may undergo an optimization process so that its most relevant parts will be used; but the ones that are eventually used are not necessarily the more ``abstract'' ones, even though when the more abstract features are used this is an optimization process.

If the above are on the right track, the ontic approach to explanation is further undercut and some epistemological concerns can be expressed on whether explanation is to appeal to realist claims about the world. Philosophers have introduced the notion of ``denotation'' which we can use to contemplate over the distinction between representations of real world entities and representations of empirical states of affairs (``real and empirical representations'' onwards), or even of hypothetical constructions that are used and represented in science. In each case, the denoting symbols of the representational schemata refer to their denotata 
%\label{ref:RNDjfcjoYIGFO}(Russell, 1956; Elgin, 2010; Salis and Frigg, 2020)
\parencites[][]{russell_denoting_1956}[][]{elgin_telling_2010}[][]{salis_capturing_2020} %
 and there are examples where scientific explanation occurs but neither the explanans nor the explanandum can claim to hold the status of being propositions that represent real objects in the world for the denoting symbols do not correspond to all parts of the target system, but only to a~portion of it. They can both therefore be credibly considered as empirical representations of the world, namely as representing parts of the observable world, albeit without being committed to bearing a~one-to-one representation of the mind-independent reality they target to. Even in case a~formal proof showing that a~representation \textit{X} of an event or phenomenon \textit{Y} is so accurate and precise that renders \textit{Y} ``real'', then, again, we need not commit to the reality of such an entity insofar as we do not observe it directly. But since most of the times scientists observe a~representational schema that denotes the existence of its target system and they do not directly observe the system in question, it seems that they work with representations and not with real world entities. If that is so, it seems that what matters when it comes to explanation is the \textit{information} transmitted by the explanatory relation to the explanndum and not whether such explanations are real and not even if the respective representations represent real entities. To illustrate this, one can examine how death-rates are explained in medicine. Both the fatality rate of a~disease, which can play the role of the explanans in these contexts, as well as the reasons a~patient or a~group of individuals died, which can be the explanandum, are in fact empirical representations and not real ones. They are both documented to miss observable parts of the world, though they are still used for explanatory purposes due to the information each representation carries.

Another problem the ontic view faces is that a~great many scientific activities contain representations of false propositions. Thought experiments for example have been characterized as mingled representations i.e. representations of empirical and hypothetical conditions that explain in context and the same is the case with similar practices such as models and computer simulations
%\label{ref:RND6qWdTjKWN2}(Karadimas, 2022).
\parencite[][]{karadimas_thought_2022}. %
 In these cases of scientific representation, the denoting symbols have no real-world denotata (at least as for the hypothetical part of the representation) and if the propositions induced by these practices are able to offer explanations, then ontological claims about explanation seem to be severely undermined for it does appear that explanation is achieved without true propositions being involved, let alone relating real objects in the world. Moreover, false propositions are not limited to mingled representations but expand to theories now considered false, but which were true in the past and which, in spite of being considered false nowadays, they can still give explanations in some contexts. Aristotle's theory for example explains better than Galileo's one the speed of falling objects in terrestrial conditions, for the latter applies only to contexts where vacuum is created 
%\label{ref:RNDIEgFfAhOy7}(Rovelli, 2015).
\parencite[][]{rovelli_aristotles_2015}.%


In order to capture such cases philosophically---which is paramount for they are central episodes in the sciences---one should focus on the epistemic side of explanation and not so much on the ontic one. If the distinction between abstract and non abstract explanations concerns us no more and all we have is representations that when applied to explanation may undergo optimization and if scientists observe representations that either miss portions of the world or induce false propositions, we can arrive at the conclusion that explanation is a~relation between representations and not a~relation between real objects in the world or even between real representations. There appear to be three main types of scientific representation: empirical, hypothetical and mingled. When the explanans are represented, the representations in question can be purely empirical (which include theories considered true as well as theories considered false) or they can be mingled. While one cannot exclude formally the possibility of being merely hypothetical ones, most of the times hypothetical constructions include elements of the empirical world and so I~take mingled representations to capture this case as well. When the explanandum is represented the representations are mostly empirical (for science typically does not engage in explanations of counterfactual or non-existent states of affairs). A~representation that attempts to answer a~why-question carries information that needs to square well with the information elicited by the propositions that represent the explanandum. Moreover, as it will be shown below, two levels of optimization can take place during explanation: crude optimization that rules out representations that are in principle relevant but become irrelevant under a~certain context, and sophisticated optimization that puts forward the most relevant propositions in each case. More episodes of scientific explanation are captured that way than via ontic approaches and, it turns out, there is little to no room for the ontic approach to present itself as a~strong philosophical viewpoint.

I~begin by showing that abstract explanations are not a~distinct category of explanation and that abstraction is only \textit{one} of the possible optimization processes that may take place. This suggests that there are no longer abstract and non-abstract explanations and so the epistemic and the ontic view both are vying to capture all of them or else they run the risk of missing all of them. I~then present two key problems the ontic view faces which the epistemic view does not: first, the fact that scientific explanation comes mostly through representations and not through direct observation and so the ontic view ends up making claims that is not in a~position to make for scientists most of the times observe representations that represent aspects of the world, not the world itself. The second hurdle is that the ontic view does not account for explanations that come from false propositions, either these are mingled propositions or empirical propositions of theories now considered false but are nevertheless explanatory. Having highlighted these drawbacks, I~present how the epistemic approach can claim to be the soundest philosophical stance on explanation by unpacking how it manages to capture nearly all instances of scientific explanation by conceptualizing them as ``relations between representations''. The conclusion is that the ontic view is rendered redundant and that the best it can do is to capture trivial cases of explanation that come through direct observation.

\section{Abstraction, representation and optimization processes}
Strevens has developed the notion of ``optimization process'' to describe the necessary procedure that needs to take place in explanations. As he puts it, a~causal explanation occurs only when the factors that make a~difference with respect to the explanandum are taken into account. These elements are the result of an optimization process that excludes factors that could make the explanation irrelevant
%\label{ref:RNDLLyTg7wnYa}(Strevens, 2011).
\parencite[][]{strevens_depth_2011}. %
 Strevens is primarily preoccupied with causal explanations but as we will see it appears that his theory on optimization process finds applications to all sorts of explanations i.e. both causal and non-causal.\footnote{Ironically enough, Strevens is often considered as an advocate of the ontic view of explanation due to his commitment to causal explanations. Here I~use Strevens's terminology to argue for the opposite, i.e. that optimization is not restricted to causal explanations, but is a~feature of explanation by and large, and that it helps us vindicate the epistemic view of explanation.}

As a~part of non-causal explanations are often considered the so-called abstract explanations. Pinckock discusses examples of such a~kind of explanations, such as the Konigsberg bridges problem, and by comparing abstract explanations to ``microphysical explanations'' concludes that the former often lead to better explanations of that explanandum
%\label{ref:RNDc7HQwlkTHR}(Pincock, 2007)
\parencite[][]{pincock_role_2007}. In Pinkcock's analysis therefore ``abstraction'' amounts to assuming away petty or confounding events of the physical world while involving in it theoretical premises and simplified assumptions about the structure of reality. Lange examines the same problem and makes the case that the attempts to explain the Konigsberg bridges problem by appealing to nomological explanations will fail because covering laws are irrelevant in this context. He puts forward a~mathematical explanation by appealing to the notion of necessity, as well as to some context-sensitive facts, such as contingent facts that co-determine the explanation
%\label{ref:RNDYM6z39b2A1}(Lange, 2013).
\parencite[][]{lange_what_2013}.%


While both these cases are considered in the literature as abstract explanations, they can be both described simply as optimization processes, in the Strevens's sense without loss of explanatoriness. As it turned out, different parts are omitted in each case: events of the physical world in Pinckock's discussion and laws of the physics in Lange's exploration of the issue. This indicates that in pursuing the optimal explanation, the focus is on the factors that will establish the relevance of the explanation and not the abstraction from events of the microphysical world per se. Hence while omitting events of the micro-world can be part of an optimization process, eliminating laws and theoretical postulates can also be. There seems to be no good reason to consider them as different classes of explanation that find different philosophical conceptualization, for in both cases it is an optimization process that takes place whereby scholars try to figure out the premises that will help them achieve explanation and to simultaneously minimize the impact of variables that could obscure the relevance of the explanation, even if such explanations are not causal ones.

One could reply that even if in Lange's approach it is laws that are excluded as irrelevant, the explanation comes from another type of ``abstract'' explanation i.e. mathematical explanation and not from a~non-abstract one, such as from a~relation between events. However, such an objection does not take into account that it is not mathematical expressions on their own that explain, but rather the mathematical expressions \textit{alongside} context-related facts---``contingent conditions'' as Lange puts it
%\label{ref:RNDTEohesAOjo}(Lange, 2013, p.506)
\parencite[][p.506]{lange_what_2013}%
---and hence it becomes unclear whether such an explanation is abstract for it does not assume away events of the micro-world, but instead it takes \textit{the most relevant of them} into account in order to explain. Since no clear threshold for abstraction is on offer 
%\label{ref:RNDRgyQZ6HDFF}(Jansson and Saatsi, 2019),
\parencite[][]{jansson_explanatory_2019}, %
 and it strikes me as if it can barely ever be one, then discriminating an abstract form a~non-abstract explanation can be a~matter of confusion. Even in case the explanation was given only by some ``abstract'' mathematical structures this would have not ratified abstract explanations as inherently different from the others for, as the examples drawn from Pinckock and from Lange show, it seems that, either the explanation is highly abstract or less-than-highly-abstract in both cases it eventually boils down to being involved in an optimization process that tries to find relevant answers to why-questions and not to an abstraction in its own right.\footnote{This seems to speak against the notion of ``idealization'' as well, which is similar to abstraction and even though they are considered not exactly identical notions, when it comes to explanation they come up with the same suggestion and so the rejection of one notion refutes the other at once. That is, some philosophers claim that abstraction is when one intentionally omits unimportant details of the world, albeit without intentionally distorting aspects of the target system that is represented, while idealization is when the target system is intentionally distorted for the sake of simplicity or clarity 
%\label{ref:RNDk62JPKOKd4}(Godfrey-Smith, 2009; Levy, 2021).
\parencites[][]{godfrey-smith_abstractions_2009}[][]{levy_idealization_2021}. %
 Speaking of their application to explanation, both notions suggest that explanation often needs to either abstract away or to distort the target system of interest in order to achieve explanation 
%\label{ref:RNDpDv3FieioN}(Love and Nathan, 2015; Potochnik, 2017).
\parencites[][]{love_idealization_2015}[][]{potochnik_idealization_2017}. %
 However, as the examples discussed here suggest, representations miss aspects of their target systems regardless of their application to explanation and so when they are used to achieve explanations, both abstraction-like representations and idealization-like ones indicate the same epistemic results: whether they put forward content that assumes away parts of the observable world (as per abstraction) or whether the content they carry distorts the target system (as idealization demands), in both cases this is no more than an optimization process that pursues the most relevant piece of information in context. Since it is the notion of ``abstraction'' and not that of ``idealization'' the one that is more central to the ontic/epistemic debate, I~will no longer consider idealization here.}

Such representational constructs (either these are highly abstract or less than that) that are used to attain explanations appear not only to be both involved in optimization processes during explanations, but they moreover seem to share some common features which they carry \textit{regardless} of their application to explanation, namely that they are imprecise representations of the parts of the world they represent. Scholars have paid close attention to the well-established fact that scientific representations are not perfect portrayals of their target systems
%\label{ref:RND3cXlv3IusE}(Frigg and Nguyen, 2017)
\parencite[][]{frigg_models_2017} %
 and that the missing parts are often events in the observable world 
%\label{ref:RNDE0NzzgEFVo}(Batterman, 2007; Potochnik, 2017).
\parencite[][]{batterman_specialness_2007}. %
 Scientists therefore may be fully aware that this is so and nevertheless accept it as a~credible representation. Consider for example the findings from serological studies i.e. from studies that attempt to measure the antibody prevalence in a~population and based on this to estimate the lethality of a~disease. The theory that guides such measurements is that the levels of immune responses against a~particular disease that exist in a~population largely determine the disease's lethality. The levels of immunity are divided with the recorded deaths from the disease in question. The number of deaths is the numerator and the number of people with estimated immune responses is the denominator; the result of this division produces the infection-fatality-ratio (IFR) estimate. Immune responses are induced by different kinds of antibodies and T-Cells and they are tracked in the blood of randomly selected individuals. If immune responses are found to a~large number of people in comparison to the number of deaths, then the IFR is low which suggests that the disease is widespread but that only a~small fraction of the infected people has died which in turn indicates that the disease is mostly harmless. Conversely, if the levels of immunity are low in comparison to the number of deaths, then the IFR is high which indicates that the disease could pose risk to a~larger segment of the population. IFRs for germs such as Sars-Cov-2 (which is said to be the virus that causes the coronavirus disease Covid-19), are constructed by measuring the antibody levels in a~population. During the Covid-19 pandemic therefore, scientists made use of some tests which were used to identify antibody-related immune responses. However, it is possible that the current antibody studies underestimate the immune responses in the population and thus overestimate the lethality of a~disease. Serological studies do not account for the T-Cell responses that are either pre-existing or are elicited after mild or asymptomatic Covid-19 and they also are structured so that they detect only IgB and IgM antibodies. They thus do not detect IgA antibodies that are also important in fighting pathogens and are also produced during infection. Those who tackle the disease through T-Cells solely or through IgA antibodies, may not develop virus-specific IgG antibodies and so the prevalence of the disease may considered to be lower than it actually is. Moreover, even if IgG antibodies are secreted, they appear to decline rapidly and so late testing may miss some cases of these antibody responses too 
%\label{ref:RNDjSz8RaoEvd}(Burgess, Ponsford and Gill, 2020).
\parencite[][]{burgess_are_2020}. %
 In spite of these downsides, scientists consider serological studies to give us a~quite reliable estimate of how lethal a~disease is, for even if they do not project with 100\% precision the lethality by somehow underestimating the levels of immunity and slightly overestimating the infection-fatality ratio (IFR) they appear to capture a~notable part of the target system, namely much of the antibody-related immune responses. Therefore the IFR of a~disease (of Covid-19 in this case) is---irrespective of its possible application to explanation---a \textit{representation} of the infected-to-dead individuals and, given the glitches that appear in identifying all immune responses, it cannot be said that via this estimate we have a~precise representation of the prevalence of the disease and hence we do not have fullaccess to the world, even though it is stated that we do have a~very good picture of how deadly a~disease is.

So it seems that representations are in no need of a~particular type of abstraction when they are used to explain, for they omit parts of the observable world they target to \textit{in the first place} and this happens \textit{both} when they represent some theoretical or mathematical postulates \textit{and} when they represent events of the world and the relations between them. Hence what is at stake when they are applied to explanation is to optimize them, namely to select the \textit{information} these representations carry that is suitable in each case.

While the discussion so far takes into account how abstraction, optimization and representation can be considered when the explanans are taken into account, but similar is the case with the explanandum. Bokulich has proposed the ``eikonic view'' of explanation in which the explanandum is in fact a~representation of its target system and different representations of it are given in different contexts
%\label{ref:RNDFXnR0YytHe}(Bokulich, 2018).
\parencite[][]{bokulich_representing_2018}. %
 While I~am in agreement with much of her analysis, the way she employs the notion of ``abstraction'' in it by claiming that the representation of the explanandum entails a~level of abstraction 
%\label{ref:RNDkTI1d1Pja5}(Bokulich, 2018, p.803),
\parencite[][p.803]{bokulich_representing_2018}, %
 is a~point that I~am taking issue with. Since explanation is inherently contextual, the representations of the explanandum will be shaped so that they will be on a~par with the purposes of the explanation as it occurs in context. Despite the fact that the classic Adam/apple example shows that not only the same representational strategies, but moreover precisely the same lexis can be used to highlight different explananda in different contexts, in more complex cases of explanation it can be assumed that different representations will be needed as well. Whichever representations are used though, whether the explanandum is to be represented with a~certain degree of abstraction this is a~matter of context and thus, again, the aspects of it that are to be represented are mainly an optimization process in which only the suitable parts of the system are included. As in the case of the explanans, the omitted factors can be theoretical postulates or unimportant (in context) events. For example, one can ask ``why countries go bankrupt?''. That question is not independent of context (hence its answer is not an abstract explanation), even though it may seem so at a~glance. The explanandum in this context requires information related to the central causes of a~country-level bankruptcy and so, it can be described as an optimization process not as an abstraction: it represents bankrupt countries via text. Of course, through textual representation the explanndum can be altered and represented in another context, thus seeking different information as an answer. In such a~case one could ask ``Why did country \textit{X} go bankrupt?'' In both examples, the target system is the historically recorded fact that countries may go bankrupt from time to time and this target system is represented in different explanatory contexts after it has undergone an optimization process so that different aspects of it are included in each case. In the former why-question the explanandum may be conceived as omitting trivial or confounding events of the world, such as the economic conditions in some countries, though in the latter it does not and is especially interested in country \textit{X}'s bankruptcy which it turn leaves open the possibility of omitting some theoretical postulates employed in the representation of the country-level bankruptcy.

Reducing abstraction to sheer optimization will turn out to be the key in the discussion on explanation and on the ontic-epistemic debate. As stated, it implies that the distinction between abstract and non abstract explanations wanes which in turn suggests that all explanations find the same philosophical conceptualization; hence either all are captured by the ontic approach or all are captured by the epistemic one. To establish that the latter is the case, we can go on to see some inadequacies the ontic view faces which will reveal not only the drawbacks of this philosophical stance but will also further solidify the epistemic approach to explanation. The first problem is related to what scientists observe when they try to offer explanations and, as I~show in the next section, they typically observe representations of the world, not the world per se.

\section{Scientists' Scope of Observation}
While this is not the place to delve into the realism-antirealism debate, it is important to reflect on what scientists observe when scientific explanations take place. First off, as it is already mentioned here and as it is widely acknowledged, it appears that scientists work with scientific representations of the world, at least in non-trivial cases of explanation. Indeed, when economists explain reduced longevity and put it down to poor economic performance in the countries that reduced life span was recorded, they work with figures representing years of life lost and economic outcomes (such as GDP fall), they do not observe directly these countries and the economic activity of their citizens and their governments. It does follow therefore that what scientists observe when attempting an explanation is a~range of representational schemata, not the ``mind-independent'' world and even if it goes without saying that scientific representations represent aspects of the world, it is also hardly questionable that scientists still have direct access to the representations, and only implicit access to the world via these representations
%\label{ref:RNDGdMuqvxKOH}(Van Fraassen, 2008, p.254).
\parencite[][p.254]{van_fraassen_scientific_2008}. %
 This is important, for even if one backs down from the epistemic view of explanation and endorses the ontic view and its central claim that explanations are relations between real objects in the world, then it is unclear how scientists can have knowledge of these relations if they reject the representational (and hence the epistemic) view of explanation. In other words, if diagrams, texts and other representations are not explanations and explanations occur in the world irrespective of the representational schemata, then those who pursue the ontic view need to show how scientists have access to these explanations. Unless this is shown, it could be argued that only scientific representations are within the scientists' scope of observation, not the real world.

Of course, I~do not wish to eliminate formally the possibility that observation of the world takes place and simultaneously non-trivial explanations are given (as for example when planets are observed), but in most cases a~representation of the world is first constructed and it is this representation and the information it carries that is then used for explanatory purposes.\footnote{However, even in such cases, the jury is still out on whether observation is direct or whether it is implicit via signals that appear on the telescope, as it happened with the observation of solar neutrinos
%\label{ref:RNDy67otqPKyL}(Shapere, 1982).
\parencite[][]{shapere_concept_1982}. %
 } When measurements in experiments take place for example, the experimental set up offers a~pile of data some of which are included in a~data-model 
%\label{ref:RNDz3zffLbFO4}(Giere, 2018)
\parencite[][]{peschard_models_2018} %
 and then, with the aid of some theoretical postulates, the elements the data model entails are interpreted and eventually the measurement outcome is represented in a~model that merges theoretical postulates and elements of the data model 
%\label{ref:RND8HXXYFcwnh}(Parker, 2017).
\parencite[][]{parker_computer_2017}. %
 If such a~representation is to be used to explain, scientists have access solely to this schema.

To further appreciate this, consider moreover an explanation of covid-induced deaths. The IFR of Sars-Cov-2 is, of course, often used to explain death rates in a~certain region or even worldwide in spite of the fact that it does not capture all immune responses of the human immune system. As for the explanandum in such a~case of explanation i.e. deaths in medicine, we need to be mindful that they barely occur as a~result of one factor and several confluent factors usually co-determine the outcome. This happens with nearly all causes of death and it happened with deaths attributed to Sars-Cov-2 as well. That is, an elderly person with pneumonia may have been also diagnosed with blood clots and heart inflammation. Either for the sake of convenience or due to some established consensus, among the several contributors typically one is pointed out as a~central cause of death which is not necessarily identical to the actual cause of death. Speaking of Covid-19, the criterion to determine cause of death was a~positive PCR test. If one dies with a~positive PCR test then is recorded as a~covid death regardless of possible co-morbidities. However, this test can find dead viral fragments and it is known that dead viral fragments do not cause illness, let alone death. But the death certificates mention ``Covid-19'' as a~cause of death even if dead viral fragments are indentified to the individual. Hence it is perfectly possible that a~number of people who were reported as dying from Covid-19 were dead due to some other causes
%\label{ref:RNDD4PC5DhpoZ}(Jefferson et al., 2020)
\parencite[][]{jefferson_are_2020} %
 thus inflating the actual number of deaths. Again, in this case, the representation misses aspects of its target system and in this respect both the death-certificates and the IFR are in accordance with how scientific representations mostly represent their target systems for both the IFR and the covid deaths can be characterized as empirical representations that represent aspects of the world they aim at, albeit without confronting themselves with the task of representing the ``real world''. It seems therefore that if one uses the IFR to explain death rates, in fact does not observe real conditions or relations between real objects but two representations and explores not the reality of the conditions each one of them describes, but whether the information induced by the explanans fits the information induced by the explanandum.

As it turns out therefore, the ontic view of explanation is seriously challenged when one considers that scientists observe representations and not the mind-independent world, and that the best part of these scientific representations are empirical and not real. However, things can get even worse for the ontic view when other cases of explanation are taken into account in which the representations used are either theories that entail propositions that are now considered false but they nevertheless explain, or a~schema in which empirical and hypothetical (namely, non-existent) state of affairs are represented, as are mingled representations, which are also, by any reasonable conception of truth, false.

\section{False Propositions, Scientific Information and the Ontic View of Explanation}
The possibility of false theories that are nevertheless explanatory was mentioned above and Aristotle's theory is not the only example of this sort. Newton's theory does explain planetary motion even though it is widely acknowledged to be a~false theory. This is a~very serious hurdle that advocates of the ontic view need to overcome, for it appears to refute the central claim of the ontic view, namely that explanations are causal relations between real objects in the world.\footnote{With the exception of Glass who recognizes that false propositions can explain, I~am so far unaware of any attempt from proponents of the ontic view (or of similar realist-leaning accounts) to even mention, let alone to address, the issue. Glass mentions that false theories explain and does not tackle the problem either, though he does confess that realist explanatory schemata such as ``inference to the best explanation'' apply only when true theories are candidates for an explanation
%\label{ref:RNDZ6NKGPe6SG}(Glass, 2021).
\parencite[][]{glass_coherence_2021}.%
} If false theories offer explanations, then it is highly questionable that explanations can be described as exhibiting explanatory relations between real objects in the world for if the objects described by the false theories do not count as real anymore then, it follows from the ontic approach itself that they do not count as explanations either. One attempt to save the ontic proposal could be to try to reconcile it with the pragmatic approach and claim that ontological commitments can be context-dependent and so objects that count as true \textit{in context} are explanatorily related thereof. According to such an attempt, false theories are considered as representing real world entities in certain contexts and not independent of them.\footnote{Rovelli 
%\label{ref:RNDTBEPz5otZb}(2015)
\parencite*[][]{rovelli_aristotles_2015} %
 makes an interesting claim that highlights strong pragmatic features not only with respect to explanation (which is not his primary concern) but with respect to science in general. Most importantly, he does not restrict his analysis to false theories, but includes also those considered true and he argues in particular that Aristotle can be found right and wrong in the same way Einstein can be found right and wrong i.e. dependent on contextual factors.} While this could be a~promising step, it becomes redundant since, when it is considered with respect to explanation, it turns out to be a~project not dissimilar to the one described above, namely it relates empirical representations (the explanans and the explanandum) and the why-question is answered either with or without ontological commitments (even if they are described as context-dependent). That is, even if one commits to the reality of the IFR or at least to the aspects it represents, the explanation of the death-rates becomes no more robust than it already is if one uses the IFR as an empirical representation that carries information that explains the death rates, for in both cases it is the information that provides the explanation. The reconciliation of the ontic and the pragmatic view therefore seems to be a~strategy which turns out to give results that are, at best, explanatorily equivalent to the ones that an epistemic approach could give.

This problem is exacerbated and the attempt to bring together the ontic and the pragmatic approach flounders when another class of false, but explanatory, propositions is taken into account which are the ones induced by prominent activities such as thought experiments. As I argue elsewhere, thought experiments (and likewise much of the modeling strategies such as computer simulations), can be described as mingled representations that carry information which explains events under several contexts
%\label{ref:RNDIcjPwwU1TW}(Karadimas, 2022).
\parencite[][]{karadimas_thought_2022}. %
 The explanatoriness of the mingled representations as well as of the representations that false theories give us is established by the content they carry which is \textit{scientific information}. Scientific information is in principle explanatory relevant 
%\label{ref:RNDIxJYVaF1UO}(Van Fraassen, 1980; Richardson, 1995)
\parencites[][]{van_fraassen_scientific_1980}[][]{richardson_explanation_1995} %
 and false propositions which carry scientific information are also in principle explanatory and while they may become irrelevant under some contexts, they cannot be excluded from explanation simply because they are not real world objects and they do not even represent relations between such objects. Moreover, one cannot discriminate them as being abstract non-causal explanations that capture different cases of explanation from singular (aka causal) ones. While mingled representations are non-causal explanations, they are not ``abstract'' explanations as opposed to singular (i.e. causal) ones for, they both share the standard features scientific representations have; mingled propositions denote in part a~hypothetical and in part an empirical state of affairs, while empirical propositions denote empirical aspects of the world. When mingled representations are used as explanations, they convey information that is at once hypothetical and empirical and when empirical representations are used they transmit information that describes a~causal explanatory relation between the target systems. Both are on a~par with the epistemic approach and at odds with the ontic view.

We have eliminated the distinction between abstract and non-abstract explanations and we have moreover underscored two central inadequacies the ontic view faces which in turn results in it being unable to hold the status of being the philosophical stance that captures all episodes of explanation. On the contrary, the epistemic view fares much better for it manages to encompass all sorts of explanation by conceptualizing them as relations between representations.

\section{A~Relation between Representations}
Different explanatory relations in different contexts are, indeed, a~possible outcome and the long debate in philosophy of science over the relevance relation is still ongoing. The standard theories are the law-like which was introduced by Hempel and Oppenheim
%\label{ref:RNDN8njICbLgt}(1948),
\parencite*[][]{hempel_studies_1948}, %
 the unification, by Friedman 
%\label{ref:RNDe7uDGGd0Mq}(1974)
\parencite*[][]{friedman_explanation_1974} %
 and Kitcher 
%\label{ref:RNDeItgQ4L0PZ}(1981)
\parencite*[][]{kitcher_explanatory_1981} %
 and causal explanation which has been shaped in diverse forms, such as causal/mechanistic 
%\label{ref:RNDAgHoEmWc9z}(Salmon, 1984; Machamer, Darden and Craver, 2000),
\parencites[][]{salmon_scientific_1984}[][]{machamer_thinking_2000}, %
 or counterfactual causality 
%\label{ref:RNDBeiToeOa7V}(Woodward, 2002),
\parencite[][]{Woodward2002}, %
 though whichever strand one takes into account, the gist is that explanation needs to describe the causal structure of the world. There are also, as already discussed, non-causal theories of explanation, which focus on how mathematical explanations can be given, as proposed by Lange 
%\label{ref:RNDr1ONC1cPsi}(2013),
\parencite*[][]{lange_what_2013}, %
 or how the mingled propositions can explain under several contexts. All theories of explanation can be captured conceptually by appealing to the epistemic view and to its representational side.

Two main classes of representations can be used to that end; empirical representations and mingled representations. Explanatory relations bear information to the explanandum via one of these categories of representations. The information is scientific and can be law-like, unificatory, causal or non-causal. In any case, it is transmitted through a~representation and it targets another representation, namely the explanandum. Each representation can be conceived as a~set of propositions that carry scientific information. The denoting symbols of the explanans is related with the denoting symbols of the explanandum.\footnote{Of course, as briefly mentioned above, scientific representations include a~great deal of denoting symbols that are often called as ``representational tools'', ranging from texts to diagrams or mathematical expressions. Since the focus here is on the content these representations carry and not on the way it is transmitted, I~will not discuss them further.} The relation is not ambivalent and thus the propositions of the explanans target the ones of the explanandum and not the converse. It moreover appears that there are two stages of optimization in explanations; the first is when among several sets of scientific propositions only those that are in principle relevant are considered and then the second stage whereby among the ones in principle relevant, scholars determine the explanatory relevant ones. Call the former ``crude optimization'' and the latter ``sophisticated optimization''. Crude optimization rules out non scientific propositions and scientific theories that are irrelevant to the context, such as, for instance, physical theories from an explanation in economics and brings to the fore several sets of explanations that are in principle relevant.\footnote{Some current trends in modeling strategies work on the assumption that a~unified mathematical picture of events in physics and economics is possible. However, even if a~mathematical representation of these events can appear in a~common schema, it does not follow that this can be explanatory, for the information required for an explanation in economics is radically different from an explanation required in physics and vice versa
%\label{ref:RND2XO8pNrRrf}(Woodward, 2016, p.125).
\parencite[][p.125]{woodward_unificationism_2016}.%
} Sophisticated optimization amounts not only to figuring out the most relevant set of those that passed the crude optimization, but also determines which relation among the propositions in the set in question can be used as an answer to the why-question. Maybe they can be all of them or some maybe required to be ruled out. The ones that are eliminated in the optimization process can be either laws of high generality, as it can happen when a~causal relation is established, or irrelevant events of the empirical world, as for example it could happen when law-like or unificatory explanations are given. As for the set of representations that represent the explanandum, they are also open both to crude and to sophisticated optimization which make specific which parts of the world require an explanation.

The representation-relation between the explanans and the explanandum can be formalized and illustrated. Let an empirical representation ($RE$) to include a~set of empirical propositions $\{RE_{1}, RE_{2}, RE_{3}…RE_{n}\}$, and likewise mingled representations $RM =\{RM_{1}, RM_{2}, RM_{3}\ldots RM_{n}\}$. Finally let the explanandum ($REx$) also to be a~set of empirical propositions $\{REx_{1}, REx_{2}, REx_{3}… REx_n\}$. The explanation occurs as shown in Fig.\ref{karadimas:fig1}. From the parts of the world to be explained ($REx, REx'$ ) crude optimization of the explanandum rules out the parts of it that are not of primary concern in context ($REx'$) and puts forward the ones that we are interested in explaining ($REx$). Then sophisticated optimization makes precise which aspects of this part of the world will be the explanandum. As shown in Fig.\ref{karadimas:fig1}, from a~set of propositions that represent aspects of this part of the world, only $REx_{1}$ and $REx_{3}$ turn out to be of particular interest and so they serve as the explanandum (both crude and sophisticated optimization are represented by the dotted upward arrows). Likewise both stages of optimization occur when explanations to these empirical representations are pursued. The in principle relevant answers are $RE$ and $RM$, while only $RE$ makes it through the crude optimization process. From the propositions entailed in $RE$ sophisticated optimization rules out the ones that are unfit for purpose in that context and makes use of two of them($RE_{1}$ and $RE_{3}$) in order to explain $REx_{1}$ and $REx_{3}$ (both crude and sophisticated optimization are now represented by the dotted downward arrows). Explanation eventually takes place when the propositions that made it through the optimization of the explanans and the optimization of the explanandum are related and in fact when the explanans target the explanandum (as represented by the rightward arrow).\footnote{Note that Fig.\ref{karadimas:fig1} is only in part used to illustrate how the relevance of each explanatory proposition is determined for this could require further elaboration that goes beyond the purposes of the current analysis; it is mostly confronted to showing that explanation is, for its best part, a~relation between diverse representational schemata and that optimization processes are involved in this relationship.} Such a~representation-relation schema enhances the epistemic view and cuts against the ontic conception of explanation.



\begin{figure}[H]
\centering
\begin{tikzpicture}[>=Stealth,
                    every node/.style={align=center},
                    node distance=2cm, scale=0.8, every node/.append style={transform shape}]

    % Nodes
    \node (inPrinciple) {\textbf{In principle relevant answers: \textit{RE, RM}}};
    \node[below=of inPrinciple] (reSet) {RE=$\{RE_1, RE_2, RE_3 \ldots RE_n\}$};
    \node[below=of reSet] (reReduced) {RE=$\{RE_1, RE_3\}$};
    \node[right=of reReduced] (reEx) {$Rex=\{REx_1, REx_3\}$};
    \node[below=of reEx] (reExFull) {$REx=\{REx_1, REx_2, REx_3 \ldots REx_n\}$};
    \node[below=of reExFull] (toExplain) {\textbf{Parts of the world to be explained:\textit{ Rex, Rex'}}};

    % Arrows and Labels
    \draw[->, dashed] (inPrinciple) -- node[right] {\textit{Crude Optimization}} (reSet);
    \draw[->, dashed] (reSet) -- node[right] {\textit{Sophisticated Optimization}} (reReduced);
    \draw[->] (reReduced) -- node[below] {\textit{Explanation}} (reEx);
    \draw[<-, dashed]  (reEx) -- node[right] {\textit{Sophisticated Optimization}} (reExFull);
    \draw[<-, dashed] (reExFull) -- node[right] {\textit{Crude optimization}} (toExplain);

\end{tikzpicture}
\caption{The representation-relation between the explanans and the explanandum.}
\label{karadimas:fig1}
\end{figure}


To make the latter point clearer and to showcase how different stages of optimization take place in practice, we can consider a~more expanded version of the IFRs' example discussed above. Suppose that experts are interested in explaining increased death rates from Sars-Cov-2 in the elderly population in the EU over the period 2020--2022 in comparison to Asian countries whereby deaths in that age-group did not soar and remained in the same ballpark to pre-pandemic levels. As ``elderly population'' is defined---as a~matter of expert consensus---the chunk of the population that is over the age of 70. The crude optimization with respect to the explanandum therefore entails eliminating as possible explananda individuals across all age groups who died from other causes \textit{across the globe} while sophisticated optimization rules out individuals \textit{in the EU and in Asia} who are below 70 and who nevertheless died of Covid-19 as well as people over 70 \textit{in the EU and in Asia} that died from other causes. There are several explanations on offer that can in principle explain such a~spike in deaths from Sars-Cov-2: the IFR of Sars-Cov-2 which is orders of magnitude greater for those over 70 in comparison to those below 70 years old
%\label{ref:RNDoEjGI2EWrg}(Axfors and Ioannidis, 2022).
\parencite[][]{axfors_infection_2022}. %
 Another explanation pertains to the levels of pre-existing T-Cell immunity: high levels of pre-existing cellular immunity in Asian countries were documented 
%\label{ref:RNDrU4d2oVGVr}(Bolourian and Mojtahedi, 2021)
\parencite[][]{bolourian_covid-19_2021} %
 which suggests that herd immunity was developed there even prior to the advent of Sars-Cov-2 
%\label{ref:RNDijz5iVum0v}(Le Bert et al., 2021)
\parencite[][]{le_bert_highly_2021} %
 thus making it difficult for the virus to infect the vulnerable groups whereas in the EU the levels of pre-existing immunity were quite low and so it was easier for the virus to spread and infect people over 70. A~third explanation could be that the average lifespan in the EU is higher than in Asia and so it was expectable to have more deaths in that age group. A~fourth explanation could put increased deaths down to economic outcomes: it is known that one of the avails of economic development is that healthcare facilities are better organized and thus offer high quality treatment to patients thereby saving lives. Other in principle relevant explanations can be put forward, but to keep things simple let us assume that only these are available. Crude optimization with respect to the explanans rules out the latter for EU is having, on average, a~stronger economy than the majority of Asian countries and so, while such an explanation is in principle relevant, it becomes irrelevant under this context. It could also rule out the discrepancy in the average lifespan between the EU and Asia, for even if this is a~matter of fact in some countries such as Afghanistan, there are Asian countries with life expectancy similar to the EU (if not higher than that), as for example Japan and Singapore, and in which the virus-related deaths did not spike 
%\label{ref:RNDwcEF9xIMlm}(Karadimas, 2023, pp.34–35).
\parencite[][pp.34–35]{karadimas_covid-19_2023}. %
 Hence we are left with two possible explanations: Sars-Cov-2's IFR and pre-existing immunity. Sophisticated optimization will figure out which of the two carries information that best explains the explanandum or whether both, combined, offer the most relevant answer.

It appears that the ontic view struggles with such instances of explanation---which are quite common in science---for they come solely through representations and not by relating through direct observation real-objects in the world. Researchers have access to diverse representational schemata such as the IFR, the levels of pre-existing immunity and the average lifespan in several countries which all three constitute the explanans, as well as the death rates of interest, i.e. people over 70 in the EU and in Asia which are the representations that induce the explanandum, all of which (the explanans and the explanandum alike), miss, as we already discussed above, observable aspects of their target systems, and in spite of that, they are used as explanations by relating via optimization, the information each one of them carries. An ontic approach could require directly observing all the involved variables that constitute the target systems of the explanans and the explanandum which would include observation of variables of a~set of events that no human being can seriously claim to be able to observe ranging, for example, from infected individuals over 70 in the EU and in Asia, monitoring them till they die to causally relate each one's death with the virus, to the immune responses of all people in these two regions in order to determine pre-existing immunity and similarly observing everyone who dies regardless of cause so that the average lifespan will be estimated. In other words, barring the presence of a~superhuman that could manage to observe all these events,\footnote{Even assuming an extant and omni-observant entity from 2020--2022, since this is a~non-trivial case of explanation (and most instances of scientific explanation are non-trivial) and thus it involves many variables that moreover span a~wide range of locations, then this individual needs to come up with a~set of representations which could summarize their observation-based findings which in turn could likely make the explanation epistemic and not ontic for explaining via representations is what lies at the heart of the epistemic approach to explanation and it seems that even such a~skilled entity could end up accessing representations and explore the information each one of them carries in order to provide an explanation. Moreover, we cannot be sure that such representations include everything that occurred in the world for they, at best, would represent what the superhuman observed but it does not follow that what they observed is identical to what in fact occurred (for example some could have developed specific T-Cell responses via infection with Sars-Cov-2 and the superhuman could consider them as members of the group with pre-existing immunity since, unless antibodies are also detected, it is indistinguishable whether cellular responses came from cross-reactive immunity or via infection with the virus in question) and so even such representations are better conceptualized as empirical and not as real.} proponents of the ontic approach need to develop a~theory showing how access to the target systems that are involved in explanations is attained without appealing to representations and so far they have not done so. Indeed, as things are, access to the ``mind-independent'' world is attained through a~set of representations which in turn greatly weakens the ontic conception of explanation.

\section{Conclusion: Is there room for the ontic view?}
The representation-relation and the focus on the relevance of the information transmitted from the one representation to another, seems therefore to square well with much of scientific practice and with much of the philosophical theorizing on explanation, thereby bolstering the epistemic approach to explanation which seems to be rendered the prevailing philosophical view. This begs the question: is there room for the ontic view of explanation?

It can be argued that the analysis here indicates that the applicability of the ontic view is severely limited. The discussion vis-à-vis abstraction, optimization and representation and its central conclusion, namely that abstraction can be part of an optimization process and is not a~distinct kind of explanation, serves as a~basis for making the case that since abstract explanations are not to be considered a~special case of explanation, the ontic view can no longer rely on the abstractness of a~particular class of explanations in order to demonstrate its ability to capture singular ones, like the causal explanatory relations, and so it needs to capture all episodes of explanation or miss all of them at once. However, the inability of the ontic view to capture instances of explanation induced by false propositions and the fundamental epistemic problem of trying to establish explanatory relations between objects that fall outside the scientists' scope of observation, reveal problems that are avoided if explanation is considered from an epistemic point of view which implies that the ontic view is far from capturing all episodes of explanation and that it becomes highly questionable if it can conceptualize even some of them.

To be sure, it cannot be officially abolished from explanation since it is still possible to capture some cases of explanation in which directly observed objects or entities can be said to be causally related. However, such explanations are mostly trivial examples of explanation and advocates of the ontic view bear the burden of showing that this is not the case. Even if such an account is offered, it is unclear that such an explanation will be superior, and thus more relevant, to a~competing one that could capture such an episode via representations. Thus scientific explanation can be said to be in line with the epistemic conception and not that much with the ontic one.

\section{Acknowledgments }
I~am grateful to two anonymous referees and to Cory Wright for constructively engaging with the manuscript.

\end{artengenv}
