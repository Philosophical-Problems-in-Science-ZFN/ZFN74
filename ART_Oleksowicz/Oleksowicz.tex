\begin{artengenv}{Michał Oleksowicz}
	{Ontic or epistemic conception of explanation: A~misleading distinction?}
	{Ontic or epistemic conception of explanation\ldots}
	{Ontic or epistemic conception of explanation: A~misleading\\distinction?}
	{Nicolaus Copernicus University in Toruń, Poland}
	{In this paper, I~discuss the differences between ontic and epistemic conceptions of scientific explanation, mainly in relation to the so-called new mechanical philosophy. I~emphasize that the debate on conceptions of scientific explanation owes much to the Salmon's ontic-epistemic distinction, although much has changed since his formulations. I~focus on the interplay between ontic and epistemic norms and constraints in providing mechanistic explanations. My conceptual analysis serves two aims. Firstly, I~formulate some suggestions for recognising that both sets of norms and constraints, ontic and epistemic, are necessary for scientific theorising. Secondly, I~emphasize that there are multiple dimensions involved in scientific explanation, rather than clear-cut alternatives between ontic and epistemic aspects. I~conclude with a~general observation that although contextual aspects of explanations are unavoidable, the epistemic-relativity of our categories, explanations and models can in fact be compatible with their objectivity. Instead of making hastily drawn ontological implications from our theories or models, we should carefully scrutinize them from the ontic-epistemic perspective.
	}
	{new mechanical philosophy, mechanistic explanation, ontic, epistemic, explanatory norms, explanatory constraints.}



\section{Introduction}
\lettrine[loversize=0.13,lines=2,lraise=-0.03,nindent=0em,findent=0.2pt]%
{T}{}he word ``explain'' is used in very different contexts. Explaining some phenomenon involves performing operations on its representations to understand the ``how'' or ``why'' of this phenomenon. The explanation is then a~matter of representing what depends upon what. In this paper, I~want to explicate the difference between the ontic and epistemic conceptions of scientific explanation (OC and EC, respectively), mainly linked to the so-called new mechanical philosophy (NMP). For those who are not familiar with the latter approach, I~will briefly mention that the NMP is a~novel revision of Old Mechanism that takes on theoretical problems from the last fifty years of post-logical empiricist philosophy of science. It is particularly focused on the issue of causal explanations of natural phenomena and offers an overview of various methodologies employed in different sciences
%\label{ref:RND74FIW6iotY}(Andersen, 2014a; 2014b).
\parencites*[][]{andersen_field_2014}[][]{andersen_field_2014-1}. %
 According to the new mechanists, the idea of mechanisms as complex causal systems in the world, and mechanistic explanations (MExs) as tools for discovering such complex systems, are both crucial for understanding natural phenomena 
%\label{ref:RNDjSxMp85xnl}(Machamer, Darden and Craver, 2000).
\parencite[][]{machamer_thinking_2000}. %
 In the beginning, NMP aimed at examining the causal ``talk'' within different sciences and discussed the normative properties that a~good explanation ought to have 
%\label{ref:RNDceJQxBwuki}(Craver, 2014).
\parencite[][]{kaiser_ontic_2014}. %
 New Mechanists undertook both these aspects, focusing on the mechanisms in their ontic and epistemic aspects, i.e., as real causal systems in the world and representations of worldly things 
%\label{ref:RNDAHeG3iVeCx}(Bechtel and Abrahamsen, 2005; Craver, 2006; Darden, 2008; Campaner, 2013; Levy, 2013; 2014).
\parencites*[][]{bechtel_explanation_2005}[][]{craver_when_2006}[][]{darden_thinking_2008}[][]{campaner_mechanistic_2013}[][]{levy_three_2013}[][]{levy_machine-likeness_2014}.%


In what follows, I~will briefly introduce the origins of the NMP's distinction between the ontic and epistemic conception of explanation. Those who are proponents of NMP literature can skip this section and go directly to the second section, in which I~analyze the debate over the ontic and epistemic norms and constraints of explanation. In the third section, I~discuss the solutions offered by Illari
%\label{ref:RNDJvTWH0oAjl}(2013),
\parencite*[][]{illari_mechanistic_2013}, %
 Kästner 
%\label{ref:RNDLzIdoXiM3k}(2018),
\parencite*[][]{kastner_integrating_2018}, %
 and Kästner and Haueis 
%\label{ref:RNDlMEupynbdN}(2021)
\parencite*[][]{kastner_discovering_2021} %
 to the long-lasting opposition between OC and EC. In the fourth section I~point out further problems linked with this debate, i.e., the ambiguity of the term ``mechanism'', and I~articulate a~dual ontic-epistemic approach, showing in what sense it may benefit for the philosophical approach to discuss the conceptions and accounts of scientific explanation. I~conclude with general observation that although contextual aspects of explanations are unavoidable, the epistemic-relativity of our categories, explanations and models can be compatible with their objectivity. Instead of hastily drawing out ontological implications from our theories or models, we should carefully scrutinize them from an ontic-epistemic perspective.

\section{New wave of ontic and epistemic conceptions}
W. Salmon, in his analysis regarding scientific explanation, points out that OC originated with José Alberto Coffa, who was:

\myquote{
a~staunch defender of the ontic conception of scientific explanation, and his theory of explanation reflects this attitude. For Coffa, what explains an event is whatever produced it or brought it about. […] The linguistic entities that are often called ‘explanations' are statements reporting on the actual explanation. Explanations, in his view, are fully objective and, where explanations of nonhuman facts are concerned, they exist whether or not anyone ever discovers or describes them. Explanations are not epistemically relativized, nor (outside of the realm of human psychology) do they have psychological components, nor do they have pragmatic dimensions
%\label{ref:RNDNpnokjkZEU}(Salmon, 1989, p.133).
\parencite[][p.133]{salmon_four_1989}.%
}

This conception was further developed by W. Salmon, who at the same time wavered between two ways of thinking about it
%\label{ref:RNDWzaMx2i7ps}(Bokulich, 2016, p.262):
\parencite[][p.262]{bokulich_fiction_2016}: %
 whether explanations exist in the world or whether they are something that reports such facts 
%\label{ref:RNDtJmP8psE1q}(Salmon, 1989, p.86).
\parencite[][p.86]{salmon_four_1989}. %
 Without entering into the historical complexities of the development of the OC and EC to the present, it suffices to say that Salmon further contrasted the OC with EC. In fact, he mainly situated his philosophical focus on explanation against C. Hempel's account. For Salmon ``two grand traditions of scientific explanation'' 
%\label{ref:RNDIgF8vp97Mu}(Salmon, 1989, pp.68–69)
\parencite[][pp.68–69]{salmon_four_1989} %
 are: the EC, characterized by its focus on logic and laws, according to which the act of explanation is to show that a~phenomenon fits into nomic nexus (generally identified with Hempel's covering model of explanation); and the OC, characterizing causality and explanation as a~causal-mechanical explanation, fitting phenomena into natural patterns and regularities 
%\label{ref:RNDRFIXyIzGaT}(Salmon, 1984, pp.84–134; 1989, pp.320–330).
\parencites[][pp.84–134]{salmon_scientific_1984}[][pp.320–330]{salmon_four_1989}.%


The OC originated with work of J. Coffa, who had no interest in the discussion on mechanisms, but it was further elaborated by W. Salmon, who was directly engaged in formulating causal-mechanical account of the OC. Although Salmon himself had a~conception of mechanisms which at first glance does not comport with conceptions of Glennan, Craver, Bechtel, Darden, Illari, Kästner, etc., it nevertheless seems that the mechanistic revival is deeply indebted to his philosophical approach
%\label{ref:RND4IDeErRSZ7}(Campaner, 2013).
\parencite[][]{campaner_mechanistic_2013}. %
 For instance, Salmon's theory already pointed out the crucial role of such notions as production and interaction, the distinction between constitutive and etiological aspects of causal explanation and the usefulness of counterfactuals if interpreted experimentally. Although further nuances of Salmon's view on scientific of explanation are not the aim of my examination here, it is essential to emphasize that Salmon's discussion of OC and EC have profoundly influenced the content of the new mechanistic debate on the metaphysics of explanation.

Among proponents of OC can be included W. Salmon, C. Craver, L. Darden, S. Glennan, P. Illari, M. Povich, T. Knuuttila. Let us now focus on core aspects of OC. While L. Darden
%\label{ref:RNDn3czZPo3Dg}(2008, p.959)
\parencite*[][p.959]{darden_thinking_2008} %
 argues that ``mechanism is sought to explain how a~phenomenon is produced, how some task is carried out, or how the mechanism as a~whole behaves'', S. Glennan 
%\label{ref:RND5qCJp8uNuc}(2002, p.S348)
\parencite*[][p.S348]{glennan_rethinking_2002} %
 argues that ``the explanation lies not in the logical relationship between these descriptions [of the parts of mechanisms] but in the causal relationships between the parts of the mechanism that produce the behaviour described''.C. Craver 
%\label{ref:RNDy7NMKjweQt}(2007, p.22)
\parencite*[][p.22]{craver_explaining_2007} %
 asserts that ``the explanandum is the release of one or more quanta of neurotransmitters in the synaptic cleft. The explanans is the mechanism linking the influx of Ca\textsuperscript{2+} into the axon terminal''. In another place, C. Craver suggests that ``all higher-level causes are fully explained by constitutive mechanisms'' 
%\label{ref:RNDh8JintwEZd}(Craver and Bechtel, 2007, p.548).
\parencite[][p.548]{craver_explaining_2007}.%
What seems to be common to the above claims is that scientific explanations, conceived in an OC manner, are mechanisms existing in the world. Thus, these explanations are not constituted by sentences, diagrams, models, but by fully objective worldly facts. The most explicit advocate of OC is C. Craver. He defends it in the following words:

\myquote{
Conceived ontically, however, \textit{the term explanation refers to an objective portion of the causal structure of the world, to the set of factors that produce, underlie, or are otherwise responsible for a~phenomenon. Ontic explanations are not texts; they are full-bodied things.} They are not true or false. They are not more or less abstract. They are not more or less complete. They consist in all and only the relevant features of the mechanism in question. There is no question of ontic explanations being ``right'' or ``wrong,'' or ``good'' or ``bad.'' They just are
%\label{ref:RNDNI89Ekqy6j}(Craver, 2014, p.40 italics added).
\parencite[][p.40 italics added]{kaiser_ontic_2014}.%
}

The crux of the problem, clearly expressed in the quote above, consists in the fact that some advocates of OC begin with the distinction between representations and worldly mechanisms that are represented, but then they claim that the term ``explanation'' refers to both the depiction of the things in the world and to things in the world. However, talking about ions in the world and talking about representations of ions in the world is not the same thing. In fact, ``what our understanding proceeds ‘through' are the representations and models of those entities and activities and the ratiocinative procedures thereon---not the activities and entities themselves''
%\label{ref:RND9UVrQMYMbS}(Wright, 2015, p.26).
\parencite[][p.26]{wright_ontic_2015}. %
 In other words, identifying explanations with the causes themselves is not only not self-evident 
%\label{ref:RNDlDFUH4k6tb}(Wright and Van Eck, 2018),
\parencite[][]{wright_ontic_2018}, %
 but confusing. The source of such confusion seems to stem from the attempt to sanction the dependence of OC on how the world is 
%\label{ref:RNDH8F40Vks7V}(Craver, 2014).
\parencite[][]{kaiser_ontic_2014}. %
 But such a~dependence is merely postulated. In reality, there is no conception of explanation that denies this sort of dependency. Any view of scientific explanation that takes explanations to be directed at or about anything at all will be compatible with such commitment. Explanations are about the world, and thus dependent on how it is. This is hardly the special feature of the OC and does not in any way distinguish ontic from non-ontic conceptions. H. de Regt 
%\label{ref:RNDPoLkvj8Pau}(2017, p.24)
\parencite*[][p.24]{de_regt_understanding_2017} %
 rightly argues that ``Salmon's distinction is misleading: explanations, including Salmon's causal-mechanical ones, are always epistemic and not ontic, in the sense that they are items of knowledge''. Any explanation seems thus to be an epistemic item or an argument in the broad sense.

Among the main defenders of EC we can find W. Bechtel, B. Sheredos, C. Wright, A. Bokulich, A. Levy, M. Nathan, D. van Eck, R. Frigg, H. de Regt. For proponents of EC, MExs are not things existing in the world but something that reports facts about things in the world. For instance, the tools of EC are descriptions, texts, diagrams or models that provide understanding on how mechanisms are responsible for certain phenomena. Wright and Bechtel
%\label{ref:RNDHKiOvm65T3}(2007, p.51)
\parencite*[][p.51]{wright_mechanisms_2007} %
 rightly argue that ``explaining refers to ratiocinative practice governed by certain norms that cognizers engage in to make the world more intelligible; the non-cognizant world does not itself so engage''. Bechtel and Abrahamsen 
%\label{ref:RNDaW8FAWhmC8}(2005, p.425)
\parencite*[][p.425]{bechtel_explanation_2005} %
 echo the previous claim, emphasizing that:

\myquote{
it is crucial to note that \textit{offering an explanation is still an epistemic activity and that the mechanism in nature does not directly perform the explanatory work}. \textit{Providing explanations, including mechanistic explanations, is essentially a~cognitive activity}. This is particularly obvious when one considers incorrect mechanistic explanationsin such a~case one has still appealed to a~mechanism, but not one operative in nature […] Thus, \textit{since explanation is itself an epistemic activity, what figures in it are not the mechanisms in the world, but} \textit{representations} of them.'' [italics added]
}

The EC stresses the fact that explanations are cognitive activities. For this reason, they highlight the ways in which science tries to grasp the \textit{explanandum}. The object of explanation is never a~mechanism \textit{simpliciter}, but the \textit{explanandum} is always embedded within a~broader explanatory context. The latter can be understood as the arrangement of instruments, scientific concepts and models, skills and activities of scientists engaged in the research programme aimed at explaining certain phenomena.

Considering the impressive development of the literature and studies dedicated to the modeling view of science, mainly are trying to answer the question of how to understand, provide, and evaluate scientific theories, laws, statements, and models
%\label{ref:RNDeAECQ8FUf6}(Meheus and Nickles, 2009; Frigg, 2022).
\parencites[][]{meheus_models_2009}[][]{frigg_models_2022}. %
 One might think that the debate between OC vs EC has been settled. In the last decade, however, the debate has shifted from the question ``what is an explanation'' to the \textit{querrelle} on ontic and epistemic norms and constraints on good MEx 
%\label{ref:RNDC8qku7xKqN}(Illari, 2013).
\parencite[][]{illari_mechanistic_2013}. %
 Such a~shift means that philosophers are focused on the question about the kinds of norms and constraints that guide MEx. There is the consensus that scientific explanation is the epistemic phenomenon under which agents develop hypotheses or models and reason under assumptions in very specific contexts. Thus, the concern is not about what scientific explanations are, but what the criteria of good scientific explanations are. We now enter into the intra-epistemic debate on norms and constraints of MEx.

\section{Norms and constraints}
From the beginning, it is important to distinguish a~``conception of explanation'', understood as a~view about what explanations are, from an ``account of explanation'', conceived of as a~view about how explanations work
%\label{ref:RNDNNRohqZNOX}(Bokulich, 2016, p.263).
\parencite[][p.263]{bokulich_fiction_2016}. %
 For instance, one can reject the OC, but at the same time endorse that many explanations are causal. It is very useful to distinguish between norms and normative constraints when looking at philosophical accounts of explanation. I~will apply the same notions that L. Kästner and P. Haueis used in their recent paper 
%\label{ref:RND3mnXNTOoaa}(Kästner and Haueis, 2021).
\parencite[][]{kastner_discovering_2021}. %
 According to them norms can be ``understood as general instructions of how to search for mechanism and how to construct good mechanistic explanatory texts'' 
%\label{ref:RNDcPwJammkki}(Kästner and Haueis, 2021, p.1638).
\parencite[][p.1638]{kastner_discovering_2021}. %
 Furthermore, ``ontic and epistemic norms can be achieved by using specific normative constraints. Different such constraints are the determinates of the determinable epistemic norm of intelligibility or the ontic norms of accuracy and completeness, respectively'' 
%\label{ref:RNDT0Ca9FcjTQ}(Kästner and Haueis, 2021, p.1638).
\parencite[][p.1638]{kastner_discovering_2021}. %
 Briefly put, while norms work as the general instructions for successful mechanistic inquiry, the normative constraints determinate the latter by limiting the search space for mechanisms in different ways.

Kästner and Haueis give the example of an ontic norm in the instruction to describe the causal structure of a~mechanism. In the case of an epistemic norm, they point out the need to increase the intelligibility of the \textit{explanandum}. How can one justify the importance of such ontic or epistemic norms? For instance, Craver stresses ``the fact that an explanation that contains more relevant detail about the responsible ontic structures are more likely, all things equal, to be able to answer more questions about how the system will behave in a~variety of circumstances than is a~model that does not aim at getting the ontic structures that underlie the phenomenon right''
%\label{ref:RNDvVNoIl63VN}(Craver, 2014, p.41).
\parencite[][p.41]{kaiser_ontic_2014}. %
 In other words, according to Craver, when we follow the ontic norm of parsing the causal structure of the phenomena, we are far more likely to provide a~\textit{bona fide} understanding of the \textit{explanandum}. In the case of epistemic norms, Sheredos argues that generality and systematicity are two prototypical epistemic norms, since they make ``intelligible any explanation's scope [i.e., explanations in a~communicative, textual and cognitive sense], unifying explanatory practices, and facilitating research and testing by delineating a~category of cases to which any explanation is presumed applicable'' 
%\label{ref:RNDBHURC5UYKi}(Sheredos, 2016, p.933).
\parencite[][p.933]{sheredos_re-reconciling_2016}.%


Let us now focus more specifically on how ontic and epistemic norms can be determined through the use of ontic or epistemic normative constraints. In the case of ontic constraints, the fundamental ones are: 1) spatial or temporal constraints, 2) the mechanism-to-model-mapping (3M) constraint. Since ``the entities and activities in a~mechanism are organized spatially, temporally and actively such that they produce the phenomenon''
%\label{ref:RNDqJsQrmOo57}(Craver and Darden, 2013, p.20)
\parencite[][p.20]{craver_search_2013} %
 it is crucial to regard the spatial and temporal organization of mechanisms. The first one consists in locations, sizes, shapes, and orientations of components; while the second one pertains to the orders, rates, and durations of stages. Both spatial and temporal constraints are then crucial for the identification of parts and activities of mechanisms. Apart from the spatiotemporal constraints, one may opt for determination of ontic norms by the use of 3M constraint. D. Kaplan and C. Craver specify the 3M requirement in the following words:

\myquote{
(a) the variables in the model correspond to components, activities, properties, and organizational features of the target mechanism that produces, maintains, or underlies the phenomenon, and (b) the (perhaps mathematical) dependencies posited among these variables in the model correspond to the (perhaps quantifiable) causal relations among the components of the target mechanism''
%\label{ref:RNDCXoSzvv47r}(Kaplan and Craver, 2011, p.611).
\parencite[][p.611]{kaplan_explanatory_2011}.%
}

This quote suggests that explanation should rely on ontic constraints to help in providing a~good causal explanation. That it is not to be fulfilled \textit{simpliciter} is obvious, if one considers that explanatory accounts are strongly idealized and contain falsehoods
%\label{ref:RNDDeQ480t60R}(Potochnik, 2017).
\parencite[][]{potochnik_idealization_2017}. %
 Being aware of the puzzling character of idealization or the explanatory incompleteness of models, Craver and Kaplan, offer a~more nuanced view of the completeness and accuracy constraints. They call it ``the ontic notion of Salmon-completeness'' and define this norm as follows: ``The Salmon-complete constitutive mechanism for \textit{P} versus \textit{P{\textasciigrave}} is the set of all and only the factors constitutively relevant to \textit{P} versus \textit{P{\textasciigrave}}'' 
%\label{ref:RNDm3UiS93oRM}(Craver and Kaplan, 2020, p.300).
\parencite[][p.300]{craver_are_2020}. %
 This ontic norm does not imply that a~model of phenomena has to be complete. Not ``all details are necessary'', but only those explanatory relevant. The ontic norm of Salmon-completeness itself points out that scientific explanation should be precise about a~given \textit{explanandum} via expressing it in contrastive terms (\textit{P} versus \textit{P{\textasciigrave}}). However, as formulated so, the ontic sense of including in explanation everything that makes a~difference to the precise phenomenon in question, does not fulfil ``ontic'' within the OC. In the latter case ``ontic'' was referring to the fact that the things in the world do account for phenomena. The ontic norm of Salmon-completeness implies that only some details are necessary for explaining the phenomenon in the broader class of explanatory relevance. This means that ontic constraints play their explanatory role if they are referred to the proper class of epistemic relevance. Thus, employing the ontic norm of Salmon-completeness pushes us to adopt the non-ontic conception of explanation.

In the case of epistemic constraints, the following are crucial: 1) heuristic strategies of decomposition and localization, 2) abstraction and idealization. In the first case, Bechtel and Richardson
%\label{ref:RNDPNyRsww72X}(2010)
\parencite*[][]{bechtel_discovering_2010} %
 rightly note that such heuristic strategies can be seen as basically compatible with the above-mentioned ontic spatiotemporal constraints. The main point of these strategies is to approximate the behaviour of the system based on the interaction of working parts of mechanisms. These strategies need, at the same time, the integration of functional and structural descriptions of mechanisms and the mapping of activities to working parts (as suggests, e.g., 3M constraint). This shows that epistemic and ontic constraints can and should be combined in mechanistic inquiry. The strength of decomposition and localization is that they facilitate ``an increasingly realistic representation of the explanatory domain, even when the initial representation is seriously distorted'' 
%\label{ref:RNDJrWoSDPsxv}(Bechtel and Richardson, 2010, p.8).
\parencite[][p.8]{bechtel_discovering_2010}.%


The case of abstraction and idealization strategies suggests that there is no straightforward mechanism-model-mapping in the case of scientific explanation
%\label{ref:RNDON84dzXHbu}(Parker, 2020).
\parencite{parker_model_2020}. %
If we use the metaphor of the map, one rather should say that scientists offer the atlas of ``explanatory maps'' when dealing with \textit{explananda}. In fact, the relation between different mechanisms and the MEx representing stuff in the world may be further illuminated by applying such a~metaphor. R. Giere, when discussing the issue of representation, points out that maps represent spatial regions from particular perspectives determined by various human interests
%\label{ref:RNDUrpu8vrGpO}(Giere, 1999, pp.81–82).
\parencite[][pp.81–82]{giere_science_1999}. %
According to him, the operative notion to describe the relationship between models and the world, is not the truth, but rather the similarity or fit, between the model and the world. The map analogy shows that maps are always partial and that they are always maps of something. Map makers and map readers, using interpretative rules, must be able to understand what certain maps represents and how to understand conventions used to prepare the map. The model-based understanding of scientific theorizing proposed by Giere means that scientists generate models using principles, specific conditions, and focus on relevant aspects and relevant degrees of similarity between the model and the target
%\label{ref:RNDd93D9DTPM1}(Giere, 2004).
\parencite{giere_how_2004}.

Craver, well aware of the model-based character of MEx, while defending OC, wants to avoid the problem of idealization by stressing that ``terms like ‘true', ‘idealized', and ‘abstract' apply to representations or models. They do not apply to the ontic structures they represent''
%\label{ref:RNDfQEEKEpKSI}(Craver, 2014, p.50).
\parencite[][p.50]{kaiser_ontic_2014}. %
 He further argues that the problem of idealization is not a~philosophical problem of explanation but of reference. According to him, ``the very idea of an idealized model of an explanation commits one, at least implicitly, to the existence of an ontic explanation against which the model can be evaluated'' 
%\label{ref:RNDguyGPkWraW}(Craver, 2014, p.50).
\parencite[][p.50]{kaiser_ontic_2014}. %
 Contrary to Craver, I~think that the very idea of an idealized model faces us with the problem of how such models convey explanatory information? Moving the issue of the idealization from the philosophical domain of explanation to that of reference, would not be a~solution to the problem of explanation. When looking at the history of science, we find cases where one can achieve understanding without having a~true representation of facts. For instance, one can evoke Maxwell's fictional vortex model treating light as electromagnetic radiation 
%\label{ref:RNDSdSMrXGzlO}(de Regt, 2015),
\parencite[][]{de_regt_scientific_2015}, %
 showing that not only the veridical representations can be explanatory 
%\label{ref:RNDHcW2pJ8NuS}(Bokulich, 2016).
\parencite[][]{bokulich_fiction_2016}. %
 It is true that ``the goal of building an explanatory text is not to provide the illusion of understanding but rather to provide bona fide understanding'' 
%\label{ref:RNDfXH0RYKxvW}(Craver, 2014, p.49).
\parencite[][p.49]{kaiser_ontic_2014}. %
 However, the commitment to a~truthful account conveying explanatory information works more as an explanatory ideal in scientific practice than the matter of facts. An \textit{explanandum} is the sum of observational data and concepts, models, hypotheses, etc. Both empirical data and scientific conceptualization have their own limits and only describe a~part of reality.

Whether science obtains an explanation of reality or the representations of reality is the question that should not receive yes/no answer if we want to further explore how scientific explanations and objects of scientific investigation are formulated. Following J. Bickle I~argue that there is no straightforward mapping of mechanisms onto the world in the case of scientific explanation. J. Bickle
%\label{ref:RNDxCAVL4hhBp}(2008)
\parencite*[][]{hohwy_real_2008} %
 enumerates four key principles in explaining neural mechanisms. These are principles of observation (occurrences of the hypothesized mechanism are strongly correlated with the occurrences of the behaviors used as experimental measures), negative alteration (intervening to decrease activity of the hypothesized mechanisms must reliably decrease the behaviors used as experimental measures), positive alteration (intervening to increase activity of the hypothesized mechanisms must reliably increase the behaviors used as experimental measures), and integration (the hypothesis about the causal nexus that produces the behaviors used as experimental measures must be connected with as much experimental data as is available about the hypothesized mechanism). These convergent four principles show that the relationships between phenomena and entities' activities are to be discovered and described in a~piecemeal way rather than merely ``given to us'' (whatever the latter claim would mean). These relationships are rather ``inferred based on correlations between changes in monitored behaviours or effects that are taken to be indicative of changes in these phenomena and activities. […] This makes explanation fundamentally epistemic'' 
%\label{ref:RNDKr1XWvUgnV}(van Eck, 2015, p.15).
\parencite[][p.15]{van_eck_reconciling_2015}.%


The MEx basically involves, on the one hand, conveying an explanatory and empirical understanding of how entities and activities are organized in the production of the phenomenon to be explained. The explanatory understanding can be expressed as a~kind of understanding-why, while the empirical understanding means that we are dealing with domains of empirical inquiry
%\label{ref:RNDpiasWJAha1}(Khalifa, 2017, pp.1–3).
\parencite[][pp.1–3]{khalifa_understanding_2017}. %
 On the other hand, mechanistic understanding results are intimately connected to expertise in the specific scientific field or research programme. In fact, understanding results from the cooperative scientific enterprise performed by many scientists, using a~plurality of methods and experimental practices. I~concur with Khalifa that representing mechanisms or causal structures,,may be treated as one of the local constraints that should be satisfied in addition to the global constraints (such as, the \textit{explanandum} is approximately true, the \textit{explanans} makes difference to the \textit{explanandum}, the \textit{explanans} satisfies our reasonable ontological requirements) placed on the explanation. This focus on representational processes conveying understanding does not imply a~purely psychologistic view on scientific explanation, but rather explores the space of complex methodological and experimental aspects present within the scientific endeavour. It is beyond the aim of this paper to treat this issue in more detail, but it is certainly worth of being further explored.

It is now time to take stock. First of all, in this section I~focused on the ontic and epistemic norms and constraints of good MEx, which apply to non-ontic conceptions of explanation. This entering into the intra-epistemic debate has shownthe model-based character of MEx and the importance of mechanistic understanding if one is dealing with the specific accounts of explanation. In my analysis of EC and critics of OC, I~followed the convention that ``sentences are strings of visual or audible symbols that express propositions; propositions are the abstract entities that carry the meaning of the sentences; facts are concrete things in the world and, unlike sentences or propositions, are not capable of bearing truth or falsity''
%\label{ref:RNDM9cSYe63WB}(Khalifa, 2017, p.148).
\parencite[][p.148]{khalifa_understanding_2017}. %
 Secondly, the current debate on conceptions of explanation has shifted from what explanations are towards a~discussion on the norms and constraints of MEx. Arguing for normative constraints of explanation was called by some authors ``the normative turn'' 
%\label{ref:RNDpbreAoTfv7}(Sheredos, 2016, p.921; Wright and Van Eck, 2018, p.1023).
\parencites[][p.921]{sheredos_re-reconciling_2016}[][p.1023]{wright_ontic_2018}. %
 According to these authors, with whom I~concur, this debate shows that the OC after the normative turn is abandoned (Bokulich, Sheredos, van Eck, Wright) or at least ``in retreat'' 
%\label{ref:RND8ePhQ8gEl7}(Potochnik, 2018).
\parencite[][]{potochnik_eight_2018}. %
 In what follows I~want to shed further light on the consequences of ``the normative turn''.






\begin{table}[H]
\resizebox{\textwidth}{!}{%
\begin{tabular}{@{}p{1.9cm}p{3.8cm}p{5.5cm}@{}}
\rowcolor[HTML]{CBCEFB} 
&
{\bfseries Ontic conception (OC)} &
{\bfseries Epistemic conception (EC)}\\
\rowcolor[HTML]{FFFFFF} 
{\bfseries Main representants} &
W. Salmon, C.F. Craver, L.~Darden, S. Glennan, P.~Illari, M. Povich, T. Knuuttila &
W. Bechtel, C. Wright, B. Sheredos, A.~Bokulich, A. Levy, M. Nathan, D.~van Eck, R. Frigg, H. de Regt\\
\rowcolor[HTML]{ECF4FF} 
{\bfseries Core idea} &
the causal structure of the world, that is, the entities and activities and the organization by which they produce the phenomenon of interest &
an intelligible model of the activities, entities and their organization that scientists can understand, manipulate, and communicate, in order to move ahead in the research program\\
 \rowcolor[HTML]{FFFFFF} 
{\bfseries Vehicles of explanation} &
full-bodied things; neither representations nor texts

these things can be: causes, mechanisms in the world, facts, events, set of factors &
representations of mechanisms in the world; such as internal mental representations or external to the cognitive agent (diagrams, linguistic descriptions, mathematical equations, physical models, etc.)\\
 \rowcolor[HTML]{ECF4FF} 
{\bfseries Norms} &
truth &
explanation aims to attain the understanding

ontic norms: the instruction to describe the causal structure of the mechanism

epistemic norms: to increase the intelligibility, generality, systematicity, integration\\
  \rowcolor[HTML]{FFFFFF} 
{\bfseries Constraints} &
all and only the relevant features of the mechanism in question

&
ontic constraints: accuracy and completeness, spatial and temporal, the mechanism-to-model-mapping (3M)

epistemic constraints: heuristic strategies of decomposition and localization, the abstraction and idealization\\
%PP \bottomrule
\end{tabular}%
}
%\caption{Types of activities.}
%\label{gim.tab1}
\end{table}






%\begin{supertabular}{|m{4.03cm}|m{5.822cm}|m{6.302cm}|}
%\hline
%&
%{\bfseries Ontic conception (OC)} &
%{\bfseries Epistemic conception (EC)}\\\hline
%
%{\bfseries Main representants} &
%W. Salmon, C.F. Craver, L. Darden, S. Glennan, P. Illari, M. Povich, T. Knuuttila &
%W. Bechtel, C. Wright, B. Sheredos, A. Bokulich, A. Levy, M. Nathan, D. van Eck, R. Frigg, H. de Regt\\\hline
%
%{\bfseries Core idea} &
%the causal structure of the world, that is, the entities and activities and the organization by which they produce the phenomenon of interest &
%an intelligible model of the activities, entities and their organization that scientists can understand, manipulate, and communicate, in order to move ahead in the research program\\\hline
%
%{\bfseries Vehicles of explanation} &
%full-bodied things; neither representations nor texts
%
%these things can be: causes, mechanisms in the world, facts, events, set of factors &
%representations of mechanisms in the world; such as internal mental representations or external to the cognitive agent (diagrams, linguistic descriptions, mathematical equations, physical models, etc.)\\\hline
%
%{\bfseries Norms} &
%truth &
%explanation aims to attain the understanding
%
%ontic norms: the instruction to describe the causal structure of the mechanism
%
%epistemic norms: to increase the intelligibility, generality, systematicity, integration\\\hline
%
%{\bfseries Constraints} &
%all and only the relevant features of the mechanism in question
%
%&
%ontic constraints: accuracy and completeness, spatial and temporal, the mechanism-to-model-mapping (3M)
%
%epistemic constraints: heuristic strategies of decomposition and localization, the abstraction and idealization\\\hline
%
%\end{supertabular}



\section{What is left after ``normative turn''?}
P. Illari
%\label{ref:RNDi87oPkTc9k}(2013, pp.248–252)
\parencite*[][pp.248–252]{illari_mechanistic_2013} %
 has tried to argue that both ontic and epistemic constraints should be recognized and reconciled since both of them are essential. ``Without the first constraint, we are not explaining the production of a~phenomenon by a~mechanism; without the second, we do not achieve the understanding essential to explanation'' 
%\label{ref:RNDWIpnHjHBcw}(Illari, 2013, p.250).
\parencite[][p.250]{illari_mechanistic_2013}. %
 Her reading of ontic and epistemic constraints clarifies that the goal of science to reveal the causal structure of the world and the goal of science to achieve a~communicable understanding are both necessary. This sort of clarification is unproblematic. However, when Illari tries to spell out differences between Craver's (OC) and Bechtel's (EC) accounts, she discusses the case of prioritising one norm over the other. I~agree that we may grant priority to the ontic or epistemic norms, but such a~move should not be guided by \textit{a~priori} principles which we employ in scientific reasoning. Moreover, according to her account, both norms work together in order to generate a~successful MEx. Without the first one we cannot describe the (causal) structure of the world, while without the second one we cannot build a~model of the activities, entities and their organization 
%\label{ref:RND58a6sGwL79}(Illari, 2013, p.250).
\parencite[][p.250]{illari_mechanistic_2013}. %
 Again, I~wholeheartedly agree with the relevance of ontic and epistemic norms and normative constraints for building MEx in an integrative way. Let us now focus on how to provide such an integration. but how to do it?

L. Kästner and P. Haueis
%\label{ref:RNDRxe1MgDpOl}(2021)
\parencite*[][]{kastner_discovering_2021} %
fill the gap left by Illari, since they convincingly show how ontic and epistemic norms work together through mechanistic inquiry as a~whole. First of all, they emphasize that mechanistic discovery typically starts with characterizing phenomenon via different epistemic activities (e.g., modeling, experimenting) in which scientists perform various epistemic operations (e.g., injecting a~current in a~neuron) to track different activities and parts of the mechanism which is investigated. Secondly, they argue that epistemic activities containing different models, skills and instruments in the process of discovery do not stand in opposition to the realism of the entities and activities constituting mechanisms actually investigated. In other words, there are interacting multiple dimensions in the elaboration of scientific explanation, rather than a~clear-cut alternative between ontic and epistemic norms and constraints. It stems from their analysis that it remains crucial to trace out different stages of the discovery process, with some phases related to the ontic point of view, and others more oriented to epistemic aspects. What is the most interesting aspect of their paper 
%\label{ref:RNDVDeuLaxDWj}(Kästner and Haueis, 2021)
\parencite[][]{kastner_discovering_2021}%
(Kästner and Haueis, 2021) is that they aim to show how ontic constraints may directly or indirectly constrain some epistemic activities; and vice versa, i.e., they show how epistemic norms may help in choosing the accuracy or completeness ontic norms when dealing with empirical findings that conflict with the currently most plausible models of a~mechanism.

Kästner and Haueis, in a~similar way to Illari, have bracketed the metaphysical issue of explanation from their discussion, by claiming that mechanistic inquiry is both ontically and epistemically constrained. They have gone further than Illari, since apart from integrating both sets of norms and constraints they have shown how ontic constraints guide mechanism discovery from the bottom up and how epistemic constraints help with anomaly resolution. It seems that the main solution to the controversy about the priority of certain norms or constraints comes from considering both of them from a~reconciliatory and diachronic perspective. That is, in scientific practice fulfilling both epistemic and ontic norms ``requires a~kind of dynamic and temporally extended zig-zag between distinct explanatory practices''
%\label{ref:RNDrahWgRA8i9}(Sheredos, 2016, p.943).
\parencite[][p.943]{sheredos_re-reconciling_2016}. %
 In fact, Illari concludes her paper by noting that any successful disentangling of our theories, ontically and epistemically constrained, ``will need to look at what is happening over time, rather than at a~single time'' 
%\label{ref:RNDJ8z4QESL9q}(Illari, 2013, p.254).
\parencite[][p.254]{illari_mechanistic_2013}. %
 ``Our good mechanistic explanations are always the result of a~struggle to satisfy both ontic and epistemic constraints'' 
%\label{ref:RNDl4xgUHW9xc}(Illari, 2013, p.254).
\parencite[][p.254]{illari_mechanistic_2013}. %
 It is the apt correlation between the methods of observation/analysis (epistemic aspect) and the data itself (ontic aspect), that brings about the success of MEx. The ontic and the epistemic norms and constraints are neither alternatives nor directly reconcilable in a~simple way. They express different moments of the explanatory strategies and complementary aspects of explanations aimed at the \textit{explananda}. Only the integration of both aspects serves explanatory purposes.

The simplicity of the last claim is a~bit striking, considering how long lasting the metaphysical debate on what scientific explanations are and the establishment of criteria for good scientific explanations. Although these debates were evolving in a~parallel way, it is important to not conflate them. What gave rise to the debate between OC vs EC, as I~suggested at the end of the previous section, was adopting the language suggesting that stuff in the world performs explanatory acts. But causal mechanisms are concrete things in the world and, unlike sentences, propositions, models, etc., are not capable of bearing truth or falsity. Finally, in order to further argue for the more nuanced view on the reconciliation and integration of various norms and constraints, I~want to bring up the problem of the boundaries and levels of mechanisms.

The criteria for individuating the boundaries between entities, activities, and mechanisms themselves are the most difficult problems to solve
%\label{ref:RNDbDdbwMKbbx}(Kaiser, 2017).
\parencite[][]{glennan_components_2017}. %
 There are different principles that may be of help in carving the mechanisms and their components, such as the individuation of natural boundaries of biological objects (e.g., the cell membrane, the skin, the chain of mountains that borders a~specific ecosystem) 
%\label{ref:RNDsQpO41Zq0I}(Darden, 2008)
\parencite[][]{darden_thinking_2008} %
 or deciding upon the strength of interactions (e.g., interactions among parts are generally conceived of as stronger than interactions between parts and environment) 
%\label{ref:RNDjsMhkXHGlh}(Wimsatt, 1974).
\parencite[][]{wimsatt_complexity_1974}. %
 Even if one applied these principles, it is possible to get different results, since decompositions of a~mechanistic system into parts depends on the explanatory context. For instance, the human body has many systems which are responsible for the various activities of the body, such as the cardiovascular, the respiratory, the nervous, the endocrine, the muscular-skeletal system, etc. The difficulty is that the part decompositions generated by the phenomena will play out in the body in different overlapping ways, e.g., the arteries and veins of the cardiovascular system are also involved in the respiratory system 
%\label{ref:RNDMZwcNCQNGE}(Kaiser, 2017, pp.37–38).
\parencite[][pp.37–38]{glennan_components_2017}. %
 Hence, how we ``cut the nature at its joints'' depends upon the circumstances in different contexts of explanation.

In this context, L. Kästner
%\label{ref:RND3ct8l6Wz9T}(2018),
\parencite*{kastner_integrating_2018}, %
rightly notes that the mechanistic talk of levels does not seem to be a~satisfactory way of defining what is at the same level in terms of local composition relations. She proposes instead an approach of epistemic perspectives, largely indebted to Giere's
%\label{ref:RNDxL43u7ZopJ}(2006)
\parencite*[][]{giere_scientific_2006} %
 approach, which I~have already mentioned. Kästner characterizes these perspectives along five dimensions (i.e., resolution---different temporal or spatial scale; specificity---what kinds of things can be detected from certain a~perspective; point of view---depending on the background theories and taxonomies assumed within certain perspective; sensitivity---with respect to specific factors; scope---allowing for investigating different portions or aspects of phenomena depending on various methodological constraints) and stresses that the characterization and individuation of them is primary an empirical affair. She explicates the advantage of such an approach in the following words:

\myquote{
Once the relations between them are worked out, epistemic perspectives characterized along dimensions […] give us the means to locate observed entities and activities at ‘the same level' or ‘different levels', a~way of distinguishing different kinds of dependency relations (such as causal and constitutive relations in mechanisms), and a~solid foundation for integrating multiple scientific observations into complex multi-perspective mechanistic explanations. We can thus piece together the mechanism mosaic while avoiding the problems associated with local, compositionally related, levels of mechanisms in this context''
%\label{ref:RNDn9fUntNx98}(Kästner, 2018, pp.77–78).
\parencite[][pp.77–78]{kastner_integrating_2018}.%
}

I~am very sympathetic to Kästner's view that both the decomposition of mechanisms and carving their levels deeply depend upon contextual elements
%\label{ref:RNDOJlg3iu5cJ}(Woodward, 2008, pp.217–220).
\parencite[][pp.217–220]{woodward_comment_2008}. %
 In fact, ``there is an inherent perspectival aspect'' 
%\label{ref:RNDVakU9ZdBQE}(Darden, 2008, p.960)
\parencite[][p.960]{darden_thinking_2008} %
 in the case of levels or boundaries of mechanisms. Such perspectivalism, however, does not necessitate ``arbitrary choices in individuating phenomena and mechanisms'' 
%\label{ref:RNDVeRtljeAUc}(Darden, 2008, p.960),
\parencite[][p.960]{darden_thinking_2008}, %
 but evidences the search for context-sensitive considerations which are the best from the point of view of specific explanatory purposes. ``That different models and observations depend on our purposes does not imply, however, that they do not show anything real. They just emphasize different features, carve out different aspects, or provide different filters'' 
%\label{ref:RNDsvuNrM9IuX}(Kästner, 2018, p.76).
\parencite[][p.76]{kastner_integrating_2018}. %
 I~think that MEx presents ``a more sophisticated picture of the relation between realistic representation and explanatory understanding'' 
%\label{ref:RNDA9J4Niu6o6}(de Regt, 2015, p.3795),
\parencite[][p.3795]{de_regt_scientific_2015}, %
 as mentioned in the previous section. On the one hand, epistemic perspectives on explanations show that scientific knowledge is inextricably bounded to the modeler's knowledge, employed tools and pragmatic concepts. On the other hand, the perspectival approach does not merely trigger unlimited explanatory pluralism. Rather admitting many epistemic perspectives is useful in the function of finding the best explanation in certain context.

Analysing the MEx through epistemic perspectives sheds further light on the relevance of ontic and epistemic norms and constraints for building MEx in an integrative way. It shows that mechanisms in the world are not doing the explanatory job, but on the contrary, investigating a~phenomenon from multiple different perspectives is what builds the MEx. Such an epistemic conception of MEx does not entail neglecting the objectively real character of causal mechanisms, but forces us to admit partial and perspectival character of the explanatory work.

\section{A~dual ontic-epistemic approach }
The aforementioned characterization of boundaries, levels or causal mechanisms fits very well the EC, i.e., boundaries, levels and causal mechanisms constitute idealized representations of processes. In the case of causal mechanisms, D. Nicholson
%\label{ref:RNDOI5sTmzKA6}(2012, p.160)
\parencite*[][p.160]{nicholson_concept_2012} %
 rightly argues that ``explanations always presuppose a~context that specifies what is to be explained and how much detail will suffice for a~satisfying answer, […] it is this very epistemic context that determines how causal mechanisms are individuated and what details are featured in them''. If this is so, it seems that the term ``mechanism'' does not necessarily refer to worldly causal mechanisms. What does it refer to then?

When speaking about mechanisms, it is quite intuitive to think about mechanisms ``out there in the world'' and MExs or models that depict them. However, as Nathan
%\label{ref:RNDoBdXaaFhyN}(Nathan, 2021, pp.171–172)
\parencite[][pp.171–172]{nathan_black_2021} %
 notes, the term ``mechanism'' is ambiguous and may refer to both aspects, that is, to things in the world and to their representations. Since MExs represent entities and activities in the world, it may seem to be quite obvious that claims about representations are claims about the mechanisms in the world. In fact, such a~blending together of two distinct claims may stem from the OC erroneous assumption that ``ontic explanations are not texts; they are full-bodied things'' 
%\label{ref:RND8iXhdK2kfY}(Craver, 2014, p.40).
\parencite[][p.40]{kaiser_ontic_2014}. %
 The ambiguity of the term ``mechanism'' would be then an additional evidence of this initial error. Although the distinction between mechanisms and mechanistic models is at the core of mechanistic literature 
%\label{ref:RNDbUiM7Sxd6n}(Glennan, 2005),
\parencite[][]{glennan_modeling_2005}, %
 the free employment of the term ``mechanism'' to both entities and their representations probably has further added to the confusion that stuff in the world performs explanatory acts. One of the consequences of the aforementioned epistemic perspectival approach would be the claim that real mechanisms are not just the represented mechanisms.

If mechanisms ``out in the world'' should be kept distinct from the represented mechanisms, what about the use of the term ``mechanism''? Should it be abandoned? According to Nathan
%\label{ref:RNDyYk69KOIvB}(2021, pp.162–190),
\parencite*[][pp.162–190]{nathan_black_2021}, %
 NMP has the merit of having shown how the concept of mechanism always has a~two-sided nature: on the one hand, mechanisms are ``out in the world'', on the other hand, they are a~model-theoretic construction and a~mode of scientific representation. Mechanisms are central for constructing a~scientific theory, although they are ``black boxes''. This means that mechanisms work as placeholders---frames or difference-makers---in a~causal explanation represented in a~model 
%\label{ref:RNDDPH3gmoyUB}(Nathan, 2021, p.133).
\parencite[][p.133]{nathan_black_2021}. %
 For Nathan, the practice of black-boxing consists of three phases: 1) the framing stage of specifying the \textit{explanandum}, 2) the difference-making stage of providing a~causal explanation of the \textit{explanandum}, 3) the representation stage which determines how the difference-makers should be portrayed via the abstraction and idealization strategies. Mechanistic models are distinguished by their underlying structure (e.g., real physical entities representationally related to some abstract systems) and their interpretational capability. Both actual reference and imagined elements are part of the model, and the model's interpretation must explain their relationship and how the model represents phenomena.

It is important to emphasize that representations in scientific practice are not only involved in the \textit{explanans}, but are also the \textit{explanandum}, which corresponds to Nathan's first stage of black-boxing. This is so because scientists explain the phenomenon-as-represented. In other words, the explanation is always conceptualized within the particular explanatory context
%\label{ref:RNDTyRxIOLRym}(Bokulich, 2011).
\parencite[][]{bokulich_how_2011}. %
 The phenomenon-as-represented means a~representation shared by a~community of researchers, not just the subjective state of cognition or particular linguistic or conceptual description. That these explanatory choices are not arbitrary, results from the fact that ``the world constrains which representations are, or are not, going to be adequate for a~given explanatory context'' 
%\label{ref:RNDqTlI1CbwwI}(Bokulich, 2018, p.802).
\parencite[][p.802]{bokulich_representing_2018}. %
 Such a~conception of explanation recognizes the presence and joint working of both ontic and epistemic norms and constraints.

My colleagues and I~have discussed the problem of the definition of species, suggesting that the concept of species is better understood within a~dual ontic-epistemic approach
%\label{ref:RNDXdDN1NebsX}(Marcacci, Oleksowicz and Conti, 2023).
\parencite[][]{marcacci_ontic_2023}. %
 We have chosen this concept as the case study, since it plays various roles in biological investigations. On the one hand, it is based on certain ontic assumptions about the properties of species and the cause of an individual's belonging to a~species; on the other hand, it describes this membership as a~consequence of domain specific methodological operations and conceptual assumptions. First of all, we argued that the concept of causal mechanism is not a~decisive argument for realism about natural kinds or species. In fact, both the consideration of underlying causal processes and explanatory interests play an indispensable role in recent approaches to natural kinds. Secondly, in agreement with Kästner's view on the possible variety of ontological commitments of epistemic perspective approach, we contend that the perspectival character of the concept of mechanism is not a~decisive argument for nominalism about natural kinds or species. Thirdly, the application of MEx entails theory-dependent pluralism about natural kinds or species.

Let us briefly further comment these three points. The last one means that the explanatory pluralist stance is the viable option from the mechanistic point of view. The interest-relativity of scientific classificatory categories, such as species, entails that various epistemic strategies and constraints remain essential and irreducible features of any biological explanation in case of species. Referring to the first and the second point, we argued that neither realism nor nominalism about natural kinds or species is the solution to the problem. The viable solution, if species are to be grasped within ontic-epistemic approach, can be formulated in the following form:

\myquote{
Close attention to the various accounts of species shows that they play important and distinct roles within the sciences. On the one hand, they work as metaphysical posits; on the other, as explanatory postulates. In the former case, the objectivity of species is what grounds the objectivity of explanations, and sound explanations of species require us to identify the relevant factors at work in evolutionary processes. As explanatory postulates, they play a~specific role restricted to the context of a~particular theoretical framework or model
%\label{ref:RNDRNZrJsP1bD}(Nathan, 2023).
\parencite[][]{nathan_causation_2023}. %
 In this case, species are essentially preliminary hypotheses or theoretical units awaiting to be replaced by more perspicuous explanatory elements'' 
%\label{ref:RNDJEaLEQhd1i}(Marcacci, Oleksowicz and Conti, 2023, p.12).
\parencite[][p.12]{marcacci_ontic_2023}.%
}

This distinction between species \textit{de re} and explanatory species provides the conceptual resources to rethink the presence of ontic and epistemic norms and constraints in the long-standing debate on natural kinds and the notion of species. This distinction parallels that one between mechanisms out in the world and explanatory mechanisms. Both these distinctions help us to note that while proponents of OC would argue that species \textit{de re} or mechanisms \textit{de re} do the explanatory job, proponents of EC argue that this is not the case. The inconclusiveness, as it may seem, of the debate on boundaries or levels of mechanisms, or on the definition of species can be expressed through the following philosophical maxim: ``distilling metaphysical implications from scientific explanations requires close attention to explanatory practice''
%\label{ref:RNDwA4lmTRBdG}(Love and Nathan, 2015, p.773).
\parencite[][p.773]{love_idealization_2015}. %
 If contextual aspects in the explanations are unavoidable, then we should moderate the ontological implications drawn from our models. Not because there is ``nothing out there'', but because there is in fact more than we expect to be there. The epistemic-relativity of our categories, explanations and models can be compatible with their objectivity. The mechanistic strategy is the illuminating one, since it pays attention to how epistemic and ontic norms and constraints are intertwined within scientific explanation, and how such an objectivity may be achieved.

\section*{Conclusions}
NMP has the merit of having shown how the concept of mechanism as a~tool for representation has always a~two-sided nature: on the one hand, mechanisms are things in the world, on the other hand, they are model-theoretic constructions. Mechanisms are central for constructing a~scientific theory, although they are theoretical placeholders, that is, indicators of what a~theory is built around. If one neglects this feature of the term ``mechanism'', one will not understand the importance of integrating ontic and epistemic norms and constraints within MEx.

Today the crux of the debate is the interplay between different norms and constraints in providing MEx. It seems that the scale is tipped towards the proponents of EC. The OC remains under attack for the set of reasons previously discussed: the stuff in the world does not perform explanatory acts; the crucial role of general principles in scientific reasoning; scientific reasoning relies heavily on abstraction/idealization/generalization; in scientific practice, we deal rather with highly idealized representations of causal processes rather than with real mechanisms; we do not have direct access to the phenomenon, but we deal with phenomena-as-represented which are previously conceptualized; currently dominant modelling strategy of science evidences the crucial role of abstraction and idealization, etc. Briefly put, there is no scientific explanation without intentional agents who try to decipher things out there in the world. However, this does not imply that there are no things ``out in the world''.

What are then the main conclusions? First of all, the ontic features of the world do not alone settle all questions about the adequacy of an explanation, but the latter is rather settled by an evaluation of our explanatory practices and the features of our models. An increased awareness of the presence of multiple dimensions within scientific explanations (e.g., ontic and epistemic norms and constraints involved in the elaboration of the concept of mechanistic boundaries or levels, or the definition of species), can aid in comprehending distinct strategies employed in the various sciences, and help to understand how they cooperate to adequately account for complex phenomena. This brings us to the last but not least conclusion.

The ontic and the epistemic norms and constraints express different moments of the explanatory procedures and two complementary aspects of specific accounts of explanations aimed at the \textit{explananda}. The latter fact implies that only the integration of both aspects serves explanatory purposes. Our analysis of ontic-epistemic debate shows that one cannot simply read off truths, or the truth about what the world is like. At the same time, it does not imply that every claim is equivalent. On the contrary, via the use of our models, concepts, and theories, we have only a~piecemeal formulation of partial knowledge about reality. Dependence on how the world is will be a~commitment entailed by any good conception and account of explanation.

\end{artengenv}
