\begin{recengenv}{Kristian Campbell González Barman}
	{Exploring the epistemic and ontic conceptions of \textit{Models and Idealizations in Science}}
	{Exploring the epistemic and ontic conceptions of \tocname{Models and Idealizations in Science}}
	{Alejandro Cassini \& Juan Redmond (eds.), \textit{Models and Idealizations in Science: Artifactual and Fictional Approaches}, Springer Iternational Publishing, Cham 2021, pp.xv+270.}


This book presents an insightful collection of papers that explore scientific modeling, idealizations, and representation from diverse philosophical perspectives. The book's first chapter offers an extensive, well-written introduction to the topic of scientific models, while the last chapter provides a~relevant annotated bibliography on the philosophy of models and idealizations in science. The ten remaining chapters consist of previously unpublished articles that explore certain intricacies of both modeling practices and the ontology and epistemology of models. Although each chapter is self-contained and independent, one can find a~few recurrent themes, such as analyses of artifactual and fictional approaches to modeling, or examinations of the role of surrogative reasoning and (de)idealization in modeling practices.

In what follows I~provide a~brief overview of these papers, pinpointing how some of their contributions might relate to the debate between ontic and epistemic conceptions of explanation.

Natalia Carrillo \& Tarja Knuuttila's chapter examines idealization in scientific modeling. A~big focal point of the debate on idealization concerns whether idealizations are beneficial or an epistemic deficiency to overcome. If one believes, as proponents of the ontic conception often do, that explanatory texts representing the ontic explanations need to be complete and accurate, then idealizations are something to be eventually replaced---whether by de-idealizing or by textually representing the ‘actual explanation'. Here, de-idealization would not only be achievable, but would constitute a~worthy goal. Those who adopt the epistemic conception tend to find intrinsic value in idealizations; for they allow scientists to identify the proper level of abstraction, enable selecting relevant factors, or are fundamental towards generalizing or towards unifying.

Nevertheless, both positions view idealizations as deliberate misrepresentations or distortions. Carrillo \& Knuuttila reject this view by adopting an artifactual approach. In this approach models are epistemic artefacts, and idealization is a~set of assumptions that align different representational tools to construct a~model aiming to answer research questions. This approach highlights the difficulty in disentangling epistemic benefits and deficiencies in the model and challenges both the idea of easy de-idealization and of idealizations simply being distortions.

Sympathisers of the epistemic conception might find a~strong argument here in that idealization makes the model possible in the first place
%\label{ref:RNDLa8baEgvD8}(Cassini and Redmond, 2021, p.57),
\parencite[][p.57]{cassini_models_2021}, %
 for example, by enabling the application of mathematical and computational tools. However, their analysis runs deeper by noting that labeling idealizations ‘distortions' assumes one has enough knowledge about the target phenomenon, a~flaw which may be present in both sides of the debate. Furthermore, talk about distortions obscures an important dimension of scientific modeling: exploring the possible (how phenomena could be produced) rather than actual. Here, the ontic conception presents a~clear disadvantage, as it does not have the resources to explicate such explanatory practices.

Mauricio Suárez \& Agnes Bolinska's chapter apply communication theory to analyze the informational content of scientific models. They argue that models can be seen as communication channels, transmitting information about their targets, whereas idealizations and abstractions can be likened to sources of informational noise and equivocation. The authors argue that this analogy can clarify certain modeling practices---for instance, shedding light on the trade-offs involved in minimizing idealization and minimizing abstraction. In this analogy, the explanandum phenomenon is the informational source, whereas the model is the courier that codifies the information.

\enlargethispage{-.5\baselineskip}

Surprisingly, the chapter does not discuss machine learning, despite the numerous parallels between their account and machine learning techniques. The authors appear to have missed an opportunity to establish a~more fruitful analogy, especially since machine learning models are often explicitly used as scientific models. For instance, the encoder-decoder architecture shares many similarities with the examples discussed in the chapter and is often employed in physics' simulations. In these contexts, concepts like loss functions, noise, or the dimensionality of encoder/decoder can provide more meaningful positive analogies for the topic at hand.

Nonetheless, the idea of quantifying idealization in terms of noise may be of interest to defenders of the ontic conception, as it measures how ‘far away' a~complete and accurate explanatory text might be. Detractors might however note that this overlooks the fact that idealizations often enhance the representational relationship between the model and target by sharpening the focus on what is of interest and ‘carving the world' at the right seams.

Staying within the analogy, ontic proponents might see a~pure signal devoid of noise of equivocation as a~worthy goal (notwithstanding that a~completely faithful representation is not equivalent to an ontic explanation). However, it is doubtful whether scientific modeling is actually concerned with trying to capture a~pure signal (i.e., to represent faithfully, whatever that may mean). Much like Borge's perfect map, such a~model would likely be of little use.

Alejandro Cassini's chapter discusses de-idealization in scientific models, emphasizing that its benefits and drawbacks depend on the model's aim. He argues that de-idealization should enhance explanations, predictions, or model effectiveness, rather than seek faithfulness. While he frames the debate in terms of (non-)representationalists and (anti-)realists, he raises points that are relevant to the epistemic/ontic debate. For instance, some models cannot be de-idealized due to their holistic nature (e.g., certain meteorological models), idealization might be irreversible in minimal models, and it is undesirable to de-idealize in models whose purpose is mathematical tractability.

José Diez's chapter outlines a~monist account of modeling. His account posits that scientific models are ensembles of entities and their relations, with some entities intended to stand for those in the target system. This includes a~contextual constraint determining the required degree of accuracy for a~given purpose. The account consists of explaining the conditions for performing a~representation and analyzing the success or adequacy of an existing representation. Supposedly this provides a~unified account of scientific modeling by providing necessary and ‘substantive' conditions without relying on strong fictionalist elements. However, it is unclear whether this account actually solves, as the author claims, the problem of representation (‘in virtue of what does the target [successfully] represent the model?'). The proposed account seems descriptive as to certain success conditions for when one can claim there has been (successful) representation, but it does not answer why the representation was successful.

Roman Frigg \& James Nguyen's chapter defends the fictionalist view of scientific models, which takes them to be analogous to characters and places in literary fiction. Their main argument lies in showing how several ‘myths' often used to discredit this view are incorrect, highlighting that it is possible to combine the fiction view with an account of scientific representation. From the perspective of the ontic/epistemic debate, the first and third ‘myths' are particularly interesting. The first myth suggests that the fiction view regards scientific products as falsehoods, while the third implies that the fiction view opposes representation. The first myth is tackled by separating two notions of fiction: infidelity and imagination. The fiction view supports the latter: models prescribe certain things to be imagined without committing to the truth status model components. Thus, the fiction view conceives of fictions as tools for learning truths about the world. The third myth is tackled by underscoring that the fiction view primarily concerns the ontology of scientific models, not their representational content. Several approaches are then suggested to combine the fiction view with different accounts of scientific representation.

Many scientific developments can be traced back to fictional uses of the imagination. For the fictionalist, viewing models as fictions affords creative freedoms when investigating certain scenarios. While some might argue that this process should simply be seen as a~simple heuristic which allows one to grasp the ontic explanation out there, there is also an argument to be made that fictions are part and parcel of the modeling endeavor.

Fiora Salis' chapter proposes an integrated fiction view for the ontology of theoretical models that combines insights from the fictional and artifactual perspectives. In the integration view, theoretical models are human-made artifacts, capable of serving different functions in various contexts while being analogous to fictional stories. These models are complex objects, consisting of model descriptions and propositional content. Model descriptions, which include linguistic and mathematical symbols prescribing specific imaginings, act as concrete representational tools and serve as props in a~game of make-believe. Scientists build models by selecting (and interpreting) the model descriptions that best serve certain purposes and contexts. However, these imaginings do not imply the existence of any fictional entities. Model content is determined by model descriptions in collaboration with principles of generation.

By placing imagination at the centre of the modeling process, the integration view resolves issues that challenge the artifactual view
%\label{ref:RNDkvhRoLSqDZ}(2021, p.173),
\parencite*[][p.173]{cassini_models_2021}, %
 such as explaining model building and development, attributing concrete properties to model systems, clarifying the notion of representation of so-called representational tools, and addressing how scientists engage in model-world comparisons. Similarly, and partly by how model descriptions and content are separated, this approach also solves several problems which afflict the fictional view, such as the non-existence of models, the unclear relationship between model descriptions and imaginary systems, difficulties in scientists sharing the same imaginings, and issues with resemblance between imaginary systems and their targets. Model descriptions and content exist and can, therefore, stand in relation. Moreover, model descriptions serve as props that, through their prescriptions to imagine, enable and constrain an agent's imaginings and allow them to share said imaginings.

I~found the integrated fiction account to be a~noteworthy account that addresses several important issues in the literature. The idea of combining the artefact and the fiction view, while simple, is well executed and makes for a~useful tool for the philosopher of science interested in modeling.

Manuel García-Carpintero's chapter posits that utterances about fictional entities and scientific models involve figurative language with clear truth-conditions. He applies this perspective to debates in semantics, specifically addressing supervaluationist models of indeterminacy.

Otávio Bueno's chapter presents a~structural account of scientific representation, arguing that reification of structures as abstract entities is unnecessary. Instead, four different strategies are proposed: adopting a~modal-structural interpretation of set theory, reconstructing relevant mathematics using second-order logic, resisting the need for a~metaphysical interpretation of set theory, and employing ontologically neutral quantifiers when quantifying over sets.

Juan Redmond's chapter presents an inferential conception of scientific representation addressing the question of how are models used to represent the world. He rejects the idea of correspondence between a~model and its target, emphasizing the importance of how users use models through interactive and dynamic processes. This dialogic approach calls into question whether there can be an ‘accurate and complete' (textual) explanation irrespective of the uses and users of a~model.

Andrés Rivadulla's chapter advocates for an instrumentalist approach to theoretical models in the physical sciences, emphasizing their utility as tools for explaining and predicting phenomena rather than as representations. In his view, theoretical models are idealized constructions that facilitate calculations, explanations, and predictions in scientific inquiry. A~key observation is that incompatible theoretical models exist for the same phenomena, emphasizing their use as tools rather than faithful representations. While this observation is perfectly compatible with an epistemic conception, ontic proponents require additional effort to accommodate this observation. Here, they either have to resolve the puzzle of how there can be two or more valid ontic explanations for the same phenomenon, to show that in fact these models are targeting different phenomena, or to show that one (or more) of the incompatible explanations is incorrect.

Overall, this book is a~valuable resource for philosophers of science, proficiently investigating topics such as modeling, the fiction view, and the artifactual view, as well as the role of de-idealization in scientific modeling. Although it was not originally focused on ontic and epistemic conceptions, it proves valuable for the debate by including several hidden gems that can be used to highlight problems with the ontic conception. This includes difficulties in accommodating various scientific practices (such as idealization and how-possibly modeling), the fact that there can be several possible explanations for the same phenomenon, or the fact that many explanations often target general, sometimes idealized, phenomena.



\autorrec{Kristian Campbell González Barman}

\end{recengenv}
