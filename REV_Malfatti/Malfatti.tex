\setcounter{secnumdepth}{0}

\pagestyle{Standard}

\begin{document}
\section*{How Science Tracks Understanding}
\section{[previous title: Scientific understanding and explanatory knowledge\textrm{]}}
Kevin McCain (2022), \textit{Understanding: How Science Explains the World}. Cambridge University Press, pp. xx + 122.

The aim of McCain's book is an ambitious one: to provide the readers with an understanding of science---i.e., of what science does, of how science works, of what science aims to achieve, and, more generally, of what makes science a~successful epistemic endeavor.

The author states in chapter 1 that, at its core, science is about explanation. Somewhat more precisely: the scientific enterprise is an enterprise of formulating, evaluating, and testing explanations for empirical phenomena. Why do scientists care about explanation? What is gained once one has access to an adequate or successful explanation for some phenomenon? Adequate or successful explanations, the author notices, are means to many ends that we value. Some of these ends are practical; other are epistemic. On one hand, a~world that we can explain adequately or successfully is a~world in which we can act effectively, i.e., in which we can predict what will happen, contribute to make things happen, and prevent things from happening (at least typically; the author explores multiple ways in which explanation and prediction or successful intervention might come apart in chapter 4). On the other hand, a~world that we can explain adequately or successfully is a~world that makes sense to us. It is a~world that we understand. Adequate or successful explanations, thus, contribute to making the world an intelligible place. The author conceives of these two goals of explanation, the practical and the epistemic, as tightly intertwined. Given that we understand a~phenomenon or subject matter, he claims, we will---at least typically---be effective in our practical interaction with it. Understanding a~phenomenon gives us some sort of power over it: it turns it into something that we can handle and, to some extent, bend to our will.

But what exactly is an explanation? What does it mean to explain a~phenomenon? What is the general structure of an explanation? What does an explanans tell us about an explanandum? These questions are notoriously difficult to answer, which is reflected in the somewhat intimidating variety of models of explanation that has been flourishing in the philosophical literature in the last decades. In chapter 2, McCain manages to offer a~very broad and yet instructive explication of the term ‘explanation'---one that could be deployed as a~sort of compass to orient oneself while navigating this variety of models. Explaining (prominently in science, but probably also in every-day life), the author suggests, is a~matter of ‘tracking dependency relations' (p. 23). Somewhat more precisely: an explanation of a~phenomenon $\textit{\varphi} $ provides answers to why- and how-questions relevant to $\textit{\varphi} $, and in doing so, it aims at showing what $\textit{\varphi} $ depends upon and what depends upon $\textit{\varphi} $ (e.g., causally, nomologically, …).

The author points out in chapter 3, while zooming in into specific kinds of scientific explanation, that scientists are after scientific \textit{knowledge} of explanations. As standardly conceived in epistemology, knowledge requires, among other things, the fulfillment of a~truth-condition. We cannot know that \textit{p} if \textit{p} is false. If we know that \textit{p}, then \textit{p} is true. If this conception of knowledge holds, then, for us to know that \textit{q} explains \textit{p}, it must be that \textit{q} \textit{correctly} explains \textit{p}. In other words: an explanation that is known is one that mirrors the facts, that depicts dependency relations actually holding in the real world. The author seems to align to this conception of knowledge when he writes: ‘[w]e can know that a[n] … explanation is correct \textit{when it is in fact correct} and we have sufficient evidence for believing that it is correct'
%\label{ref:RNDJLAvZWufd3}(McCain, 2022, p.38, emphasis added).
\parencite[][emphasis added]{mccain_understanding_2022}. %
 And yet, a~couple of pages later, the author claims something different. He writes: ‘we can know that a[n] … explanation is correct by possessing evidence that makes the truth of that explanation beyond a~reasonable doubt' 
%\label{ref:RNDgz8KhdOAJt}(McCain, 2022, p.40).
\parencite[][p.40]{mccain_understanding_2022}. %
 This second quote suggests the following: whether we have knowledge of an explanation or not is not a~matter of how accurately the explanation mirrors the facts, but rather of how the evidence that we have that supports it, and probably also on our evidential standards---standards that are not carved in stone, but likely to change over time. These strike me as two quite different conceptions of (scientific) knowledge, that at least \textit{prima facie}, are not easy to reconcile. If the second conception holds, what counts as scientific knowledge at a~certain point in time might be overturned at a~later point in time. We can have scientific knowledge of explanations that are extremely well-grounded in light of our evidential standards and yet fail to mirror the facts. This is not possible under the first conception. If we know an explanation, then it must be true.

How could this tension be resolved? Maybe what McCain is telling us is that genuine knowledge of explanations, i.e., knowledge of explanations that are correct, functions at best as a~regulative ideal in the scientific endeavor. In scientific practice we have no way of stepping out of our representational systems and comparing them to an independent reality. Thus, we have no infallible way to tell whether we really know, and whether the explanations that we formulated and deploy are indeed corresponding to the facts. The best we can do in practice is hold true or commit ourselves to those explanations that we reasonably judge as acceptable---i.e., those that perform best in light of our evidential standards. Given that we have sufficient grounds to endorse an explanation, we have scientific knowledge of it. But scientific knowledge is not necessarily genuine knowledge; it can be directed to representational systems that, despite all evidence suggesting the contrary, do not fulfill a~truth-requirement. The final chapter of the book, chapter 8, seems to provide at least some support to this reading. In the chapter, the author investigates the role of the inference to the best explanation in the production of scientific knowledge and he makes it clear that the function of an inference to the best explanation is to help identify not truth, but what is reasonable to endorse in the given epistemic circumstances.

While this reading is certainly in line with the book's overall spirit, I~am not sure it fully captures what McCain has in mind. Here and there throughout the book, one gets the impression that truth and genuine knowledge are more for the author than regulative ideals orienting the scientific endeavor. Somewhat more radical realist tendencies shine through the book's pages. Consider, for example, chapter 5: there, the author deals with the question of how explanations are evaluated and explores the role of empirical and theoretical virtues in such evaluation. At the end of the chapter, in what seems like a~sort of ‘optimistic meta-induction', he claims that in light of how successful we have been in the past by letting empirical and theoretical virtues orient our theory choice, it is reasonable for us to trust that ‘such virtues are good indicators of the truth'
%\label{ref:RNDK2PoYKCaPA}(McCain, 2022, p.68).
\parencite[][p.68]{mccain_understanding_2022}. %
 Of the truth, then---not of some weaker epistemic desideratum such as reasonable acceptability! In chapter 6, truth peeps out prominently again. While investigating the role that explanations play in fostering understanding, the author asks: ‘Is truth important for understanding or scientific explanations?' 
%\label{ref:RNDldWoASlJ1P}(McCain, 2022, p.78).
\parencite[][p.78]{mccain_understanding_2022}. %
 His answer does not leave much room for interpretation: ‘Yes, truth is extremely important for both. While a~false explanation might be such that if it were true it would provide understanding, genuine understanding requires accurate explanations' (\textit{ibid}). Even what look like inaccurate scientific representations, such as idealized models, the author claims in chapter 7 that they provide us with understanding only insofar as they make us appreciate truths about dependency relations that we would otherwise overlook.

Genuine understanding, according to the author, requires truth and is gained via explanations that mirror the facts. We need genuine knowledge of explanations in order to understand; a~less demanding epistemic state such as scientific knowledge (as clarified above) won't do if it does not guarantee that truth has been reached. Now, McCain clearly does not take understanding to be something that only a~final science will achieve. He takes understanding to be instantiated in real-life scientific practice. Our scientific understanding grows, he claims; we make advancements in understanding
%\label{ref:RNDJHqYrKUr9K}(see, e.g. McCain, 2022, p.68).
 But then, he must be endorsing some form of scientific realism. That is, he must believe that science not only targets truth, but has actually achieved it (at least to some extent); he must believe that science has managed to formulate explanations that are not only worthy of being endorsed in the given epistemic circumstances and in light of our evidential standards, but actually correct, i.e., corresponding to reality (at least approximately so).

I~think McCain's excellent book would have gained in further clarity and depth if such a~realist or optimistic stance towards science would have been not just presupposed and used as an unquestionable basis to build on, but rather put on the table, made explicit, articulated, and defended against alternatives. The book certainly succeeds in its aim: it provides the readers with an understanding of science; but as scientific realism is not the only available and viable stance, what is offered is just one way in which science could be understood.

Federica I. Malfatti

University of Innsbruck

\section*{Bibliography }
McCain, K., 2022. \textit{Understanding: How Science Explains the World}. Understanding Life. [online] Cambridge: Cambridge University Press. https://doi.org/10.1017/9781108997027.
\end{document}
