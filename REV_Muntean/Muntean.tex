\begin{recengenv}{Ioan Muntean}
	{Mechanisms ‘all the way down'?}
	{Mechanisms ‘all the way down'?}
	{\textit{Mechanistic Explanations in Physics and Beyond}, Brigitte Falkenburg and Gregor Schiemann, editors, \textit{European Studies in Philosophy of Science}, Vol. 11, Springer International Publishing, Cham 2019, pp.220}
	


\noindent Reviewed volume includes contributions presented at a~conference in Dortmund, Germany, organized in 2016 by the \textit{Académie Internationale de Philosophie des Sciences}, \textit{The German Research Foundation}, and the \textit{Technical University of Dortmund}. Most of the materials come from the conference presentations and gravitate around a~central theme: applying the ‘new mechanistic philosophy' to physics, where mechanistic explanations and models have not been typically used (although some contributors in Part III tackle the applications of mechanisms to economics, medicine, computer science, or tangentially, biology).\footnote{Unless otherwise specified, this review refers to the pagination of the reviewed
%\label{ref:RNDRMNPn8lKAk}(Falkenburg and Schiemann, 2019)
\parencite[][]{falkenburg_mechanistic_2019} %
 volume.}

Most contributions shed new light on some traditional topics in the philosophy of science, such as explanation, reductionism/emergentism, levels of reality, discovery, models, and so on. Specifically, as potential unifying themes, most authors delve into one or more of the following questions and issues:

\begin{enumerate}[label={(\arabic*)}]
\item Are mechanistic explanations or models adequate somewhere other than in ‘special sciences' (biology, neuroscience, cognitive science, psychology) where they originated, especially in ‘fundamental science' (mainly physics)? Are these explanations better in physics than other types of explanation (nomological, mathematical, etc.)?

\item If the concepts of mechanism and causation are closely interspersed, how do we generalize mechanistic philosophy to physics, where causation is not omnipresent and nomological explanations, models, and theories abound?

\item Is the world composed of mechanisms ‘all the way down' (up to and including the ‘most fundamental entities in the world': quarks, strings, spin networks, or whatever is ‘down there')?

\item Does the mechanistic worldview as developed during and right after the Scientific Revolution (Descartes, Newton, Leibniz, Huygens, Kant, etc.) overlap with the explanations and models of the ‘new mechanistic philosophy' (Salmon, Bechtel, Glennan, Machamer et al., etc.)?
\end{enumerate}

The reviewer and the reader will find several original and new ideas and proposals to address these questions.

As most readers know, the ‘new mechanistic philosophy' (NMP) originated in general philosophy of science, specifically in the literature on explanations, models, and scientific progress. Conventionally, Salmon's works from the 1980s is taken as a~starting point of the NMP
%\label{ref:RNDkeFk5DYoko}(Salmon, 1984; 1989).
\parencites[][]{salmon_scientific_1984}[][]{salmon_four_1989}. %
 NMP 
%\label{ref:RNDX9eKpxXj0q}(Glennan, 2017; Illari and Glennan, 2017),
\parencites[][]{glennan_new_2017}[][]{illari_routledge_2017}, %
 also called in this volume the ‘New Mechanism', is the constellation of ideas in the philosophy of science, epistemology, and metaphysics that addresses (1)–(4) and the like. NMP has been inspired by the practice of special sciences (biology, cognitive science, neuroscience, psychology, sociology, economics, computational science, engineering, etc.), and most case studies in the literature originate here, not in physics or chemistry. Before the 2016 conference in Dortmund, only a~handful of authors had addressed 
%\label{ref:RNDieQgeeJ3zc}(Illari and Williamson, 2011; Kuhlmann and Glennan, 2014).
\parencites[][]{illari_what_2011}[][]{kuhlmann_relation_2014}. %
 The \textit{Routledge Handbook of Mechanisms and Mechanical Philosophy}, which was published after the conference, includes no less than fourteen contributions in a~section called ‘Disciplinary perspectives of mechanisms' 
%\label{ref:RNDMvaY1WtNZh}(Illari and Glennan, 2017).
\parencite[][]{illari_routledge_2017}. %
 However, only one of its contributions makes any substantive reference to physics 
%\label{ref:RNDK8ceNaOmGR}(Kuhlmann, 2017).
\parencite[][]{kuhlmann_mechanisms_2017}. %
 As expected, the present volume fills this missing gap in the literature on mechanisms.

The philosophy of science literature on scientific explanation, models, and discovery has been dominated for at least four decades by approaches near NMP. Many characterizations of mechanisms are available, and most would include concepts from metaphysics and epistemology: entities, activity, organization, parts, levels, structures, function, and so forth
%\label{ref:RND2uaEkNnyXX}(Illari and Glennan, 2017).
\parencite[][]{illari_routledge_2017}. %
 Above all, mechanistic philosophy explains the behavior of a~composite system by reference to (i) its parts and (ii) the interaction among these parts. However, given the breadth of the contributions and the attempts to generalize existing approaches to mechanisms over discipline boundaries, the volume needs a~uniform definition with some chances of generalizing to ‘physics and beyond'.

Rather than delving into the definitional complexity of mechanisms, most authors in this volume consider Glennan's delineation of ‘minimal mechanism' as the working definition of mechanism: ‘A mechanism for a~phenomenon consists of entities (or parts) whose activities and interactions are organized so as to be responsible for the phenomenon'
%\label{ref:RNDQE2CoxQLoA}(Glennan, 2017, p.17).
\parencite[][p.17]{illari_routledge_2017}.%


According to the mechanistic philosophy, the most successful explanations in special sciences, as enumerated in (1), are mechanistic, and special sciences are in the business of discovering mechanisms. This thesis about science's success and progress is sometimes associated with a~different claim about the world: the world consists of mechanisms. The strong metaphysical commitment to mechanism is that the world is composed of mechanisms. This volume addresses whether there are mechanisms ‘all the way down'. The difference between the methodological questions (1) and (2), the more metaphysical (and ontological) question (3), and the historical (4) pervades most contributions to this volume. Interestingly, the authors do not visit the potential difference between the ontic and the epistemic concepts of mechanisms \textit{in physics}, which has been debated in the literature for at least two decades
%(Salmon, 1984; Glennan, 2002; Wright, 2012; Illari, 2013).
\parencites{salmon_scientific_1984}{glennan_rethinking_2002}{wright_mechanistic_2012}{illari_mechanistic_2013}.

A~word of caution is in order here. One must acknowledge that scientific practice and informal language used to communicate science do \textit{not} help this endeavor. The term ‘mechanism' is explicitly and extensively used in all areas of physics and chemistry. The presence of a~word in a~discipline is illusory, nevertheless. Do terms such as ‘mechanism' have the same meaning in physics, chemistry, ‘special sciences', or beyond science, and into history, art, or religious studies? A~related question is looming in the background: is NMP historically contiguous with the terms ‘mechanism' or ‘machine' as used during or right after the Scientific Revolution? The worry that we overuse or overreach some terms to include ‘mechanisms', ‘energy', ‘entropy', ‘information', and ‘complexity' over the disciplinary boundaries is genuine. It is trite to say that many areas of physics use the suffix ‘-mechanics'. Falkenburg lists almost a~dozen of areas of physics where the term ‘mechanism' is used extensively: generation of turbulences within fluids; tsunami generation by submarine mass flows; generation of turbulences within the accretion disk of astrophysical objects such as active galactic nuclei, quasars, or black holes; pulsation of stars and giant planets; transport, propagation, or diffusion of charged particles or photons; slowing down or acceleration of charged particles; interactions of particles in particle physics, including the Higgs ‘mechanism'; the mechanism of decoherence in condensed matter physics. Falkenburg argues that NMP does not apply well or smoothly to most cases
%\label{ref:RNDe1ddoD7jum}(Falkenburg and Schiemann, 2019, pp.84–85).
\parencite[][pp.84–85]{falkenburg_mechanistic_2019}. %
 How do we escape the linguistic trap and find genuine and relevant generalizations of mechanistic thinking to physics?

\enlargethispage{1.5\baselineskip}
In the first contribution to the volume, S. Psillos \& S. Ioannidis differentiate metaphysical claims, \textit{e.g.} answers to (3), from methodological claims about scientific practice: explanations, models, and prediction as answers to (1) or (2). The former relates to concepts in metaphysics such as causation, part-whole relation, or levels of reality: the most natural gambit is to relate mechanisms to causation as strongly as possible. The latter is a~thesis about how science advances by providing explanations and models of the world. Science advances by discovering new mechanisms, stipulating new mechanistic explanations, or creating new mechanistic models. This is dubbed ‘the mechanistic methodology'. Psillos \& Ioannidis claim that addressing (4) is essential and that philosophers during the Scientific Revolution were interested in the stronger claim, the metaphysical mechanistic thesis. Descartes had developed a~methodological thesis of continuity according to which properties of the invisible world are similar to properties of the visible world. Hence, one can assume that gravity is identical enough to the mechanisms governing the flow of liquids (whirlpools). Descartes and other so-called ‘Old Mechanists' suggest that the world operates like machines (as artifacts we are very familiar with, as we created them). If continuity is correct, there are mechanisms all the way down to gravity and beyond gravity. The controversy was alive during the 17\textsuperscript{th} century: Newton denied that mechanistic explanations extended to gravity and preferred a~more law-based explanation of gravity. Nevertheless, Huygens, Leibniz, among others, did not buy the idea that gravity follows a~law, but they demanded a~mechanistic explanation of forces in physics again. Newtonian mathematical or non-causal explanations were the alternative to the Old Mechanistic metaphysics.

Reflecting on NMP, Psillos \& Ioannidis think of mechanisms as an elaborate (theoretical) way of speaking of causal pathways in the world. This reduces the mechanisms to theoretical ways of talking about causation. Each scientific field uses a~specific language to describe these causal pathways; in some disciplines (biology, genetics, neuroscience), this language must include mechanistic terms. However, the scientists decide what language describes ``causal pathways'' and not the metaphysicians. As this is a~matter of language and pragmatism, a~metaphysical commitment is unnecessary. The two authors conclude that the NMP is too metaphysically loaded, similar to the ‘Old Mechanist Philosophy' (OMP). As a~line of criticism, Psillos and Ioannidis conclude that according to the more plausible methodological mechanism thesis, mechanisms can be replaced or displaced in any scientific discipline.

In a~more historical vein, co-editor G. Schiemann's own contribution argues that one way to answer (4) is to assume that the ontological commitment to mechanisms during and after the Scientific Revolution was more preëminent but that the NMP inherited partially a~peculiar ontological commitment from OMP. He takes the ontological commitment of the physicists and philosophers during the Scientific Revolution as either monistic or dualistic, based on their commitment to the existence of matter and forces. Newton qualifies as a~dualist in this sense. ‘The early modern pair of concepts of matter and force is structurally related as regards the contrast it draws to the contemporary conceptual pair of entity and activity'
%\label{ref:RNDtheZoBYIa9}(2019, p.43).
\parencite*[][p.43]{falkenburg_mechanistic_2019}. %
 Schiemann extends this monism-dualism distinction to contemporary mechanistic philosophy: Glennan qualifies as a~monist, as activities and organizations are properties of parts. Machamer et al. claim to be dualists because entities and activities are fundamentally different. Another touching point between the old and the NMP is a~commitment to levels of reality (both the contemporary monists and the dualists need this division).

Importantly, D. Dieks' contribution confronts head-on all questions (1)–(4). Starting from the contrast between Descartes' mechanistic worldview and Newton's mathematical formalism of physics, Dieks emphasizes some problems that Newtonians had with concepts such as ‘time,' ‘becoming,' or causation, which were needed in explaining phenomena beyond the simple kinematics. However, these concepts are not mechanistic in nature. Moreover, the Lagrangian and Hamiltonian approaches to mechanics use mathematical and non-causal concepts (energy, momentum, phase space, symmetries, etc.) that are not immediately compatible with OMP.

Maxwell's equations were initially formulated as mechanical properties of a~medium (the ether), like Descartes' vortices. Still, their contemporary interpretation has no mechanical medium, and the most likely interpretation is Lagrangian (as noted by Poincaré). Mechanical models of electromagnetism are possible, and they come in handy as far as one is committed to the existence of ether. The model with ether also assumes an infinite number of point charges as one can choose the number of particles interacting to be anything, even infinite. Dieks suggests that a~field interpretation of electromagnetism taints a~mechanistic interpretation of Maxwell's equation and sides with Poincaré, who thinks that mechanistic models are underdetermined in this case.

As other authors in this volume suggest, there is a~strong connection between causation and the mechanistic worldview. Although causation in classical electromagnetism can be accommodated (given some caveats), Dieks claims that causation must be local to operate in the minimal model definition of Glennan: activities, organization, and interaction must be localized. Although this may work in classical electromagnetism, action-at-distance is most likely incompatible with the mechanistic worldview. The structure of the quantum space of a~simple system (composed of two particles only) is too rich and far too non-local for a~mechanistic view at this level. The superposition principle, which is a~core element of quantum mechanics, vindicates the view that the state of a~composite quantum system cannot be reduced to the states of its parts. Knowledge of all the properties of the two parts is not enough to determine the state of the whole. Dieks uses here an idea promoted by Glennan and Kuhlmann: decoherence masks the quantum properties of systems and gives us the illusion of semi-classical behavior that can be captured by a~mechanistic view---i.e., where causation and mechanistic models may work. Dieks argues against Kuhlmann and Glennan's caveat that restricting in practice quantum models to cases of decoherence does not make these models plausible: ‘So there are features of reality, detectable in principle, that show that the literal content of the ontological claims of the mechanistic explanation strategy is false'
%\label{ref:RNDzs4ibx92U1}(2019, p.60).
\parencite*[][p.60]{falkenburg_mechanistic_2019}. %
 For Dieks, mechanical models in quantum systems may work as idealized and simplified views about the world (conditionalized on the strength and peculiarity of decoherence), but they are not enough to ontologically commit quantum mechanics to the mechanistic worldview.

If the description of the world proffered by quantum mechanics is non-local and holistic, then there is still a~critical function that the mechanical model plays: the way we understand the world. Dieks yields that mechanical models in quantum physics or even classical theories (electromagnetism) provide the conceptual grip needed to understand the world better
%\label{ref:RNDWeWiLOXral}(2019, p.63).
\parencite*[][p.63]{falkenburg_mechanistic_2019}. %
 Dieks contrasts mathematical explanations with mechanical ones and concludes they are both valid. However, in quantum contexts, non-mechanistic explanations are more accurate and truthful, although mechanistic models (given the decoherence caveat) can help us better understand quantum systems. This is nevertheless a~good reason to suspect that mechanistic explanations cannot be fundamental in physics.

In the other co-editor's own contribution to the volume---one of the longest---there are some affirmative answers to questions (1), (2), (3), and (4), and some counterexamples to (1) and (2). B. Falkenburg distinguishes between the NMP and OMP but focuses on what levels of reality are used in mechanistic explanations. She is open to the idea that despite differences, the Old mechanistic ideas can be generalized to the practice of physics of the 21\textsuperscript{st} century. She argues that Descartes and Kant used a~multi-level mechanical model of the universe: in his astronomy lectures, Kant used levels of description based on the size of celestial objects and their life and becoming.

In NMP, we also need what are called ‘levels of description', more precisely, level-based decomposition and recomposition of systems. Falkenburg argues that this is similar to Galileo and Newton's proposed method of analysis and synthesis and is imported successfully into contemporary neuroscience and biology. Nevertheless, this multi-level mechanistic view declined sharply in the 20\textsuperscript{th} century. Falkenburg takes clues from the literature on scientific practice of the 21\textsuperscript{st} century and concludes that the mechanistic view can be generalized after the last century's rift to the practice of the 21\textsuperscript{st} century. Consequently, contrary to Dieks's suggestion, mechanisms can be generalized to fields, as classical fields admit a~causal interpretation
%\label{ref:RND9eRRnkX0RH}(Salmon, 1984, p.239; referred at: Falkenburg and Schiemann, 2019, p.72).
\parencites[][p.239]{salmon_scientific_1984}[referred at:][p.72]{falkenburg_mechanistic_2019}. %
 What is needed to generalize mechanisms in contemporary physics? In physics, it is often possible to express the causal processes underlying a~mechanism in the precise terms of laws of physical dynamics 
%\label{ref:RNDDIsIGwIDYA}(2019, p.73).
\parencite*[][p.73]{falkenburg_mechanistic_2019}.%


The familiar methods of top-down and bottom-up modeling in biology and neuroscience are for Falkenburg methods that generalize the views of Descartes, Newton, and Kant and illustrate the need for multi-level analysis. Their ideas are mirrored in the commitment to the existence of levels in \parencite{machamer_thinking_2000}.

Mechanical models can be heuristically successful. Although false, they explain when they are used at a~higher level. Falkenburg takes the kinetic theory of gases as a~successful bottom-up model that explains a~lot of thermodynamic phenomena without being true. The kinetic theory is a~bottom-up, likely a~false model. Unlike Dieks, Falkenburg considers locality constraints (discussed in detail in Bechtel and Glennan) too restrictive: the parts do not need to be localized
%\label{ref:RNDTZUbWGjcwy}(2019, p.81).
\parencite*[][p.81]{falkenburg_mechanistic_2019}. %
 We can delocalize parts of the system and generalize them to fields or non-localized entities (cf. Dieks). For Falkenburg, the part-whole relation does not need to be restricted to spatial or temporal localized domains: they can be generalized such that field interactions or superpositions of quantum subsystems qualify as part-whole relations.

Although we do not have a~well-established concept of causality in physics and elsewhere, ``it is possible to generalize the notion of a~mechanism in an unconventional way''
%\label{ref:RNDDNF66RahKi}(2019, p.82).
\parencite*[][p.82]{falkenburg_mechanistic_2019}. %
 The parts are now idealized mathematical entities (she follows here a~suggestion of Malisoff, a~biochemist from the 1940s). We trivially replace real objects in the world with mathematical placeholders (point particles, infinite distances, etc.).

However, given the linguistic usage of the word ‘mechanisms' in physics, Falkenburg argues that the minimal mechanism description offered by Glennan or Salmon is inadequate in most cases. What about quantum systems? Here, the two short sections (§5.3.3.3 and §5.4.2) in which Falkenburg tries to address mechanisms in quantum physics are not satisfactorily developed. In her cursory note, Falkenburg suggests that a~sum rule is all we need to connect parts of the system to the whole. Unlike Glennan and Kuhlmann or Dieks, Falkenburg's strategy to generalize mechanisms to quantum systems does not need decoherence. We only need a~conventional ‘sum' rule
%\label{ref:RNDxZGDQ6pZIK}(2019, p.83).
\parencite*[][p.83]{falkenburg_mechanistic_2019}. %
 These last sections of the paper, although promissory, are underdeveloped (and see the reviewer's perspective on the overall missing points of this volume). The idea of a~sum rule is not developed at all (the reviewer reminds the reader that there is a~plurality of sum rules in quantum mechanics with somewhat different meanings).

In a~different note, in addressing (1) and (2), M. Ghins tries to amend the standard view on mechanisms of Glennan and Machamer by revisiting the original mechanistic approach of Salmon and Dowe. The key concept used is that of causal laws identified by a~formal criterion: they must contain the time derivative of a~relevant quantity. In this sense, causation is not an informal concept anymore but has a~more elaborate definition. This, in return, may solve two problems of the mechanistic view: the regression ‘difficulty' and the ‘circularity problem'.

In short, Ghins is poised to solve the circularity and the bottom-up problem of NMP. If mechanisms are causal and designed to explain causation, we have a~regress problem: the lower-level causation present in a~mechanism needs a~deeper mechanism to explain it, and so forth. Then we face the question: where do we ‘bottom out' mechanisms? Where do we stop with mechanistic explanations?
%\label{ref:RNDgYyKqMFXot}(2019, p.99)
\parencite*[][p.99]{falkenburg_mechanistic_2019} %
 Where the mechanisms end, the laws of nature must govern without being explained mechanistically. Ghins believes that Glennan and other mechanistic philosophers must admit that fundamental laws are not explained causally or mechanistically. Then, are mechanistic explanations genuinely fundamental?

Given these issues, Ghins is willing to give up the minimal mechanisms and return to the notion that mechanisms are elaborate descriptions of causal processes. In adding to Salmon's original idea of a~mechanism, Dowe took the causal process as a~transmission of an invariant quantity. ‘A causal interaction is an intersection of worldlines that involves the exchange of a~conserved quantity'
%\label{ref:RNDQ1uUPagtQR}(2019, p.103).
\parencite*[][p.103]{falkenburg_mechanistic_2019}. %
 Salmon and Dowe avoided introducing the idea of laws into the discussion of mechanisms, but for Ghins, objects are stable worldlines that \textit{need} laws of nature. More precisely, causal laws of nature. A~mechanism in Ghins is defined as a~complex system of \textit{nomic} causal interactions that explains the behavior of the system as changes of some relevant properties in time. Moreover, a~law is causal only if it contains a~time derivate 
%\label{ref:RNDFF5F8lP2QV}(2019, p.106).
\parencite*[][p.106]{falkenburg_mechanistic_2019}. %
 In Ghins or Salmon's modified view, mechanisms depend on fundamental causal laws. The temporal variation of properties characterizes the behavior of mechanisms.

\enlargethispage{1.5\baselineskip}
The reviewer notices a~couple of problems here. If some laws are causal, the immediate question is whether non-causal laws play any explanatory role in this variant of the mechanistic philosophy. If there are non-causal laws, are they explanatory idle concerning the mechanism? Do they play any role? It would seem so. For example, symmetry conservation laws are not causal, but do they play a~role in mechanistic philosophy? And if symmetry and other types of conservation laws should play a~role in mechanisms, then this aspect needs some elaboration in Ghins' account.

The following contributions emphasize levels, as both the methodological and metaphysical elements of NMP. In the M. Buzzoni chapter, levels are taken to be crucially perspectival, i.e., relative to a~context, a~point of view, and the goals of scientists. This new framework is helpful in clarifying various types of intertheoretical relationships. Levels are relative to a~choice of theoretical perspective, and ‘the question concerning the sameness of mechanism or level cannot be answered without a~perspectival approach'
%\label{ref:RNDkaTPI4tJxt}(2019, p.118).
\parencite*[][p.118]{falkenburg_mechanistic_2019}. %
 Buzzoni lines up with those who emphasized the context-relativity of mechanisms (Pâslaru, Rueger, McGivern, etc.) and somehow against Craver, who believes more in an ontic concept of mechanistic explanation that is more-or-less context-independent. Buzzoni offers a~few examples of relationships among knowledge claims in science: ‘strong relations' (when two theories contribute together to understanding) and ‘weak relations' (when two theories compete and overlap in respect of evidence) being the most relevant. Buzzoni does not show how these intertheoretical relationships clearly affect the debates on the nature of mechanisms. One suggestion would be to connect more explicitly with the epistemic account of mechanistic explanations. The whole discussion on intertheoretical relations would benefit from an example drawn from the mechanistic literature. It is also unclear how different mechanistic explanations based on different levels and perspectives compete or complement each other.

The chapter from H. Lenk moves the discussion on mechanisms into a~different conceptual landscape: the cognitive process of interpreting mechanisms, called here ‘schematizations' and ‘interpretative constructs'. They are mental/cognitive higher-level constructs resulting from interpretative processes at the representational level. Lenk aims to offer a~meta-theoretical and methodological approach to NMP. This connects to the standard difference between mechanism schemas and mechanisms sketches in Machamer et al.
%\label{ref:RNDG6lX5EtsSV}(2000).
\parencite*[][]{machamer_thinking_2000}. %
 The approach is inspired by both Kant's forms of judgment and Cassirer's levels of interpretation, and applies directly to how mechanisms are schemas interpreting causation. Lenk uses Woodward's 
%\label{ref:RNDaN3gvLLNuu}(2013)
\parencite*[][]{woodward_mechanistic_2013} %
 attempt to limit the applicability of NMP and avoid the limitless scope of mechanistic explanations. This does partially address the question (3) by explicitly stating distinguishing levels and meta-levels in a~cross-disciplinary sense. Lenk urges the New Mechanists to think about how lower-level mechanisms relate to those on higher-level and how these are interpretative schemas of causal processes.

The reviewer is not yet convinced that such a~general idea of multi-level interpretation can relate directly to the link we need to draw between mechanisms in the special sciences and physics. Lenk does not immediately bring in case studies of interpretation schema that could connect physics to special sciences.

J. Faye critically discusses the hierarchy and multi-level of reality needed in NMP. For pragmatic reasons, a~multi-level reality worldview, \textit{e.g.} what Oppenheim \& Putnam
%\label{ref:RNDP6dJbpnftE}(1958)
\parencite*[][]{oppenheim_unity_1958} %
 suggested in the 1950s, can be justified. But is the metaphysical assumption of multilevel reality suspect? One problem with the vertical view of reality is the causal impotence of higher-order causes and the logical inconsistency of downward causation. Neither reductionism nor emergentism can solve some of the puzzles of the vertical view. As this vertical perspective is questioned, Faye reconsiders some alternatives. Most notably, the horizontal view of reality in which systems exist at the same level but have categorical properties and dispositions. Dispositions are relational properties of systems that ‘cannot be merely grounded in the intrinsic properties of the categorical basis' 
%\label{ref:RNDvgzLUTIQ4W}(2019, p.177).
\parencite*[][p.177]{falkenburg_mechanistic_2019}. %
 A~manifested disposition is an ‘extrinsic property of the system brought into existence by its interaction with an environment E' 
%\label{ref:RNDs3Rl1rJHpU}(2019, p.179).
\parencite*[][p.179]{falkenburg_mechanistic_2019}. %
 Therefore, the interaction with an environment is enough in Faye's horizontal perspective to include NMP \textit{sans} the multi-level ontology. Faye does not immediately address how physics can use mechanistic models even in this horizontal framework. The only example---the flock of starlings---does not belong to physics directly so it is not clear how it illustrates NMP's generalization to physics (if any).

\enlargethispage{1.5\baselineskip}
M. Kuhlmann introduces the idea of what's called ‘econophysics', a~putative discipline that would seek to import models from statistical physics into economics. For example, ferromagnets and financial markets act similarly. Kuhlmann restricts this analysis to a~methodological pluralism embedded in an ontological reductionism. They both illustrate the macroscopic behavior of a~system based on the interaction of the micro-components. Many details of the microscopic components do not matter in both cases. Most critics of econophysics argued that the similarities between an Ising model in physics and what happens with financial markets are insufficient to build models in economics based on models in condensed matter physics. Kuhlmann thinks it is more appropriate to work with mechanisms in both disciplines rather than mathematical models: the mechanistic account of explanation is premised on the ‘interactive organization (between the parts of the mechanism) that does the explaining. And sometimes it is not all the details of the interactive organization that matter but just some structural details of it'
%\label{ref:RNDoxEcWlr0jq}(2019, p.195).
\parencite*[][p.195]{falkenburg_mechanistic_2019}. %
 The lack of a~renormalization group or scale invariance in the financial market are such details that do not matter. What is then a~structural detail? The suggestion is to move from causal explanation accounts to structural accounts and, hence, structural mechanisms. If one stays at the structural level, ‘there is a~common mechanism in diverse systems such as ferromagnets and financial markets' 
%\label{ref:RNDocKv0JrmcI}(2019, p.198).
\parencite*[][p.198]{falkenburg_mechanistic_2019}. %
 The promised structural notion of mechanism is unfortunately underdeveloped in this material. One suggestion is that model building is more important for structural mechanisms than explanation.

Another problem is that Kuhlmann does not immediately relate structures in structural mechanisms to mathematical structures governing the behavior of the two systems. This would be an exciting add-on to his perspective to recognize the importance of mathematical similarities in the structures of the two systems. Last, we also have computational models in two disciplines that use similar numerical simulations to obtain predictions from one area Ising model to financial markets. In what sense are these models related to NMP? It is a~question worth pursuing
%\label{ref:RND0e6fD3LeCL}(see a~possible connection in Humphreys, 2019).
\parencite[see a~possible connection in][]{humphreys_knowledge_2019}.%


The last contribution by V. Fano, P. Graziani, M. Tagliaferry, \& G. Tarozzi addresses how to relate a~physical system to a~given computation. When one has an abstract model of a~computation, what exactly can implement it in a~physical system? Fano et al. offer an alternative to Piccinini's
%\label{ref:RNDmggUJ71uqQ}(2015)
\parencite*[][]{piccinini_physical_2015} %
 view about realization (\textit{aka} implementation) and adopt a~more or less standard ‘mapping' strategy between a~Turing machine's states and a~physical system's states. Fano et al. depart from Piccinini's standard mechanistic approach and limit implementations by physical laws and presumably avoid some standard objections by Putnam and Kripke 
%\label{ref:RND8KrpaeEjhW}(2019, pp.217–218).
\parencite*[][pp.217–218]{falkenburg_mechanistic_2019}. %
 As in the case of the previous contribution, it is not immediately clear whether Fano \textit{et al.} actually generalize or employ any of NMP's assumptions in their approach.

The overall result of this volume is impressive, but the reviewer has several questions looming at the end. The reader can find some clear answers to questions (1), (2), and (4) as well as contributions that do not offer that much with respect to the title and the main aim of the book. For example, the standard model of particle physics quantum field theory (for example, Feynman diagrams, gauge theories, or Higgs mechanisms) is sporadically mentioned but never directly addressed. Of course, to address the question (3), it would be beneficial to integrate fundamental physics, probably even including attempts to discuss mechanisms in quantum gravity. It is also strange the spacetime theories are barely mentioned in this volume. Dieks' contribution relates immediately to Glennan's and Kuhlmann's papers
%\label{ref:RNDJ0aqrYVH3I}(Kuhlmann and Glennan, 2014; Kuhlmann, 2017),
\parencites[][]{kuhlmann_relation_2014}[][]{kuhlmann_mechanisms_2017}, %
 but a~more intricate work would address the sum rule (cursorily discussed by Falkenburg) as well as semi-classical models in quantum mechanics where one can intuit forms of mechanistic reasoning---\textit{ditto} about areas of modern physics such as cosmology, astrophysics or condensed matter. Finally, there is a~feeling that chemistry is mainly ignored in this volume. Materials about mechanisms in fundamental physics (particle physics and perhaps quantum gravity), chemistry, or biochemistry would complete this outstanding volume.

\autorrec{Ioan Muntean}
\autorrecaffil{University of Illinois Urbana-Champaign}

\end{recengenv}
