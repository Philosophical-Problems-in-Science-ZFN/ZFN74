\begin{artengenv}{Mark Povich}
	{A conventionalist account of distinctively mathematical explanation}
	{A conventionalist account of distinctively mathematical explanation}
	{A conventionalist account of distinctively mathematical\\explanation}
	{University of Rochester}
	{Distinctively mathematical explanations (DMEs) explain natural phenomena primarily by appeal to mathematical facts. One important question is whether there can be an ontic account of DME. An ontic account of DME would treat the explananda and explanantia of DMEs as ontic items (ontic objects, properties, structures, etc.) and the explanatory relation between them as an ontic relation
	%\label{ref:RNDmqCpelKAoq}(e.g., Pincock, 2015; Povich, 2021).
	\parencites[e.g.,][]{pincock_abstract_2015}[][]{povich_narrow_2021}.
	 Here I~present a~conventionalist account of DME, defend it against objections, and argue that it should be considered ontic. Notably, if indeed it is ontic, the conventionalist account seems to avoid a~convincing objection to other ontic accounts 
	%\label{ref:RNDmvr1OEkooz}(Kuorikoski, 2021).
	\parencite[][]{kuorikoski_there_2021}.%
	}
	{new mechanical philosophy, mechanistic explanation, ontic, epistemic, explanatory norms, explanatory constraints.}




\section{Introduction}
\lettrine[loversize=0.13,lines=2,lraise=-0.03,nindent=0em,findent=0.2pt]%
{D}{}istinctively mathematical explanations\footnote{Also sometimes called ``extra-mathematical explanations''
%\label{ref:RNDQxzDipnlYQ}(e.g., Baron, 2016; 2020).
\parencites[e.g.,][]{baron_explaining_2016}{baron_counterfactual_2020}.
 } (DMEs) explain natural phenomena primarily by appeal to mathematical facts. DMEs have been receiving a~lot of attention for a~few good reasons 
%\label{ref:RNDd5qcOqvBva}(Steiner, 1978; Colyvan, 1998; Baker, 2005; 2009; Mancosu, 2008; Saatsi, 2011; 2012; 2016; Lyon, 2012; Lange, 2013; 2016; 2018; Pincock, 2015; Reutlinger, 2016; Craver and Povich, 2017; Povich, 2020; 2021).
\parencites[][]{steiner_mathematics_1978}[][]{colyvan_can_1998}[][]{baker_are_2005}[][]{baker_mathematical_2009}
[][]{mancosu_mathematical_2008}
[][]{saatsi_enhanced_2011}[][]{saatsi_mathematics_2012}[][]{saatsi_indispensable_2016}
[][]{lyon_mathematical_2012}
[][]{lange_what_2013}[][]{lange_because_2016}[][]{lange_reply_2018}
[][]{pincock_abstract_2015}[][]{reutlinger_is_2016}[][]{craver_directionality_2017}
[][]{povich_modality_2020}[][]{povich_narrow_2021}. %
 Some philosophers 
%\label{ref:RNDVYT0mVgnmw}(e.g., Baker, 2005; 2009; contra Bangu, 2008; and Saatsi, 2011)
\parencites[e.g.,][]{baker_are_2005}[][]{baker_mathematical_2009}[contra][]{bangu_inference_2008}[and][]{saatsi_enhanced_2011}
 take them to play a~crucial role in enhanced indispensability arguments, providing good evidence for the existence of the mathematical objects to which they appeal. One important question is whether there can be an ontic account of DME. An ontic account of DME would treat the explananda and explanantia of DMEs as ontic items (ontic objects, properties, structures, etc.) and the explanatory relation between them as an ontic relation 
%\label{ref:RNDu0QCIecZD4}(e.g., Pincock, 2015; Povich, 2021).
\parencites[e.g.,][]{pincock_abstract_2015}{povich_narrow_2021}
 Here I~present a~conventionalist account of DME, defend it against objections, and argue that it should be considered ontic. Notably, the conventionalist account seems to avoid a~convincing objection to other ontic accounts 
%\label{ref:RND8aPKFh5Vhk}(Kuorikoski, 2021).
\parencite[][]{kuorikoski_there_2021}. %
 I~take my arguments to be far from conclusive, but to show that such a~view is worth considering.

In Section 2, I~elaborate on DME and present a~paradigmatic example that I~will use throughout the paper. My arguments ought to apply, \textit{mutatis mutandis}, to other examples.\footnote{I~discuss many more examples in my forthcoming book
%\label{ref:RNDaw681UWJFh}(Povich, 2024).
\parencite[][]{povich_rules_2024}.%
} In Section 3, I~briefly explain two recent ontic accounts of DME from Pincock 
%\label{ref:RND5U7XhNRPdW}(2015)
\parencite*[][]{pincock_abstract_2015} %
 and myself 
%\label{ref:RND5m41TcqfUm}(Povich, 2021).
\parencite[][]{povich_narrow_2021}. %
 I~will also recount Kuorikoski's 
%\label{ref:RNDgrMlYwZAvk}(2021)
\parencite*[][]{kuorikoski_there_2021} %
 objection to ontic accounts of DME. In Section 4, I~explain the kind of conventionalism to which I~will appeal. In Section 5, I~explain how to give the previously presented ontic accounts a~conventionalist twist, which deflates their platonism. This deflating allows ontic accounts to escape Kuorikoski's objection. However, one might legitimately wonder whether deflated accounts are still ontic.\footnote{One might also legitimately wonder whether there \textit{are} DMEs. For the sake of this paper, I~assume that there are.} In Section 6, I~argue that they are. In Section 7, I~respond to objections.

\section{Distinctively Mathematical Explanations}
DMEs work primarily by showing a~natural explanandum to follow in part from a~mathematical fact---a fact modally stronger than any fact about causes, mechanisms, and even natural laws. A~DME shows that the explanandum had to happen, in a~sense stronger than any ordinary causal law can supply
%\label{ref:RNDh56BsHW3hR}(Lange, 2013).
\parencite[][]{lange_what_2013}. %
 One example of DME, which I~will use throughout, is Trefoil Knot 
%\label{ref:RND1Dg3laPj7h}(Lange, 2013).
\parencite[][]{lange_what_2013}. %
 The explanandum is the fact that Terry failed to untie his knot. The explanantia are the empirical fact that the knot is a~trefoil knot and the mathematical fact that the trefoil knot is distinct from the unknot (i.e., mathematically cannot be untied). The explanantia mathematically ensure that Terry fails to untie the knot, for his success is mathematically impossible.

There are several distinctive features of DME, accounting for which serve as desiderata for any account of DME:

\textit{The Modal Desideratum}: an account of DME should accommodate and explicate the modal import of some DMEs.
%\label{ref:RNDsrCfbwgYJp}(Baron, 2016)
\parencite[][]{baron_explaining_2016}%


As I~just mentioned, there is a~modal robustness to Terry's failure---he had to fail. An account of DME should capture and, preferably, explicate that modal force.

\textit{The Distinctiveness Desideratum}: it should distinguish uses of mathematics in explanation that are distinctively mathematical from those that are not.
%\label{ref:RNDXLjd6te7Hq}(Baron, 2016)
\parencite[][]{baron_explaining_2016}%


This is emphasized by defenders of the enhanced indispensability argument
%\label{ref:RNDdV33l54PzS}(EIA, e.g., Baker, 2009).
\parencite[EIA, e.g.,][]{baker_mathematical_2009}
 According to the EIA, we ought to believe in the existence of (certain) mathematical objects because they play an indispensable explanatory role in science. The examples to which defenders of the EIA appeal are DMEs. For them, what is distinctive about DMEs is the explanatory---not merely representational---role that mathematics plays in them. Bromberger's 
%\label{ref:RNDr6ZBZNc7iO}(1966)
\parencite*[][]{bromberger_why-questions_1966} %
 flagpole is a~well-known example of an explanation that uses mathematics but is not a~DME. The explanandum is the fact that the length of a~flagpole's shadow is $L$. The explanantia are the empirical facts that the angle of elevation of the sun is $\theta$ and that the height of the flagpole is $H$ and the mathematical fact that \textit{tan $\theta  = H/L$}. Most party to the debate on DME agree that this is not a~DME. Precise explanations of why may depend on one's account of DME, but the central idea is that in this example the mathematics is playing a~merely representational role, where this means that the mathematics is merely representing what is in fact doing the real explanatory work (e.g., the physical causes). Any account of DME should count Trefoil Knot as a~DME and not Bromberger's flagpole.

\textit{The Directionality Desideratum}: it should accommodate the directionality of DMEs.
%\label{ref:RNDpNCryhTOo7}(Craver and Povich, 2017)
\parencite[][]{craver_directionality_2017}%


Craver and I~argue that Trefoil Knot can be ``reversed''\footnote{These are not strict reversals---simple swaps of explanandum and explanans---like the well-known reversal of Bromberger's flagpole. Henceforth, I~will drop the scare quotes. } to form an argument that fits Lange's
%\label{ref:RNDAa0GuWxb6D}(2013)
\parencite*[][]{lange_what_2013} %
 account of DME but is not explanatory. Simply take the explanandum and the empirical premise, swap and negate them, akin to turning a~\textit{modus ponens} into a~\textit{modus tollens}. Thus, the ``reversed'' explanandum is the fact that Terry's knot is not trefoil. The empirical explanans is the fact that Terry untied his knot. The mathematical explanans is the same: the trefoil knot is distinct from the unknot. Reversed Trefoil Knot and other such reversals should not count as DMEs.

\section{Ontic Accounts of DME}
In this section, I~briefly present two ontic accounts of DME: Pincock's
%\label{ref:RNDTciBy7Emua}(2015)
\parencite*[][]{pincock_abstract_2015} %
 abstract dependence account and my Narrow Ontic Counterfactual Account 
%\label{ref:RNDbhq5S1147V}(NOCA; Povich, 2021).
\parencites[NOCA;][]{craver_constitutive_2021}. %
 I~focus on ontic accounts since these are especially well-suited for EIAs, and I~focus on Pincock's and mine in particular since these are two of the few that are explicitly billed as ontic.\footnote{Reutlinger 
%\label{ref:RND8lBeEyDQSV}(2016)
\parencite*[][]{reutlinger_is_2016} %
 also gives a~counterfactual account of DME, but it is not explicitly ontic, nor does it rely on countermathematicals. If it is ontic, my arguments to follow apply.} According to these accounts, purely mathematical claims refer to Platonistic facts and applied mathematical claims refer to instantiations of mathematical objects. Other accounts of DME that are not explicitly billed as platonistic could be given an platonistic spin, and my arguments that follow plausibly undermine their ability to serve in EIAs as well.\footnote{For example, Lange 
%\label{ref:RNDHz16LgNSws}(2013)
\parencite*[][]{lange_what_2013} %
 specifically refers to his account as a~modal, rather than an ontic, one. According to his, in a~DME, the purely mathematical premises refer to facts ``modally stronger than ordinary causal laws'' and the empirical premises refer to facts that are ``understood to be constitutive of the physical task or arrangement at issue,'' a~condition that is never clearly explicated. Lange 
%\label{ref:RNDaukI8SLSY4}(2021)
\parencite*[][]{lange_what_2021} %
 explicitly addresses the metaphysics of DME and defends ``Aristotelian realism,'' according to which ``mathematics concerns mathematical properties possessed by physical systems,'' which is explicitly anti-Platonist. Lange's Aristotelian realist construal of DME will obviously be of no use in an EIA, but one could give Lange's 
%\label{ref:RNDoIkoor7Zay}(2013)
\parencite*[][]{lange_what_2013} %
 basic account an ontic spin, e.g., by claiming that the facts ``modally stronger than ordinary causal laws'' are Platonistic facts and the empirical facts that are ``understood to be constitutive of the physical task or arrangement at issue'' are instantiations of mathematical objects. The arguments of this paper would then straightforwardly apply to that ontic account.} Ontic accounts like mine explicitly rely on countermathematicals, which I~address below, but I~do not take such reliance to be essential to ontic accounts.

I~think that, ultimately, my account is basically a~version of Pincock's, though I~elaborate the kinds of counterfactual that the ontic relation of instantiation supports, or must support to figure in a~DME, and I~explicitly argue that the resulting account satisfies the above desiderata. Also, it is important to note that Pincock did not intend to give an account of DME. He wanted to argue 1) that there is a~kind of explanation involving abstract entities, which he called ``abstract explanation,'' 2) that abstract explanation is not causal, and 3) that causal explanation and abstract explanation both count as explanation in virtue of providing information about objective dependence relations. It is clear though that the examples usually given of DME, including Trefoil Knot, are abstract explanations in Pincock's sense. Being ontic accounts of DME that rely on ontic relations between abstract mathematical and concrete phenomena, Pincock's and mine would be especially suited to enhanced indispensability arguments
%\label{ref:RNDv3JhQx4QXd}(Baker, 2009)
\parencite[][]{baker_mathematical_2009}%
---if either account is the right account of DME, then, it would seem, platonism straightforwardly follows. However, my central argument is that conventionalism about mathematical necessity undermines this inference by deflating ontic accounts of DME, yet, arguably, does not undermine their status as ontic accounts.

I~start with Pincock's
%\label{ref:RNDSuX4pPqtP3}(2015)
\parencite*[][]{pincock_abstract_2015} %
 account. Pincock motivates his abstract dependence account using the explanation of Plateau's three laws for soap-film surfaces and bubbles:

First, a~compound soap bubble or a~soap film spanning a~wire frame consists of flat or smoothly curved surfaces smoothly joined together. Second, the surfaces meet in only two ways: Either exactly three surfaces meet along a~smooth curve or six surfaces (together with four curves) meet at a~vertex. Third, when surfaces meet along curves or when curves and surfaces meet at points, they do so at equal angles. In particular, when three surfaces meet along a~curve, they do so at angles of 120 with respect to one another, and when four curves meet at a~point, they do so at angles of close to 109.
%\label{ref:RNDfyRgHrM1oc}(Almgren and Taylor, 1976, p.82; quoted in Pincock, 2015, p.858)
\parencites[][p.82]{almgren_geometry_1976}[quoted in][p.858]{pincock_abstract_2015}%


The explanation for these laws relies on the mathematical proof that certain mathematical objects called ``almost minimal sets'' satisfy Plateau's three laws and that soap films instantiate almost minimal sets. As Pincock writes, ``Many mathematical structures have concrete systems as instances. The almost minimal sets have soap films as some of their instances, and this is what makes facts about sets relevant to facts about soap films''
%\label{ref:RNDOQRNG55vL4}(Pincock, 2015, pp.865–866).
\parencite[][pp.865–866]{pincock_abstract_2015}.%


Pincock suggests that the kind of explanation involved here---so-called abstract explanation---is akin to causal explanation on Woodward's
%\label{ref:RNDcdd5tj9yuR}(2003)
\parencite*[][]{woodward_making_2003} %
 interventionist account, though shorn of its interventionism. Woodward emphasizes that the ability to answer what-if-things-had-been-different questions (i.e., w-questions) regarding the explanandum---thus, knowledge of information about counterfactual dependence relations---is constitutive of explanation. Woodward even suggests that there could be non-causal explanations in cases where information about counterfactual dependence relations is provided, but those relations cannot sensibly be interpreted as involving interventions 
%\label{ref:RNDlTVy6BAYqo}(Woodward, 2003, p.221).
\parencite[][p.221]{woodward_making_2003}. %
 Similarly, for Pincock, what makes both causal and abstract explanations \textit{explanations} is that they reveal ``objective dependence relations''. The relation between almost minimal sets and soap films is not a~causal one, but it is, Pincock argues, a~kind of objective dependence relation---what he calls abstract dependence\footnote{Pincock is not clear about the relation between abstract dependence and counterfactual/countermathematical dependence, but the very idea of a~dependence relation, as well as the comparison with Woodward's account, seems to imply counterfactual dependence, regardless of whether the dependence relation in question can be \textit{reduced} to or \textit{analyzed} in terms of counterfactuals. Regardless, my argument that conventionalism renders invalid any EIA starting from Pincock's account and ending in Platonism does not depend on this.}. In the case at hand, Pincock suggests that the abstract dependence relation in question is that of being an instance or instantiation 
%\label{ref:RNDEKwBYukKgF}(Pincock, 2015, p.865).
\parencite[][p.865]{pincock_abstract_2015}. %
 Thus, an abstract explanation (or at least one important kind of abstract explanation) seems to be an explanation in which a~concrete object is shown to be a~certain way because it is an instantiation of an abstract object that is that way.

I~also generalizes Woodward's interventionism within an ontic account of explanation, and I~also suggests that the ontic relation involved in DMEs is instantiation
%\label{ref:RNDfPdv7YOdxu}(Povich, 2018; 2021).
\parencites*[][]{povich_minimal_2018}[][]{povich_narrow_2021}. %
 However, I~was specifically concerned with giving an account of DME that satisfies the above desiderata. According to my Narrow Ontic Counterfactual Account (NOCA), an explanation is a~DME just in case either a) it shows a~natural fact (weakly) necessarily to depend counterfactually only on a~mathematical fact, or b) it shows a~natural event to be necessitated by a~component natural fact that (weakly) necessarily counterfactually depends only on a~mathematical fact. For example, the fact that Terry's trefoil knot is distinct from the unknot fact (weakly) necessarily counterfactually depends only on the mathematical fact that the trefoil knot is distinct from the unknot. This means that in every world where Terry's trefoil knot is distinct from the unknot, that fact counterfactually depends only on the mathematical fact that the trefoil knot is distinct from the unknot. The fact that Terry did not untie his trefoil knot is necessitated by the fact that Terry's trefoil knot is distinct from the unknot; and the former fact ``contains'' the latter fact (i.e., all the objects and properties that compose the latter fact are present in the former). Thus, both the fact that Terry's trefoil knot is distinct from the unknot and the fact that he did not untie his trefoil knot are (according to clauses a) and b) of NOCA, respectively) distinctively mathematically explained by the mathematical fact that the trefoil knot is distinct from the unknot.

I~argued that ontic structures, objects, etc. are required to be the truthmakers for the counterfactuals to which NOCA appeals. According to NOCA, in DMEs there is a~kind of counterfactual dependence of natural facts on mathematical facts. What \textit{is} this dependence relation? I~suggested two possible ontic relations---grounding and instantiation---but I~prefer instantiation. For example, the concrete natural fact that Terry's trefoil knot is distinct from the unknot is an instantiation of the abstract mathematical fact that the trefoil knot is distinct from the unknot.\footnote{I.e., the object and property that compose the natural fact are instantiations (or realizations) of the object and property that compose the mathematical fact.} Thus, the following is equivalent (at least extensionally) to NOCA: an explanation is a~DME just in case either a) it shows a~natural fact to be an instantiation of a~mathematical fact, or b) it shows a~natural event to be necessitated by a~component natural fact that instantiates a~mathematical fact.

I~will not go through the rigmarole of explaining how NOCA is supposed to satisfy the three desiderata above, but I~want to point out that Platonism plays no role in NOCA's ability to satisfy them. I~appealed to Platonic objects to provide truthmakers for the relevant counterfactuals. That Platonism plays no role in NOCA's ability to satisfy the three desiderata can be seen by the fact that the two clauses of NOCA satisfy all three desiderata by themselves, without relying on any specific metaphysics of mathematics. In the case of NOCA, unlike other appeals to the ontic
%\label{ref:RNDZqmHop1WyE}(e.g., to causation in the case of Bromberger's flagpole; Salmon, 1984; 1989),
\parencites[e.g., to causation in the case of Bromberger's flagpole;][]{salmon_scientific_1984}{salmon_four_1989}
 the ontic is unnecessary to secure explanatory directionality and thus satisfy the directionality desideratum. In the present paper, I~will argue that Platonism is not required to provide truthmakers for the relevant counterfactuals. Conventionalism gives the nominalist a~way to understand the counterfactuals involved in DMEs without positing abstract mathematical facts or abstract ontic dependence relations.

Note that ontic dependence accounts of DME must rely on countermathematicals: counterfactuals with mathematically impossible antecedents. For example, I~argued that of the following countermathematicals, the first but not the second is weakly necessarily true, that is, true in every world where Terry's trefoil knot is distinct from the unknot:

Were the trefoil knot isotopic to the unknot, Terry's trefoil knot would have been isotopic to the unknot.

Were the trefoil knot isotopic to the unknot, Terry's untieable\footnote{By ``untieable,'' I~mean ``able to be untied,'' not ``unable to be tied''.} knot would have been trefoil.

Of course, on a~standard Lewis'
%\label{ref:RNDGaPJI1Stcn}(1973)
\parencite*[][]{lewis_counterfactuals_1973} %
 semantics of counterfactuals, all counterpossibles (i.e., counterfactuals with impossible antecedents) are vacuously true. One straightforward amendment to the Lewisian account is to introduce impossible worlds 
%\label{ref:RND1r81DbbJBp}(Brogaard and Salerno, 2013; see Kocurek forthcoming for a~survey of approaches to counterpossibles).
(\cite[][]{brogaard_remarks_2013}; see \citeauthor{kocurek_counterpossibles_2021} forthcoming for a~survey of approaches to counterpossibles). %
 One feature of conventionalism is that it can provide an account of countermathematicals that avoids ontological commitment to impossible worlds. More on this in the next section.

Before that, I~want to describe Kuorikoski's
%\label{ref:RNDki4qKSwXXc}(2021)
\parencite*[][]{kuorikoski_there_2021} %
 objection to ontic accounts of DME. Kuorikoski argues that any ontic account of DME, such as Pincock's or mine, cannot accommodate the Woodwardian ‘same-object condition,' which requires that in counterfactual reasoning we really are reasoning about the \textit{same} object under different conditions. According to Kuorikoski, when reasoning \textit{countermathematically} we cannot distinguish whether we are conceiving of a~change in a~given mathematical structure or simply a~different mathematical structure. As Kuorikoski puts the objection, ``if there is no difference between changing a~specific property of a~mathematical object into something else and simply contemplating the properties of a~different mathematical object, we lose the very distinction between explanatory and classificatory information'' 
%\label{ref:RNDJZSzhn2Nwt}(Kuorikoski, 2021, p.197).
\parencite[][p.197]{kuorikoski_there_2021}. %
 The idea is that, in countermathematicals like ``Were the trefoil knot isotopic to the unknot, Terry's trefoil knot would have been isotopic to the unknot,'' I~did not give a~stipulation-independent reason to think that a~‘trefoil knot' isotopic to the unknot would still \textit{be} a~trefoil knot. (Nor did Pincock provide anything similar, \textit{if} his account needs to rely on countermathematicals.) This is required for the counterfactual to express an explanatory relationship between antecedent and consequent. Such stipulation-independent reasons would basically amount to a~theory of the essential and accidental properties of all mathematical objects involved in DMEs. Not only is this task daunting, but there is no guarantee that upon its completion, all the countermathematicals involved in DMEs will come out as same-object-satisfying, i.e., that they will involve countermathematicals whose antecedents state changes in the object's accidental properties. And even if by sheer luck all countermathematicals involved in current DMEs come out as same-object-satisfying, there seems nothing to prevent a~DME that appeals to the essential properties of a~mathematical object, failing to make the associated countermathematical same-object-satisfying. E.g., suppose that being prime is an essential property of 3-3 wouldn't be 3 if it weren't prime. There's no guarantee that there are no DMEs that appeal to the fact that 3 is prime. The countermathematical in that case would be ``if 3 weren't prime, …'' which by assumption isn't same-object-satisfying.

\section{Conventionalism}
There are two conventionalistic philosophies that for the purposes of this paper I~will treat as equivalent:Amie Thomasson's
%\label{ref:RND2jiXZRaSkC}(Thomasson, 2020; 2021)
\parencites*[][]{thomasson_norms_2020}[][]{thomasson_how_2021} %
 modal normativism and Jared Warren's 
%\label{ref:RNDWXX6GYZLtJ}(2020)
\parencite*[][]{warren_shadows_2020} %
 conventionalism 
%\label{ref:RNDDaxm9zG8rz}(see also Kocurek, Jerzak and Rudolph, 2020; Sidelle, 1989).
\parencites[see also][]{kocurek_against_2020}[][]{sidelle_necessity_1989}. %
 I~will help myself to the language of both in the following sections, and I~think that either can provide adequate deflations of the previously described ontic accounts. I~will briefly explain both views, and why I~will treat them as equivalent.

Thomasson's
%\label{ref:RNDNFqBVUrbeH}(2020; 2021)
\parencites*[][]{thomasson_norms_2020}[][]{thomasson_how_2021} %
 modal normativism is somewhat similar to expressivism about metaphysically necessity and possibility. Although Thomasson's normativism concerns specifically \textit{metaphysical} modality, it is easily generalizable to mathematics. According to mathematical normativism, mathematical claims do not describe, in any \textit{substantive} sense, anything, but instead are object language \textit{expressions} of conceptual/semantic\footnote{I~ignore this distinction here. This should not affect my argument.} rules\footnote{Rules which may include empirical variables to account for \textit{a~posteriori} necessities 
%\label{ref:RNDpLIRWpKKAE}(Sidelle, 1989; Thomasson, 2020; Warren, 2022b),
\parencites[][]{sidelle_necessity_1989}[][]{thomasson_norms_2020}[][]{warren_priori_2022}, %
 but these, as well as \textit{de re} necessities, are irrelevant to the present discussion.} or consequences thereof\footnote{This is reminiscent of Wittgenstein 
%\label{ref:RNDveCPDVTLw2}(1978; 2013).
\parencites*[][]{wittgenstein_remarks_1978}[][]{wittgenstein_tractatus_2013}.%
}. A~mathematical claim such as ``3 is prime'' is an expression of a~semantic rule according to which it is correct to apply ``is prime'' when it is correct to apply ``3''.\footnote{Throughout, when I~say ``mathematics'' or ``mathematical,'' I~am talking about pure mathematics. I~leave aside how conventionalists can account for applications of mathematics. See Warren 
%\label{ref:RNDBKxUC33qca}(2020)
\parencite*[][]{warren_shadows_2020} %
 for a~discussion of conventionalism and applicability.}

I~say that Thomasson's normativism is ``somewhat'' similar to expressivism, because she accepts the existence of modal truths, facts, etc., as long as all of these terms are understood in suitably deflationary senses that are clearly distinguished from the senses these terms have in talk of non-modal, empirical truths, facts, etc.
%\label{ref:RNDiqR1lXqvf2}(Thomasson, 2020; see also Baker and Hacker, 2009).
\parencites[][]{thomasson_norms_2020}[see also][]{baker_mathematical_2009}.%
\footnote{Obviously, the problem of so-called ``creeping minimalism'' in metaethics 
%\label{ref:RNDQ2YiFqHUje}(Dreier, 2004)
\parencite[][]{dreier_metaethics_2004}%
---i.e., how to distinguish moral expressivism from moral realism once the expressivist adopts semantic minimalism or deflationism---applies here. Here I~merely point the reader to what I~think is a~promising way of solving this problem 
%\label{ref:RND2ymZ30jae9}(Simpson, 2020).
\parencite[][]{simpson_creeping_2020}. %
 Adjusting Simpson's solution to the topic of modality, he would hold that normativism differs from its rivals by not having to appeal to modal facts to explain (the content of) modal language and thought. See also Brandom's 
%\label{ref:RNDF9Erbg5aw8}(2008)
\parencite*[][]{brandom_between_2008} %
 explanation of modal language. The conventionalist might also appeal to analyticity: what distinguishes conventionalism from Platonism is that the conventionalist takes mathematical claims (including existence claims) to be analytic, whereas the Platonist doesn't. This might seem to misclassify neo-Fregeans who are Platonists yet think that mathematical claims (including existence claims) are analytic 
%\label{ref:RNDQhilbgm1ux}(e.g., Hale and Wright, 2001).
\parencite[e.g.,][]{hale_reasons_2001}.
 However, Thomasson 
%\label{ref:RNDPEGfLLIHLH}(2014)
\parencite*[][]{thomasson_ontology_2014} %
 and Warren 
%\label{ref:RNDb369MWb7Jv}(2020)
\parencite*[][]{warren_shadows_2020} %
 are both generally sympathetic to neo-Fregeanism, with Warren 
%\label{ref:RND5aflc6WgKW}(2020, pp.198, 203)
\parencite*[][pp.198, 203]{warren_shadows_2020} %
 calling it ``conventionalist-adjacent,'' and it is often grouped with metaontological deflationisms or minimalisms.}

According to Warren's
%\label{ref:RND9BoO0xacnm}(2020)
\parencite*[][]{warren_shadows_2020} %
 conventionalism, all mathematical truths in a~language are fully explained by (the validity of) the basic inference rules of that language. Warren isn't exactly clear about what ``fully explained'' means. Being derivable from the basic inference rules is clearly sufficient for ``full explanation,'' as the example in the next paragraph shows.

Notably, according to Warren's
%\label{ref:RND0s19vn1bAL}(2020)
\parencite*[][]{warren_shadows_2020} %
 conventionalism, it is not the case that mathematical truths \textit{describe} conventions. You could say that arithmetical truths describe numbers because their terms refer to numbers, but such reference---and, therefore, existence---is a~trivial byproduct of our arithmetical language. For example, let us assume our arithmetical language is formally modeled by first-order Peano arithmetic, one of whose basic inference rules allows the derivation of ``N0'' (i.e., ``zero is a~number'') from no premises. From this, we can easily derive ``there is a~number'' via the introduction rule for the existential quantifier. Thus, the existence of numbers is a~trivial byproduct of our arithmetical language. Thomasson and Warren are both deflationary ``trivial realists'' in mathematical ontology.

There are some obvious differences between Warren's and Thomasson's views, but I~will treat them as equivalent for my purposes because the differences will not affect the use to which I~put them. One difference that is unimportant is Warren's emphasis on inference rules and Thomasson's emphasis on application conditions. This should not affect my arguments since Thomasson is aware that application conditions might not be the only kind of semantic rule that is expressed by necessary statements, and Warren accepts that application conditions can be meaning-determining
%\label{ref:RNDrDWfeootwN}(Warren, 2022a).
\parencite[][]{warren_inferentialism_2022}. %
 A~potentially more significant difference is the apparent fact that Thomasson is an expressivist and Warren is not---he's an inferentialist. However, Thomasson is not an expressivist in the traditional sense, which is one reason she prefers the term ‘normativism'. Normativism is not a~semantic or metasemantic thesis, like traditional expressivism, which has an ‘ideational' (meta)semantics; normativism is for Thomasson a~functional thesis, a~thesis about the function of a~piece of language. In fact, like Warren, she is an inferentialist about meaning 
%\label{ref:RNDlu3haKVlUi}(Thomasson, 2020, p.79),
\parencite[][p.79]{thomasson_norms_2020}, %
 and the functional thesis is entirely open to Warren. As my argument progresses, I~will make clear how normativists and conventionalists can say the same thing.From now on, when I~say ``conventionalist,'' I~will tend to mean Thomasson's version, but everything I~argue is open to Warren as well.

As we saw, ontic accounts of DME rely on non-vacuous countermathematicals. Fortunately, modal normativism has already been used to provide an account of non-vacuous counterpossibles with metaphysically impossible antecedents
%\label{ref:RNDvDa9KLLOiQ}(Locke, 2021)
\parencite[][]{locke_counterpossibles_2021} %
 and of non-vacuous counterpossibles with (meta)logically impossible antecedents 
%\label{ref:RNDFzjULutBPU}(Kocurek and Jerzak, 2021).
\parencite[][]{kocurek_counterlogicals_2021}. %
 Consider the counterpossible: Were Goliath (the statue) to survive being flattened, it would be an abstract object. According to Locke, such counterpossibles have non-vacuous readings that express\footnote{It is important to note that expressing what would be the case if actual semantic rules had been different is not the same as expressing actual semantic rules. According to normativism, only necessities express actual semantic rules, so only if a~counterpossible is necessary (and some may be; see below) does it express actual semantic rules. One could simply avoid talk of ``expressing'' here by saying that non-vacuous readings of counterpossibles involve changing semantic rules. Thus, when I~say, ``conventionalist account of counterpossibles,'' I~do not mean conventionalism \textit{about} counterpossibles, viz., the view that counterpossibles express actual semantic rules or consequences thereof; most do not. I~simply mean what the conventionalist says is going on in non-vacuous counterpossibles, viz., that we consider actually adopting different semantic rules.} the consequences of changing our semantic rules only as much as the antecedent demands.\footnote{This is the conventionalist analogue of ``in the nearest possible world where the antecedent is true''. In general, the conventionalist can give a~semantic interpretation of Baron, Colyvan, and Ripley's 
%\label{ref:RNDLfrkpOfYBF}(2017)
\parencite*[][]{baron_how_2017} %
 account of the evaluation of countermathematicals---instead of conceiving ourselves as ``twiddling'' mathematical facts and thinking through the ramifications, we ``twiddle'' concepts, their application conditions, etc., and think through the ramifications.} This counterpossible expresses the claim that if the application conditions of statue-names like ``Goliath'' were changed so as to continue to apply after flattening, Goliath would be an abstract object. Locke argues that this is false, because when we imagine changing the application conditions of ``Goliath'' only so much that it continues to apply after being flattened, we have not changed that part of the application conditions that ensures it only applies to concrete objects. Kocurek and Jerzak 
%\label{ref:RNDWdfRgEcq4B}(2021)
\parencite*[][]{kocurek_counterlogicals_2021} %
 argue for the same idea regarding counterfactuals with (meta)logically impossible antecedents. According to them, a~counterlogical such as ``If intuitionistic logic were correct, the continuum hypothesis would be either true or not true'' has a~non-vacuous reading that expresses the consequences of accepting the intuitionist's semantic rules for ‘or' and ‘not'. On this reading, it is false. Note that these ideas are entirely open to Warren,\footnote{He in fact cites Einheuser approvingly several times. Her work is discussed in the following paragraph of the main text.} who could treat counterpossibles as expressing what would be true according to different conventions.

For the previous claims to make sense, it is important to introduce a~distinction. Einheuser
%\label{ref:RNDN3nCbBU2zH}(2011)
\parencite*[][]{einheuser_toward_2011} %
 called readings of counterfactuals on which we consider actually adopting different semantic rules ``counterconceptual'' readings and readings on which we do not change our semantic rules ``countersubstratum'' readings.\footnote{A~counterconceptual reading of a~counterfactual bears similarities to some two-dimensionalists' notion of considering a~possible world as actual 
%\label{ref:RND6nfdXiogHU}(Stalnaker, 2001).
\parencite[][]{stalnaker_considering_2001}. %
 I~think there are many significant commonalities between conventionalism and some versions of two-dimensionalism, especially Stalnaker's, but that is beyond the scope of this paper. (See also the mention of Chalmers and Stalnaker in Section 7 below.)}\footnote{The conventionalist can agree with the Lewisian that all countersubstratum readings of counterpossibles are vacuously true. This is Kocurek and Jerzak's view.} Note that these do not refer to kinds of counterfactual but to ways of reading counterfactuals. Using this distinction, we can say that according to the conventionalist, counterpossibles are non-vacuous on counterconceptual readings.

In many instances of counterfactual reasoning, we automatically give countersubstratum readings of counterfactuals, that is, we continue to use our actual semantic rules
%\label{ref:RND3l26anzAPK}(Kripke, 1980; Wright, 1985; but see Kocurek, Jerzak and Rudolph, 2020 for cases where it is natural to give counterfactuals counterconceptual readings).
(\cites[][]{kripke_naming_1980}[][]{wright_defence_1985}[but see][]{kocurek_against_2020} for cases where it is natural to give counterfactuals counterconceptual readings). %
 It is plausible that this is how we naturally read so-called ``independence conditionals'' such as ``even if our semantic rules had been different, the necessities would not have been different'' 
%\label{ref:RNDPMxxF4i7Xw}(Thomasson, 2007a; see also Sidelle, 2009; Thomasson, 2020).
\parencites[][]{thomasson_modal_2007}[see also][]{sidelle_conventionalism_2009}[][]{thomasson_norms_2020}. %
 The conventionalist can accept this: countersubstratum readings of that counterfactual are indeed true.

\section{Conventionalism about DME}
I~propose to extend the conventionalist treatment of counterpossibles with metaphysically and (meta)logically impossible antecedents to counterpossibles with mathematically impossible antecedents. Thus, I~take non-vacuous countermathematicals to express consequences of changes in the rules governing mathematical concepts.\footnote{Here I~will not rely on any particular account of the distinction between mathematical and non-mathematical concepts, which should not matter for my argument. The distinction may turn out to be disjunctive---a mathematical concept is either an arithmetical concept or a~geometrical concept or…, where an arithmetical concept is a~concept of quantity, a~geometrical concept is a~concept of space, etc. I~do not think it is necessary for my argument that there should even be a~clear distinction. } For example, return to the countermathematicals:

Were the trefoil knot isotopic to the unknot, Terry's trefoil knot would have been isotopic to the unknot.

Were the trefoil knot isotopic to the unknot, Terry's untieable knot would have been trefoil.

The conventionalist should interpret these as expressing something like:

Were the semantic rules governing the application of the term ‘trefoil knot' such that, wherever it applied, ‘isotopic to the unknot' also applied, Terry's trefoil knot would have been isotopic to the unknot.

Were the semantic rules governing the application of the term ‘trefoil knot' such that, wherever it applied, ‘isotopic to the unknot' also applied, Terry's untieable knot would have been trefoil.

where we consider actually adopting the semantic rule specified in the antecedent. These conventionalist interpretations should preserve the original countermathematicals' truth values. Thankfully, it seems that they do---the first counterfactual above is (weakly) necessarily true and the second is not. That is, in every world where Terry has a~trefoil knot, the first counterfactual, but not the second, is true. Remember that we are to consider actually adopting the semantic rule specified in the antecedent: were we to imagine actually adopting the semantic rule that wherever ‘trefoil knot' applies, ‘isotopic to the unknot' applies, then ‘isotopic to the unknot' would have applied to the knot of Terry's to which ‘trefoil' applies; so, via semantic descent, Terry's trefoil knot would have been isotopic to the unknot.

I~just characterized these conventionalist interpretations \textit{modally}: the first is (weakly) \textit{necessarily} true and the second is not. The necessity involved here does not seem to be mathematical necessity, so the \textit{mathematical} conventionalist \textit{need} not say that it expresses a~semantic rule, but I~will suggest a~way to say just that: the first counterfactual, but not the second, is a~consequence of actual semantic rules governing the terms therein. Given that the first countermathematical is similar to a~case of universal instantiation (i.e., ``if for all x, x~is F, then a~is F''), I~suggest that it follows from semantic rules governing the logical terms involved, such as ``wherever'' and/or the counterfactual conditional ``if… were…, then… would be…''.\footnote{Does this commit me to a~kind of logicism? I~do not think so. I~would only be worried if I~were committed to the claim that all purely mathematical truths are expressions of rules governing logical concepts. I~certainly am not committed to that; and note that the countermathematical in question is not purely mathematical.}

What about clause b) of NOCA? This clause deals with necessitation, and though it is unclear whether this is \textit{mathematical} necessitation, I~will suggest a~way to say that such necessitation claims express semantic rules. According to NOCA, the fact that Terry's trefoil knot is distinct from the unknot necessitates that he will fail to untie his trefoil knot. Necessitation claims are usually cashed out as necessary conditionals: necessarily, if Terry's trefoil knot is distinct from the unknot, then he will fail to untie his trefoil knot. That conditional is arguably an expression of actual semantic rules governing the terms therein---it expresses what Thomasson class the ‘analytic entailment' of the consequent by the antecedent.

Pincock's account of abstract explanation appeals to instantiation. Instantiation (or perhaps some kind of realization) is the relation that many Platonists take to hold between a~mathematical object (or property) and the concrete objects (or properties) that are its instances. Instantiation is appealed to by both Pincock and I, and we both take it to be an ontic relation between abstract mathematical and concrete objects. For the conventionalist, instantiation should be seen not as an ontic relation but as a~semantic one. To say that some concrete object instantiates a~mathematical object is just to say that the relevant mathematical concept applies to it. Instantiation is concept application.\footnote{I~cannot here say anything about what I~take the relation of concept application to be
%\label{ref:RNDaREPpk2UEf}(though I~am sympathetic to Thomasson, 2007b).
\parencite[though I~am sympathetic to][]{thomasson_ordinary_2007}. %
 I~do not think my arguments require any particular account of that relation, though obviously an across-the-board nominalist (not merely a~nominalist about mathematical objects) will want an account that does not appeal to abstract objects.} Notice that the relation of concept application has features that are important for the explanatory aims of Pincock and I. They both rely on the asymmetry of instantiation to buttress their theories' explanatory credentials and exclude certain reversals and other potential counterexamples. Concept application too is an asymmetric relation\footnote{At least in the relevant cases, such as in DMEs, where concept application is intended to take the place of the instantiation of an abstract object by a~concrete object. As Earl Conee (personal communication) pointed out to me, perhaps the application of the concept \textit{concept} to itself is not asymmetric. However, I~do not think this is a~case where we would say a~concrete object instantiates an abstract object.}: ‘trefoil knot' (or ‘almost minimal set') applies to some concrete object, but that concrete object does not apply to ‘trefoil knot' (or ‘almost minimal set'). On NOCA, the mathematical fact that the trefoil knot is distinct from the unknot does not explain why the knot Terry untied is not trefoil because that mathematical fact is not instantiated in that natural fact\footnote{As I~have already noted, NOCA does not need the instantiation relation to exclude this reversal---the counterfactual clauses of NOCA already do that. I~suggested that the instantiation relation is what makes true the counterfactual clauses of NOCA.}. According to the conventionalist interpretation of NOCA, the mathematical fact that the trefoil knot is distinct from the unknot does not explain why the knot Terry untied is not trefoil because the mathematical concept ``trefoil knot'' does not apply to anything in that natural fact (for the same reason that the trefoil knot is not instantiated in that natural fact---there is no trefoil knot there!). On Pincock's account, abstract explanations show that a~concrete object possesses a~certain property because it is an instantiation of an abstract object that possesses a~that property. According to the conventionalist interpretation of Pincock's account, in abstract explanations some predicate is shown to apply to a~concrete object in virtue of the fact that its predication is analytically entailed by the predication of a~mathematical predicate to it. Soap films are shown to satisfy Plateau's laws because they instantiate almost minimal sets, which instantiate Plateau's laws. According to the conventionalist interpretation, soap films satisfy Plateau's laws because ‘almost minimal set' applies to them, and the application of ‘almost minimal set' to a~concrete object analytically entails the application of ‘satisfies Plateau's laws' to it.

Let us take stock so far. I~presented Pincock's
%\label{ref:RND0QHJg2WudE}(2015)
\parencite*[][]{pincock_abstract_2015} %
 and my 
%\label{ref:RNDpByLZfHbuj}(Povich, 2021)
\parencite[][]{povich_narrow_2021} %
 ontic accounts of DME. I~then explained conventionalism and showed how it can accommodate non-vacuous countermathematicals.\footnote{Baron 
%\label{ref:RNDZu6AhWPgPc}(2020)
\parencite*[][]{baron_counterfactual_2020} %
 presents an account of DME that relies on countermathematicals, but it is unclear if he takes his account to be an ontic one. } Finally, I~gave a~semantic account of the instantiation relation as concept application.

Before addressing the question of whether conventionalist interpretations of Pincock's abstract dependence account and NOCA are still ontic, let me explain how the conventionalist can escape Kuorikoski's objection. Recall that the challenge is to adhere to the same-object condition. For conventionalists, the same object is the term/concept, individuated syntactically, merely with a~different meaning/content. Of course, this means that Kuorikoski is right that countermathematicals are importantly different from standard ontic counterfactuals, and that there is something more ``representational'' about countermathematicals---this shouldn't be surprising since, after all, conventionalists take mathematical truths to express rules for the use of language---but also note how for the conventionalist countermathematicals are importantly \textit{different} from the clearly \textit{epistemic} counterfactuals with which Kuorikoski contrasts ontic counterfactuals, such as his Sisley example
%\label{ref:RNDSN2OlWX2p2}(Kuorikoski, 2021, p.196).
\parencite[][p.196]{kuorikoski_there_2021}. %
 We are to imagine that a~museum has a~policy that all and only Sisleys are hung in room 18. The counterfactual ``If this painting were in room 18, then it would be a~Sisley'' is false when read ontically but true when read epistemically as a~claim about what it would be rational to believe if the antecedent were true. But counterconceptual interpretations of countermathematicals are \textit{not} epistemic claims like this, for what is true according to a~convention is not an epistemic matter. The distinction between ontic and epistemic readings of counterfactuals cuts across the distinction between countersubstratum and counterconceptual readings of counterfactuals. The counterconceptual reading of, e.g., ``Were the trefoil knot isotopic to the unknot, Terry's trefoil knot would have been isotopic to the unknot'' does not concern what it would be rational to believe were the trefoil knot were isotopic to the unknot, nor what it would be rational to believe were the semantic/conceptual rules governing the application of the term/concept ‘trefoil knot' such that, wherever it applied, ‘isotopic to the unknot' also applied. Neither of those epistemic questions involves a~shift in conceptual scheme or convention; in that sense epistemic readings are like countersubstratum readings. The counterconceptual reading concerns what is true according to the convention specified in the antecedent. This leads us directly to our central question---whether conventionalism strips accounts of DME of their ontic status.

\section{The Ontic Status of Conventionalist Accounts of DME}
We have seen thatcounterconceptual readings of countermathematicals are not epistemic in Kuorikoski's sense; they do not concern what it would be rational to believe. But the question still remains whether conventionalism deprives ontic accounts of DME of their ontic status. Here I~present an argument that it does not. First, I~need to be clear about the ontic. By ``ontic,'' I~mean mind-independent, in the sense that whether the explanans, explanandum, and explanatory relation between them exist is not up to \textit{us explainers}. But I~do not think this should rule out conventions as ontic. I~don't think the definition of ``ontic'' should rule out the possibility of ontic explanation in the cognitive and social sciences, including sociology and linguistics; brains, beliefs, social (including linguistic) conventions, and so on are perfectly objective in the sense that matters for ontic explanation, and only on exceedingly controversial and exceedingly rare philosophical views can such things not enter into causal or other natural relations. Brains and linguistic conventions are in principle scientifically manipulable and apt to figure in causal explanations, as explanantia and as explananda.

To illustrate the ontic status of conventionalist explanations, let us simply think about Woodwardian
%\label{ref:RNDbpR6iOo0Jj}(2003)
\parencite*[][]{woodward_making_2003} %
 interventionism from the conventionalist perspective, using the distinction between counterconceptual and countersubstratum readings of counterfactuals. Take the countermathematical: ``Were the trefoil knot isotopic to the unknot, Terry's trefoil knot would have been isotopic to the unknot''. The conventionalist says we should interpret this as expressing something like: ``Were the semantic/conceptual rules governing the application of the term/concept ‘trefoil knot' such that, wherever it applied, ‘isotopic to the unknot' also applied, Terry's trefoil knot would have been isotopic to the unknot''. Read in the usual, countersubstratum way, this is simply false. Read counterconceptually, it is true. But note that this can be interpreted in terms of Woodwardian interventions, from which many ontic proponents, including Pincock and I, take our inspiration. Thus, imagine an intervention on the concept/term ``trefoil knot'' that changes the semantic/conceptual rules governing it. I~take it that such an intervention would amount to an intervention on people's brains or on their social conventions (an intervention which would again presumably have its intended effect via changes in people's brains). How exactly this could work depends on the metaphysics of concepts, which I~cannot get into here. I~will just note two important things about this suggestion: 1) Woodward does not require that interventions be physically possible, so difficulty in imagining what this would look like in practice is no objection to it. 2) We need to individuate terms/concepts syntactically, or some other way such that changes in the rules governing the term/concept do not change the term/concept itself. As I~mentioned above when addressing Kuorikoski's objection, the ‘same object' here is a~syntactically individuated term/concept. (Terms/concepts are individuated this way for Chalmers' 
%\label{ref:RNDZPu3dFKff2}(2004, pp.169–170)
\parencite*[][pp.169–170]{chalmers_epistemic_2004} %
 orthographic contextual intensions and Stalnaker's 
%\label{ref:RNDRMiIGC6eMs}(1978; 2001)
\parencites*[][]{stalnaker_assertion_1978}[][]{stalnaker_considering_2001} %
 diagonal propositions.) Otherwise, we will not have the same term/concept pre- and post-intervention.

Our intervention would change which claims the people upon whom we intervened make and which beliefs they have---they would now assert that Terry's trefoil knot is isotopic to the unknot. Of course, we, the interveners, using our actual semantic rules, would not say that, post-intervention, Terry's trefoil knot is isotopic to the unknot. We would say that Terry's trefoil knot is still distinct from the unknot, and we would take our intervention merely to have demonstrated a~causal or mechanistic relation between their brain states or social conventions and what they think and say. Of course, that is true, but the conventionalist about mathematics can say more. If \textit{we} were actually to adopt their post-intervention semantic rules, \textit{we} would say that Terry's trefoil knot is isotopic to the unknot. Counterconceptual readings of interventionist counterfactuals show that there is a~kind of ``counterconceptual causal''\footnote{Craver
%\label{ref:RND3KjOi7jkbp}(2007; Craver, Glennan and Povich, 2021)
\parencites*[][]{Craver2007}[][]{craver_constitutive_2021} %
 adjusts Woodward's interventionism to give an account of mechanistic/constitutive, rather than causal, relevance. Perhaps it would be better to say that there is a~``counterconceptual mechanistic'' dependence here, depending on what is intervened upon (e.g., the brain or social conventions).} dependence here---a dependence that one can see only by switching semantic rules. The idea here is that x~counterconceptually depends on y~just in case the counterfactual ``were ${\sim}$y the case, then ${\sim}$x would be the case'' is true on a~counterconceptual reading. So, since the counterfactual, ``Were the semantic/conceptual rules governing the application of the term/concept ‘trefoil knot' such that, wherever it applied, ‘isotopic to the unknot' also applied, Terry's trefoil knot would have been isotopic to the unknot'' is true on a~counterconceptual reading, the fact that Terry's trefoil knot is distinct from the unknot counterconceptually depends on the semantic/conceptual rules governing the application of the term/concept ‘trefoil knot'. If we think of the antecedent as brought about by an intervention, à la Woodward, then there is ``counterconceptual causal'' dependence. When we give the previous counterfactual a~counterconceptual reading and conceive the antecedent as brought about by an intervention, it is true. Thus, the fact that Terry's trefoil knot is distinct from the unknot ``counterconceptually causally'' depends on the semantic/conceptual rules governing the application of the term/concept ‘trefoil knot'.

It is important to note that the view is not that Terry failed to untie his trefoil knot because the way the mathematical concepts ``trefoil knot'' and ``unknot'' are used. The conventionalist can recognize that falsity of that claim just as she can recognize the falsity of the standard reading of the counterfactual, ``If the concept ‘trefoil knot' were used differently, Terry would've untied his trefoil knot''. Nevertheless, the view is that ``Terry failed to untie his trefoil knot because the way the mathematical concepts ‘trefoil knot' and ‘unknot' are used'' is getting at \textit{something} important in a~\textit{roundabout} way, a~way which was the purpose of this section, and the concept of counterconceptual dependence, to explicate.\footnote{I~think the ``something important'' is also brought out by similar work on conventionalism and analyticity
%\label{ref:RNDokRcGbxqYe}(e.g., Topey, 2019; Warren, 2020; Donaldson, 2021).
\parencites[e.g.,][]{topey_linguistic_2019}{warren_shadows_2020}{raven_analyticity_2021}.
 These authors, each in their own way, argue that there are some non-linguistic facts (e.g., those expressed by analytic truths) that can be explained by convention, contra opponents of truth by convention 
%\label{ref:RNDr5cdKi4UbB}(e.g., Boghossian, 1996).
\parencite[e.g.,][]{boghossian_analyticity_1996}.
} On an account like Lange's or NOCA, the explanandum of a~DME is ``narrow'' or as Lange puts it, the empirical explanans is presupposed in the context of the why-question. This makes the explanandum-statement basically analytic for a~conventionalist. Suppose we want to explain why Terry failed to make his triangle four-sided or failed to make his sister a~bachelor, say by widowing her. In these cases, I~think it is uncontroversial that it would be \textit{adequate} for an explanation of Terry's failure to cite \textit{only semantic conventions}.
%\label{ref:RNDs3n6R4XbMR}(See Donaldson, 2021 for a~defense of this kind of idea.)
(\cite[See][]{raven_analyticity_2021} for a~defense of this kind of idea.) %
 One \textit{needn't} cite semantic conventions; one could also explain his failure by appeal to the fact that bachelors are (necessarily) men. But for the conventionalist that is simply an expression of a~conceptual rule and the explanation that cites the rule directly is adequate on its own. I~submit that on an account like Lange's or NOCA any impression that DMEs are different than this is an illusion. On these views, one is simply metaphysically confused if one has in mind some more metaphysically robust explanandum for DMEs. (Compare the objection: ``But you can't explain the \textit{FACT} that Terry failed to make his sister a~bachelor by appeal to only semantic conventions!'' This betrays a~confusion about the metaphysical lightness of the fact itself. See again Topey 
%\label{ref:RNDSK368UfwAN}(2019),
\parencite*[][]{topey_linguistic_2019}, %
 Donaldson 
%\label{ref:RNDHxhVdPfAvA}(2021),
\parencite*[][]{raven_analyticity_2021}, %
 and Warren 
%\label{ref:RNDR9OMuSs0X6}(2020).
\parencite*[][]{warren_shadows_2020}.%
)

On other accounts of DME, like Pincock's, the explanandum is not narrow, so its description is not analytic. However, the conventionalist will say similar things about the mathematical premises in Pincock's account. Soap films satisfy Plateau's laws because they instantiate almost minimal sets and it is a~mathematical fact that almost minimal sets satisfy Plateau's laws. The conventionalist can agree that the fact that soap films instantiate almost minimal sets (or that ``almost minimal set'' applies to them) is an empirical, non-conventional fact. However, for the conventionalist, the mathematical fact that almost minimal sets satisfy Plateau's laws is an expression of conceptual rules. The explanatory status of this fact has the same two-faced character as the one discussed in the previous paragraph. The conventionalist 1) can accept that this mathematical fact partly explains the explanandum, and 2) can hold that ``soap films satisfy Plateau's laws in part because of how terms are used'' is false on the standard reading of that claim, yet 3) can hold that it is true on a~counterconceptual reading. For the conventionalist, DMEs on Pincock's account are no different than the following: why is Bob an unmarried man? Because Bob instantiates the property of being a~bachelor and bachelors are unmarried men. The fact that Bob is (or instantiates the property of being) an unmarried man is an empirical, non-conventional fact. ``Bachelors are unmarried men'' is an expression of conceptual rules. The explanation would be just as adequate if it appealed to a~\textit{semantic} fact here: because Bob is a~bachelor and ``bachelor'' means \textit{unmarried man}. According to the conventionalist, DMEs on a~Pincockian view are no different than this.

Note that accepting that this kind of counterconceptual dependence is explanatory seems not to require any significant revision in our ordinary concept of explanation. Kocurek, Jerzak, and Rudolph
%\label{ref:RNDVXwxSYw92e}(2020, p.7)
\parencite*[][p.7]{kocurek_against_2020} %
 point out that there are many times when we accept that counterconceptual dependence is explanatory. They give the following nice example. In 2006, the International Astronomical Union's (IAU) revised the scientific definition of ‘planet'. According to this new definition, Pluto is no longer classified as a~planet. Kocurek et al. maintain that the following claims are literally true:

Whether or not Pluto is a~planet \textit{depends} on what definition the members of the IAU agree on.

Part of what \textit{explains} why Pluto is not a~planet is the IAU's decision in 2006 to redefine ‘planet'.

\textit{Because} of the IAU's decision in 2006, Pluto is not a~planet.
%\label{ref:RNDdWzAZdN5Lw}(Kocurek, Jerzak and Rudolph, 2020, p.7, my emphasis)
\parencite[][p.7, my emphasis]{kocurek_against_2020}%


Kocurek et al. consider a~Gricean attempt to explain this away. According to the Gricean, these claims express literal falsehoods, and we should instead understand them as communicating something explicitly metalinguistic, e.g., ``Part of what explains why Pluto is not classified as a~planet is the IAU's decision in 2006 to redefine ‘planet'''
%\label{ref:RNDTCPAG7TOTQ}(2020, p.7).
\parencite*[][p.7]{kocurek_against_2020}. %
 Kocurek et al. argue in response that ``[t]he defender of this line owes us a~theory of how these utterances are transformed into explicitly metalinguistic ones. We think that the prospects for such a~theory are not good because the exact nature of the transformation into an explicitly metalinguistic sentence is highly unsystematic'' 
%\label{ref:RNDsqR9scnFbl}(2020, p.7).
\parencite*[][p.7]{kocurek_against_2020}. %
 They go on to defend this last claim, but we needn't continue it here. My point is just that accepting counterconceptual dependence as explanatory doesn't seem to do significant damage to our ordinary concept of explanation. One might try to argue that a~proper philosophical explication of the ordinary concept should exclude counterconceptual dependence, but I~have argued here we have good reasons for including it.

Now, one might claim that this isn't \textit{really} an ontic account of DME. On ontic accounts, the explanandum, explanans, and the dependence relation between them are distinct, \textit{ontic} things. Yet, what I~described in the previous paragraph is merely a~case where the \textit{same} fact is \textit{described} or \textit{conceptualized} in two different ways. It is not a~case where the explanandum ontically depends on some other fact: nothing about Terry's knot or the soap films really changed, only what people think and say about them. As I~conceded in the last section, I~think there is something right about this, namely, that the conventionalist views pure mathematics as expressing rules for the use of language. This is especially on accounts that narrow the explanandum-statements until they are analytic, like Lange's or NOCA. (On a~conventionalist interpretation of Pincock's account, there is still the empirical, non-conventional fact that soap films instantiate almost minimal sets [or that ``almost minimal set'' applies to them] that plays an explanatory [causal] role.) Still, I~do not think the ontic proponent need fear. First, the hypothetical intervention into peoples' brains or social conventions or whatever clearly \textit{is} one to which no ontic proponent would object---it plainly illustrates an ordinary ontic (causal or mechanistic) explanation of what people think and say. Second, according to the conventionalist, there \textit{just is not anything else here to explain}. Mathematics is just a~reflection of how people talk---shadows of our syntax
%\label{ref:RNDTpR77YNOyK}(Warren, 2020).
\parencite[][]{warren_shadows_2020}. %
 So, everything there \textit{is} to explain can be explained ontically. No worry for the ontic proponent, then. I~think those who would object that conventionalist accounts are not really ontic are really objecting to conventionalism, they are objecting that there must be something more to explain.

I~know that I~have not conclusively established that conventionalist accounts of DME are ontic, but I~hope to have convinced you that the claim that they are isn't as obviously false as it might have seemed.\footnote{Here is another brief argument for the ontic status of conventionalist DMEs. There are three extant conceptions of explanation: the ontic, the modal, and the epistemic
%\label{ref:RND661VXGmT9y}(Salmon, 1989).
\parencite[][]{salmon_four_1989}. %
 Conventionalist DMEs certainly don't seem to fall into a~modal or epistemic conception, for they don't show that their explananda had to occur, nor that they were expected to occur. By ``epistemic conception,'' some just mean that explanation is a~representational act. Nothing I've said here disagrees with that. For, by ``ontic conception'' I~don't mean that explanations are themselves ontic; I~just mean that they explain by appeal to the ontic 
%\label{ref:RNDBRUTyr1Jcm}(see Craver, 2014).
\parencite[see][]{kaiser_ontic_2014}. %
 Since conventionalist DMEs aren't modal or epistemic, they are ontic. Of course, this argument relies on there being only three conceptions of explanation. I~challenge those who don't think conventionalist DMEs are ontic to explain what they are and why.}

\section{Objections }
One objection to my conventionalist treatment of countermathematicals is that it only works for certain explananda, namely those that only depend on a~mathematical fact, and not for cases like the well-known magicicadas or hexagonal honeycombs
%\label{ref:RNDgEgmnfffl8}(Lyon and Colyvan, 2007),
\parencite[][]{lyon_explanatory_2007}, %
 whose explananda also depend on natural facts. However, the countermathematicals in these cases work the same way as the others.

We need a~conventionalist evaluation procedure for countermathematicals. We could, like Baron et al.
%\label{ref:RNDyPeqnRHzLS}(2017),
\parencite*[][]{baron_how_2017}, %
 hold fixed the ``morphism'' between mathematical structures and empirical structures, so that changes in mathematical structures have ramifications into empirical structures. For a~conventionalist, this would mean holding fixed that the concept in question applies, ramifying into the world accordingly, i.e., we imagine a~world where our changed concept still applies. I~don't think this idea damages the conventionalist picture in any way, but I~think we can do something much simpler, without talk of worlds altogether, which seems to track better how we actually reason through countermathematical scenarios. Consider the following evaluation procedure: a~countermathematical is true when and only when\footnote{``When and only when'' means just that---this is just your standard biconditional. I~am not here giving truth-conditions for countermathematicals. That would imply that I~take countermathematicals to \textit{mean} something about rules of inference and derivability, but I~don't. Nor am I~giving truthmakers for countermathematicals.} the new rule of inference expressed in the antecedent licenses the derivation of the consequent from given empirical (and other unchanged mathematical) background premises. Obviously, other inference rules that have not been changed are also allowed in the derivation. This is similar to what Lewis 
%\label{ref:RNDnIdHVo93fG}(1973)
\parencite*[][]{lewis_counterfactuals_1973} %
 called the metalinguistic theory of counterfactuals.\footnote{Unlike most other defenders of metalinguistic theories, I~do not prefer it because I~have some metaphysical problem with possible worlds. Conventionalists let a~thousand languages bloom. And I~don't intend to commit myself to the linguistic ersatzist view that possible worlds just \textit{are sets} of sentences or something of that sort, though what I~say is consistent with such a~view. (Even the modal realist will admit that to each world there corresponds a~unique set of propositions describing it---the linguistic ersatzist simply claims that the correspondence is identity 
%\label{ref:RND8TAlKRwwoS}(Bennett, 2003, p.303).
\parencite[][p.303]{bennett_philosophical_2003}.%
) I~prefer this metalinguistic theory of countermathematicals not for metaphysical reasons but because it doesn't require the complication of holding fixed that the changed concepts apply, it seems to describe better what we actually do when evaluate countermathematicals, and I~feel it just comports better with the conventionalist idea that mathematical truths express rules of inference.}

This evaluation procedure works for the countermathematicals at issue in DMEs regardless of whether the explananda are narrowed. Take the countermathematical ‘if 13 hadn't been a~prime number, then North American cicadas wouldn't have had 13-year life-cycles'. In the present case, we imagine actually adopting the rule that ‘prime' does not apply to anything ‘13' applies to. This can be seen as a~rule of inference for descriptions containing ‘prime' and ‘13'. Since the consequent follows from given background premises (e.g., that having a~life-cycle period that minimizes intersection with other periods is evolutionarily advantageous and that prime periods minimize intersection) using the new rule of inference expressed in the antecedent, it is true on this evaluation procedure that the cicadas wouldn't have had 13-year life-cycles. Let us examine the reasoning of Baron et al.
%\label{ref:RND3c0hBln4I0}(2017, p.11)
\parencite*[][p.11]{baron_how_2017} %
 regarding this countermathematical and show that this is exactly what they are doing---using the mathematical claim in the antecedent as an inference rule to reach the consequent via given background premises---their platonistic excesses are just that. They are concerned to show specifically the truth of the countermathematical, ``If, in addition to 13 and 1, 13 had the factors 2 and 6, North American periodical cicadas would not have 13-year life cycles''. Here is their reasoning:

To evaluate this counterfactual, we start in the mathematics. [1]~We hold fixed as much as we can by changing multiplication to behave like multiplication*. This leaves 13's factors as desired. This gives us a~structure, S*, that is just like the natural numbers, except that 13 is not prime, and factorises via 2 and 6. [2]~Because we are holding fixed the relationship between the mathematical and physical structures, the physical structure that is now being mapped onto S* must twist to keep up with the counterfactual change. [3]~The result is that an interval of 13 years is now divisible into six two-year segments, or into two six-year segments. [4]~It follows from this that a~cicada with a~13-year life cycle will overlap with predators that have two-year and six-year life cycles and [5]~thus that 13 is not an optimal way to avoid predation. [6]~So cicadas won't evolve 13-year life cycles. [7]~So [the countermathematical] is true. (11)

The first three claims, which I've collectively labeled [1], are, according to the conventionalist, simply telling us to imagine adopting a~new inference rule according to which 13 has the factors 1, 2, 6, and 13, while leaving all other inference rules unchanged. Claim [2] is the morphism claim we discussed above, the conventionalist analog of which would be holding fixed that the concept in question applies. Since we are here illustrating a~different evaluation procedure that relies simply on descriptions and not on worlds, we can ignore claim [2]. Claim [3] is simply an application of our new inference rule: since 13 has the factors 1, 2, 6, and 13, an interval of 13 years is divisible into six two-year segments, or into two six-year segments. Claim [4] is inferred from [3] using normal inference rules that have not been changed, which is fine since [1] tells us that only one inference rule has changed. Similarly, claim [5] is inferred from [4], and claim [6] is inferred from [5] and empirical background premises (e.g., that nothing suboptimal will evolve), using normal inference rules that have not been changed. They conclude [7], that the countermathematical is true. Thus, they have concluded that the countermathematical is true because its consequent can be inferred from given background premises using the inference rule specified in the antecedent (and any other unchanged inference rules).

The same points apply, \textit{mutatis mutandis}, to the hexagonal honeycombs and other cases. For example, if the structure that divides a~planar region into regions of equal area using the least total perimeter were not a~hexagonal grid, then the honeybees' combs would not have been a~hexagonal grid. Here we imagine adopting the rule that ‘hexagonal grid' does not apply to anything ‘structure that divides a~planar region into regions of equal area using the least total perimeter' applies to. Again, since the consequent follows from given background premises (e.g., that producing the largest honeycomb cells using the least wax is evolutionarily advantageous) using the new rule of inference expressed in the antecedent, it is true on the metalinguistic evaluation procedure that the combs would not have formed a~hexagonal grid. In general, whenever mathematical necessities appear ineliminably in a~scientific explanation, they play the normative role of making explicit the conceptual norms linking the mathematical concepts applied in its empirical explanans-statement(s) to mathematical concepts applied in its empirical explanandum-statement. That is their function as expressions of rules of inference, rules for transforming empirical descriptions.

The metalinguistic theory of counterfactuals faces a~notorious problem: the problem of cotenability
%\label{ref:RNDzS832W51oO}(Goodman, 1947).
\parencite[][]{goodman_problem_1947}. %
 Consider the counterfactual, ``If this match had been struck, it would have lit''. According to the metalinguistic theory, this is true if and only if ``it lights'' can be derived from ``this match is struck''. Obviously, this derivation doesn't work without further premises. But what further premises is it legitimate to include? Certainly allowed are laws of nature and premises that are implicit in the context of the conversation we are having. And equally certainly, we cannot allow the truth that the match was not struck. That would generate a~contradiction and, assuming classical logic, every consequent would follow. Goodman 
%\label{ref:RNDL0xTeuTJ0x}(1947)
\parencite*[][]{goodman_problem_1947} %
 argued that cotenability with the initial premise (i.e., ``this match is struck'') was a~condition for inclusion into the further premises, where a~sentence S~is cotenable with the initial premise P~if and only if it is not true that if P~were true, then S~would be false. Of course, he knew that this was circular, since the definition of cotenability was given in counterfactual terms. The problem of cotenability is to provide a~definition that isn't in counterfactual terms.

My metalinguistic approach to countermathematicals may avoid this problem because we are only changing a~rule of inference. We are not changing any premises; we are using actual, rather than counterfactual, premises. Obviously, we need to know what the given premises are, and there are decisions that need to be made about what premises and rules of inference to hold fixed in counterconceptual scenarios, but these decisions exactly parallel the decisions about what to hold fixed in any account of countermathematical reasoning (e.g., the decisions Baron et al. must make about what ontic facts and morphic relationships to hold fixed). I~want to emphasize that I~am merely showing what the conventionalist must hold fixed to make the countermathematicals come out true. As Baron et al.
%\label{ref:RNDvFScHRESHs}(2017, p.12)
\parencite*[][p.12]{baron_how_2017} %
 note, ``To ask whether it is reasonable to hold these facts fixed when evaluating counterfactuals is to call into doubt the truth of the counterfactuals at issue''. However, even if this does not avoid the problem of cotenability, I~am not trying to give a~reductive account of counterfactuals generally, so I~have no problem relying on counterfactuals to explicate countermathematicals. Lewis 
%\label{ref:RNDF8OUiQeF7o}(1973, p.69)
\parencite*[][p.69]{lewis_counterfactuals_1973} %
 offers a~possible worlds solution to the problem. According to Lewis, ``$\chi $ is cotenable with an entertainable antecedent $\varphi $ at a~world $i$ if and only if $\chi $ holds throughout some $\varphi${-permitting sphere around $i$''. Defining cotenability this way makes the metalinguistic approach logically equivalent to the possible worlds approach 
%\label{ref:RNDLbJYXPo964}(Lewis, 1973, p.69).
\parencite[][p.69]{lewis_counterfactuals_1973}. %
 Perhaps the conventionalist wouldn't have a~problem adopting this so long as the talk of possible worlds can be understood deflationarily. Again, though, I~don't think there is any problem with the evaluation procedure where we hold fixed that the changed concept applies.

Another objection is that counterconceptual readings of countermathematicals incorrectly make the dependence of natural fact on mathematical fact into a~dependence of the meanings of descriptions of natural fact on the meaning of descriptions of mathematical fact. I~think there is something right about this objection, but it is obviously question-begging: it assumes anti-conventionalism---it assumes that pure mathematics offers substantive descriptions of mathematical facts. What is right about it is something I~do not take to be objectionable: that mathematical truths express rules of description---semantic rules---and there is nothing more to them. That is objectionable to many, but it is just conventionalism.

Another objection is that conventionalism makes a~mystery of why everyone in the world adopts the same semantic conventions and why mathematics has any explanatory power (in the standard, ``non-distinctive'' sense not at issue in the DME debate). First, the objection would prove far too much if it were correct. I~do not think either the agreement on, or the explanatory power of, mathematics is any less mysterious according to Platonism or any other (non-empiricist\footnote{The empiricist could explain (some of) the agreement of mathematicians by appeal to the empirical regularities to which all mathematicians have access and that, according to them, (at least basic arithmetic and geometrical) mathematical truths describe.}) anti-Platonism. Take fictionalism, for example. The objection applies equally to it---fictionalism makes just as much a~mystery of why everyone in the world adopts the same fictions and why some of these mere fictions have explanatory power. Second, and more substantively, there are a~few direct answers the conventionalist can give as to why mathematics has explanatory power and why everyone in the world adopts the same semantic rules---though, importantly, note that there does exist disagreement in mathematics, just as in logic
%\label{ref:RNDdf7lcsEF4g}(e.g., Balaguer, 2017; Beall and Restall, 2006; Davies, 2005; Priest, 2013; 2021).
\parencites[e.g.,][]{balaguer_mathematical_2017}{beall_logical_2006}{davies_defence_2005}{priest_mathematical_2013}{priest_note_2021}
 Some answers to the latter question can and have been given, mutatis mutandis, by the fictionalist. For example, as Colyvan 
%\label{ref:RND3YGPUUD5q6}(2011)
\parencite*[][]{colyvan_fictionalism_2011} %
 notes, the fictionalist can appeal to constraints on writing the fiction of mathematics, such as that ``new installments'' (i.e., theories) in the fiction be self-consistent, consistent with past installments, and not introduce unnecessary ``characters'' (i.e., entities). The conventionalist can appeal to these as constraints on the creation of semantic rules as well. The conventionalist (and fictionalist) can also appeal to a~shared (culture- or species-specific) aesthetic sense 
%\label{ref:RNDG48oUMDBhK}(see Steiner, 1998 for a~provocative discussion of the role of aesthetics in mathematical theorizing).
(\cite[see][]{steiner_applicability_1998} for a~provocative discussion of the role of aesthetics in mathematical theorizing). %
 Mathematicians whose proposed semantic rules fail to meet these constraints are sanctioned by the mathematical community, inducing further agreement. Finally, the conventionalist can also explain agreement, at least in basic arithmetic and geometry, in the way that the empiricist does---by appeal to empirical regularity.\footnote{If Maddy 
%\label{ref:RNDdYrW70Jdnt}(1990)
\parencite*[][]{maddy_realism_1990} %
 is right that we can perceive some sets, perhaps some of basic set theory can also be accounted for this way.} This is an idea prominent in Wittgenstein, who argued that the propositions of basic arithmetic and Euclidean geometry were empirical generalizations ``hardened into rules'' (i.e., rules of inference) and ``put in the archives'' (i.e., made immune from empirical refutation) 
%\label{ref:RNDgf0aydLTGg}(Bangu, 2018; Steiner, 1996; 2009; Wittgenstein, 1976; 1978).
\parencites[][]{cahill_later_2018}[][]{morton_wittgenstein_1996}[][]{steiner_empirical_2009}[][]{diamond_wittgensteins_1976}[][]{wittgenstein_remarks_1978}.%
\footnote{Perhaps this accounts for Kant's judgment that ``7 + 5 = 12'' is synthetic \textit{a~priori} 
%\label{ref:RNDTctn51HDDM}(Kant, 1781; 1787, B15).
\parencites[][]{kant_kritik_1781}[][B15]{kant_kritik_1787}.%
}

Regarding the standard, ‘non-distinctive' explanatory power of mathematics, it is perfectly consistent for the conventionalist to say that (many) mathematical concepts have empirical content and that applied mathematical propositions are straightforwardly descriptive. Conventionalism is a~theory of mathematical modality, not of the content of mathematical concepts. Conventionalism is thus compatible with the claim that mathematical concepts have (or can have, after suitable empirical interpretation) empirical, descriptive content\footnote{Compare Waismann's
%\label{ref:RNDljVV2EbIR3}(1986, p.66)
\parencite*[][p.66]{shanker_nature_1986} %
 description of Russell's position: ``For Russell the propositions of mathematics are, to be sure, \textit{a~priori}---they are tautologies---but the concepts are empirical.}'' and that this content contributes to mathematics' (non-distinctive) explanatory power by mapping 
%\label{ref:RND74y3JP7yMr}(Pincock, 2011; Bueno and French, 2018),
\parencites[][]{pincock_mathematics_2011}[][]{bueno_applying_2018}, %
 indexing 
%\label{ref:RNDvMBGTmoZ5S}(Melia, 2000),
\parencite[][]{melia_weaseling_2000}, %
 or representing 
%\label{ref:RNDissumSeKbC}(Saatsi, 2011)
\parencite[][]{saatsi_enhanced_2011} %
 explanatorily relevant quantities, magnitudes, etc. If this seems strange, consider a~comparison. Conventionalism about metaphysical modality is compatible with the claim that empirical concepts with descriptive content can figure in necessary truths. ‘Bachelor' is a~concept with empirical, descriptive content and ``bachelors are unmarried men'' expresses a~semantic rule governing it. Similarly, ‘triangle' is a~concept with empirical, descriptive content and ``triangles have three sides'' expresses a~semantic rule governing it. Conventionalism about mathematical modality does not rob mathematical concepts of their explanatory power. Furthermore, mathematical semantic rules need not be ‘arbitrary,' as evidenced by the Wittgensteinian idea mentioned above that basic arithmetic and geometric truths are empirical generalizations ``hardened into rules''.\footnote{Paul Audi (personal communication) helpfully suggested another sense in which semantic rules generally are non-arbitrary: presumably, the \textit{reason} that, e.g., ``unmarried'' applies if ``bachelor'' applies, is that the features of the \textit{world} in virtue of which the former applies are a~subset of those in virtue of which the latter applies. A~similar point is made by Thomasson 
%\label{ref:RNDOnevpezuFe}(2007b, p.70).
\parencite*[][p.70]{thomasson_ordinary_2007}.%
} Perhaps there remains for the conventionalist some version of Wigner's 
%\label{ref:RNDb7IfvG6Ilu}(1960)
\parencite*[][]{wigner_unreasonable_1960} %
 problem of the ``unreasonable effectiveness of mathematics''---though I~think the Wittgensteinian idea goes a~long way to dispelling this---but this is a~problem for everyone, and here I~can only refer the reader elsewhere 
%\label{ref:RND0zPbOHlJ9a}(see e.g., Steiner, 1998; Bangu, 2006; Clark, 2017; Bueno and French, 2018).
\parencites[see e.g.,][]{steiner_applicability_1998}{bangu_steiner_2006}{clark_wittgenstein_2017}{bueno_applying_2018}


Next, there is the worry that deflating instantiation by treating it as expressing facts about concept application results in too many things being counted as DMEs.\footnote{I~thank an anonymous reviewer for raising this objection.}
 For example, we want to explain why Claire has 5 apples. Because she has 2+3 apples and 2+3=5. The narrow explanandum would be the fact that Claire, who has 2+3 apples, has 5 apples. This weakly necessarily counterfactually depends only on the mathematical fact that 2+3=5; if 2+3 were not equal to 5, then Claire, who has 2+3 apples, would not have had 5 apples. Thus, we have a~DME of why Claire has 5 apples. I~can't appeal to instantiation qua concept application to exclude this case, since the concepts ‘5' and ‘2+3' both apply in this scenario. Thus, there are as many DMEs as there are equations.


This is a~great example, but I~think the objection misses the mark. Let me note four things. First, it is not obvious to me that explanations like this are always bad. It seems like this would be a~good explanation for someone who didn't know that 2+3=5, although I~admit that perhaps my intuitions are conflating explanation and evidence. Second, the truth of the relevant countermathematical doesn't depend on its being read counterconceptually nor on my semantic account of instantiation. For example, I~don't see why it wouldn't be true according to Baron, Colyvan, and Ripley's
%\label{ref:RNDe430e9fHNk}(2017)
\parencite*{baron_how_2017}
platonistic evaluation procedure. Third, why might some say 2+3 can't be (inflationarily) instantiated? Presumably they would say it's because 2+3 is not a~mathematical object. But why? Don't ‘2+3' and ‘5' both refer to the same object? After all, that's when an identity statement is true---when the expressions flanking the identity symbol refer to the same object. 2+3 is not a~strange conjunctive object composed of 2 and 3; it \textit{is} 5. 2+3 is instantiable, because 5 is instantiable and 2+3 is identical to 5 (via identity elimination or Leibniz's law---the indiscernibility of identicals, not the identity of indiscernibles). So, non-}conventionalists will also have to accept the instantiability of 2+3. Could non-conventionalists argue that 5, and, so, 2+3, isn't instantiable? But then what is the relation between the number 5 and Claire's 5 apples? If it isn't instantiation, call it ``shminstantiation''. Clearly shminstantiation is a~relation that can figure in DMEs, since many DMEs appeal to numbers to represent various quantities and magnitudes. Surely, we don't want to say there are no DMEs that appeal to numbers. Fourth, because of the last two points, many other accounts seem to render this a~DME too. If Mary's having 23 strawberries is constitutive of the physical task or arrangement at issue in Lange's
%\label{ref:RNDRnojrDhjJI}(2013)
\parencite*{lange_what_2013}
famous example, presumably having 2+3 apples is in this case, so Lange's account counts it as a~DME. Perhaps Lange could say that having 5 apples can be constitutive of a~physical task or arrangement at issue, but having 2+3 apples can't. It's hard to see how that could be, given that having 5 apples and having 2+3 apples are identical facts.\footnote{Lange
%\label{ref:RNDvRCt2CMpmM}(2016, pp.xviii–xix)
\parencite*[][pp.xviii–xix]{lange_because_2016} %
 mentions identity explanations favorably: ``that Samuel Clemens and Mark Twain are identical explains non-causally why they have the same height, weight, and birth dates''. Similarly, Lange might accept that there is some context where the fact that Claire's 2+3 apples are identical to her 5 apples explains non-causally why they have the same mass, price, etc. However, Kim's 
%\label{ref:RND0bNU7o7oVU}(2011, pp.104–105)
\parencite*[][pp.104–105]{kim_philosophy_2011} %
 arguments against identity explanations in the philosophy of mind may be relevant here.}
 Pincock's
%\label{ref:RNDvhfodFVk2O}(2015)
\parencite*{pincock_abstract_2015}
abstract dependence account similarly seems to have to accept this as a~DME. Claire has~5 apples because her apples instantiate the property of being 2+3 (in quantity), and 2+3=5.

Finally, I~may be able to rule this case out by arguing that it is a~case of denying one of the why-question's presuppositions, which is something distinct from explanation (Roski 2021). The narrowed why-question presupposes that Claire's having 2+3 apples and her having 5 apples are distinct facts, and the putative explanation undermines this. The why-questioner thus gains understanding, certainly, but this understanding is not explanatory. Furthermore, it is merely in virtue of learning that 2+3=5 that the why-questioner learns that Claire's having 2+3 apples and her having 5 apples are not distinct facts, so the putative explanation succeeds in undermining the presupposition regardless of whether the content of that knowledge (that 2+3=5) is interpreted conventionalistically. In other words, this move doesn't require any particular metaphysics of what the fact that 2+3=5 consists in. However, this move is not available to Pincock because for him the explanandum is not narrowed, i.e., the empirical explanans is not presupposed, so the why-question doesn't presuppose that Claire's having 2+3 apples and her having 5 apples are distinct.\footnote{The move is available to Lange, since he narrows the explananda of DMEs, but it would seem to be inconsistent with his approval of identity explanations. He could still accept the legitimacy of identity explanations if he could show that there are contexts wherein non-identity is not presupposed. This seems implausible though. In his
%\label{ref:RNDOp8SReDP9x}(2022)
\parencite*[][]{lange_defense_2022} %
 debate with Roski 
%\label{ref:RND5lQZ4BUocZ}(2021)
\parencite*[][]{roski_defence_2021} %
 over ``really statistical'' explanations, he says that an indication that \textit{p} is a~presupposition of the question ``Why is \textit{p} the case?'' is that it is pragmatically infelicitous to say ``I do not want to assume that \textit{p} is the case. But why is \textit{p} the case?''. However, it seems to me similarly pragmatically infelicitous to say ``I don't want to assume that Samuel Clemens and Mark Twain are distinct. But why are they so similar?'}' I~conclude that, if counting this case as a~DME is a~problem for conventionalism, it's a~problem many non-conventionalists seem to have as well.

Finally, Benacerraf
%\label{ref:RNDqwNRq9VFfu}(1973)
\parencite*[][]{benacerraf_mathematical_1973} %
 provides one of the most powerful objections to conventionalism.\footnote{Many consider Gödel's incompleteness theorems to provide the most powerful objection to conventionalism. According to some (including Gödel himself), the theorems decisiviely refute conventionalism and support Platonism. Much ink has been spilled on this, and I~merely point the reader to some ideas that might be helpful to the conventionalist 
%\label{ref:RNDE3oRLe2xAy}(Moore, 1999; Floyd and Putnam, 2000; Sayward, 2001; Awodey and Carus, 2004; Berto, 2009; Lampert, 2018; Warren, 2020).
\parencites[][]{moore_more_1999}[][]{floyd_note_2000}[][]{stalnaker_considering_2001}[][]{awodey_how_2004}[][]{berto_goparadox_2009}[][]{lampert_wittgenstein_2018}[][]{warren_shadows_2020}. %
 } His is a~challenge to provide a~homogeneous semantics for mathematical and non-mathematical discourse. A~homogeneous semantics would treat the following two sentences as both having the logical form of the third:
\begin{enumerate}
\item There are at least three large cities older than New York.
\item There are at least three perfect numbers greater than 17.
\item There are at least three FG's that bear R~to a
%\label{ref:RND526Vo2xdAz}(Benacerraf, 1973, p.663).
\parencite[][p.663]{benacerraf_mathematical_1973}.%
\end{enumerate}


I~believe Creath
%\label{ref:RNDboOUMlWUbf}(1980)
\parencite*[][]{creath_benacerraf_1980} %
 has mostly adequately addressed this problem and that he is right that Benaceraff begs a~central question in demanding a~substantive referential conception of truth for a~homogeneous semantics. I~return to this below. Here I~argue that the conventionalist can agree that sentences 1 and 2 have the logical form of sentence 3. Conventionalists tend to be deflationists. Deflationists of the kind I~have mind 
%\label{ref:RND61fesBLyAf}(Carnap, 1950; Schiffer, 2003; Thomasson, 2014; see also Price, 2011)
\parencites[][]{carnap_empiricism_1950}[][]{Schiffer2003}[][]{thomasson_ontology_2014}[see also][]{price_naturalism_2011} %
 have argued that, e.g., a~proposition like ‘It is possible that p' analytically entails ‘There is a~possible world where p,' which analytically entails ‘There are possible worlds'. Deflationists take such analytic entailments to have no substantive ontological implications; possible worlds are hypostatizations of our possibility-talk. Deflationists have made similar arguments for other kinds of entity. For example, ‘The ball is red' analytically entails ‘The ball has the property of being red,' which analytically entails ‘There are properties'. Again, such analytic entailments have no substantive ontological implications; properties are hypostatizations of predicates. The conventionalist can similarly say that ‘There are three mice' analytically entails ‘The number of mice is 3,' which analytically entails ‘There are numbers' (see Hale and Wright 
%\label{ref:RND5lQwD7PrI2}(2001)
\parencite*[][]{hale_reasons_2001} %
 for similar arguments, though I~do not intend to commit myself to their entire neo-Fregean program). Thus, nothing prevents the conventionalist from saying that sentence 2 has the logical form of sentence 3. This is analytic; sentence 2 analytically entails sentence 3. In fact, ‘There are at least three perfect numbers greater than 17' analytically entails ‘Certain mathematical objects stand in a~certain relation to each other', just as ‘There are at least three large cities older than New York' analytically entails ‘Certain cities stand in a~certain relation to each other'. This is just what Benaceraff demands of a~homogeneous semantics.

Here is where I~think Creath
%\label{ref:RND4HQTVOzMBq}(1980)
\parencite*[][]{creath_benacerraf_1980} %
 hits the nail on the head: to demand more than this, to demand that the semantics invoke a~\textit{substantive} notion of reference, so that the ontology mirrors the semantics, is to beg the question. Conventionalism need not be a~primitive expressivism. Just as the sophisticated metaethical expressivist 
%\label{ref:RNDcKQHwsMKqh}(e.g., Blackburn, 1993; 2005)
\parencites[e.g.,][]{blackburn_essays_1993}{kalderon_quasi-realism_2005}
 can speak of moral truths, reference, beliefs, knowledge, assertions, propositions, facts, and descriptions, the conventionalist can speak of mathematical truths, facts, etc. She could even say that a~true mathematical proposition describes a~mathematical fact, as long as all of these terms are understood in suitably deflationary senses that are clearly distinguished from the senses these terms have in talk of empirical, facts, etc., as I~mentioned earlier 
%\label{ref:RNDCD6D0thYUH}(Thomasson, 2020; see also Baker and Hacker, 2009).
\parencites[][]{thomasson_norms_2020}[see also][]{baker_mathematical_2009}.%


\section{Conclusion}
I~presented Pincock's
%\label{ref:RNDV4InNchalc}(2015)
\parencite*[][]{pincock_abstract_2015} %
 and my 
%\label{ref:RNDftEpP2mrOO}(Povich, 2021)
\parencite[][]{povich_narrow_2021} %
 ontic accounts of DME. I~explained conventionalism and extended it to DME. I~proposed counterconceptual readings of countermathematicals, and I~gave a~semantic account of the instantiation relation as concept application. These resources allow the nominalist to accept the existence of DMEs while denying Platonism, thus blocking the indispensability-inference from DMEs to Platonism. The conventionalist can also disagree with the \textit{critics} of indispensability arguments who simply deny the existence of DMEs by arguing that the mathematics is merely playing a~representational 
%\label{ref:RNDqLFnAlCkMe}(Saatsi, 2011)
\parencite[][]{saatsi_enhanced_2011} %
 or indexing role 
%\label{ref:RNDBIy3v6uoqM}(Melia, 2000),
\parencite[][]{melia_weaseling_2000}, %
 not an explanatory one. The critics often also mistakenly think that DME entails Platonism. The conventionalist can accept that the mathematics \textit{is} doing something explanatory, and she can even accept ontic\footnote{Baron 
%\label{ref:RNDkOCr5Rf6mi}(2020)
\parencite*[][]{baron_counterfactual_2020} %
 presents an account of DME that relies on countermathematicals, but it is unclear if he takes his account to be ontic. Conventionalism can be applied to Baron's account as well.} accounts of its explanatoriness, such as Pincock's or NOCA, suitably deflated.

There is much work to be done. I~do not pretend to have provided a~complete defense of conventionalism, nor of the metaontological deflationism that might be required for the conventionalist to answer Benaceraff's objection. All this is attempted in more detail in my forthcoming book
%\label{ref:RNDf8XJHULymW}(Povich, 2024).
\parencite[][]{povich_rules_2024}. %
 I~am under no illusion that it is impossible for there to be a~distinctively mathematical explanation that conventionalism will not be able to accommodate. What I~hope to have convinced the reader of here is that conventionalism at least provides a~promising avenue for the nominalist to accept DMEs while denying Platonism, thus undermining the enhanced indispensability argument, and that conventionalist DMEs as still arguably ontic.
 
 
 
 
 
 
\end{artengenv}