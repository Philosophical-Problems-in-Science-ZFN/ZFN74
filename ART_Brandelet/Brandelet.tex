\begin{artengenv}{Antoine Brandelet}
	{Can fiction and veritism\\go hand in hand?}
	{Can fiction and veritism go hand in hand?}
	{Can fiction and veritism go hand in hand?}
	{Université de Mons, Université de Namur, Belgium}
	{The epistemology of models has to face a conundrum: models are often described as highly idealised, and yet they are considered to be vehicles for scientific explanations. Truth-oriented---veritist---conceptions of explanation seem thereby undermined by this contradiction. In this article, I will show how this apparent paradox can be avoided by appealing to the notion of fiction. If fictionalism is often thought to lead to various flavours of instrumentalism, thereby weakening the veritist hopes, the fiction view of models offers a framework much richer than it seems at first sight. To do so, I will call upon the concepts of modality, counterfactual structure and credible worlds. In the end, veritism of explanation and fiction can indeed go hand in hand, but the scope of explanations we can hope to draw from models must be more precisely delineated.
	}
	{fiction, explanations, idealisations, models, veritism.}



\section{Setting the Problem}


\lettrine[loversize=0.13,lines=2,lraise=-0.03,nindent=0em,findent=0.2pt]%
{M}{}odels are ubiquitous in science. It is widely acknowledged that they are of central epistemological importance, and understanding their nature and function has become one of the most discussed topics in philosophy of science. There is no agreement on the general framework in which we need to address the modelling problem, and finding one is probably not even desirable. The reason for the variety of available approaches is obvious: diversity of uses and diversity of objects. Models are supposed to serve many functions: represent physical systems, provide explanations or help in theory construction, to cite only a few. Also, many kinds of objects, abstract and concrete, are considered as models: sets of mathematical equations, algorithms, graphs, drawings, or scale-models, for example. If we want to understand the explanatory power of models, which is our main topic here, we must tackle this problem of diversity.

With the acknowledgement of the wide use of models also comes a much more difficult conundrum. One of the main features of modelling practices is the idealisation\footnote{Here, I call ``idealisation'' the general process of simplification, the exact distinction between abstraction, idealisation and/or approximation is not relevant for my point.}: models are always simpler than the systems or situations they depict or explain. ``Models are generally caricatures of the natural world''~\parencite{Chakravartty2001}. The examples abound: the billiard-ball model of gases considers molecule collisions as perfectly elastic, predictions of planetary motion are obtained via the hypothesis that planets are perfectly spherical and with constant mass density, the Lotka-Volterra predator-prey model describes two populations with no outside influences, models in economics are concerned with perfectly rational agents making their decisions from all the available information.

This seems at first problematic, not to say paradoxical: models are vehicles for explanations, and yet they contain idealisations, distortions, purely fictional objects or even impossibilities. Following Elgin \parencite*{Elgin2017}, facing the use of these ``felicitous falsehoods'', one could argue for the relaxation of truth as the main epistemic goal of science, hence her critique of \textit{veritism}.

In the context of scientific explanation, veritism is the view that considers truth as a necessary condition for explanation~\parencite{Pincock2021}. Broadly conceived, I think a position needs to meet two conditions to be considered as veritist: firstly, there must be a form of correspondence or similarity between the model and the physical system it represents and, secondly, the framework must provide a way to distinguish between ``good'' and ``bad'' explanations. I will elaborate on that in section \ref{sec:veritism}.

We are faced with a dilemma: either the idealised models refute the veritist position, or the idealised models are not epistemically legitimate and the explanations provided by such models must be rejected. The second horn is clearly not in line with scientific practice: highly idealised models are often used to explain the phenomena observed, and the model-based explanations are considered a fully legitimate part of scientific knowledge. Veritism then seems refuted.

A similar debate is taking place in the context of representation. Besides explanation, models are also used to represent target physical systems. Explanation and representation are two distinct but closely related problems. If explanation faces the conundrum of veritism, representation is concerned with the more general problem associated to the realism/antirealism debate. Representations cannot be strictly qualified as true or false, but there seems to be something like a ``resemblance'' or ``correspondence'' between the model and the target that is at work.

In both cases, the worries are concerned with conceiving a model-world relation as a basis of knowledge, even if the ``veritist need not endorse any specific account of how models represent''~\parencite{Pincock2021}. I think this is true, but here, I want to show how one could use resources provided by responses to the representation problem to clarify the veritism conundrum.

Recently, Roman Frigg (see for example~\parencite{Frigg2009}, or, with James Nguyen~\parencite{Frigg2016, Frigg2020}) has developed the so-called ``fiction-view of models'', in which models are broadly conceived as \textit{Epistemic Representations}. This solves the aforementioned problem of diversity, as representative models are any object that is used as a vehicle to support surrogative reasoning. According to Frigg, models are fictions in the sense that they are an invitation to engage in a prop-oriented game of make-believe, understood in its Waltonian sense~\parencite{Walton1990}. Leaving technical details aside (see e.g.~\parencite{Toon2012} for a discussion of Walton's view), the main idea is that when using a model, we accept assumptions knowing that they are false but which acceptance function as principles of generation. We enter the fictional world they describe by pretending they are true. In other words, every proposition of a model could be preceded by an ``as-if'' clause referring to the rules of the game of make-believe.

The fiction view is intended to solve the problem of representation (i.e. how can models provide inferences from the model to the represented target system?). Here, I will focus on the explanatory power of models, but I think these fictional resources can be of great help to the explanation problem. This is what I will defend in section~\ref{sec:fiction}.

Expressed in this way, the fiction view seems to support purely instrumentalist and non-veritist conceptions of explanation. Models can incorporate any kind of false assumptions as soon as they empower explanatory or representational power, undermining all the truth-oriented hopes of the veritist or the realist. I think that even if make-believe puts fiction at the center, there is still a way to defend a slightly modified version of veritism. In subsequent sections, I will examine in more detail the consequences of this fiction-oriented view to find a way for veritism in the make-believe framework. More specifically, engaging in a game of make-believe is better understood as the process of building a counterfactual story, which highlights the importance of counterfactual reasonings in model-based explanation (Section \ref{sec:counterfactuals}). Models depict possible worlds from which we hope to draw inferences, and justifying these inferences is at the core of our present problem. I claim it is the counterfactual structure exemplified by the model that supports these inferential steps, and as models contain something inherently modal, the kind of explanations models provide also have something to do with modality. This is what I will explore in Section \ref{sec:modality}.

Section \ref{sec:veritism} sets the counterfactual structure as the focal point of the model-world relation the veritist is looking for. I show how the fiction view can help demarcate between valid and invalid explanations once the modal aspect of the model is appropriately understood. I then examine how it influences the kind of explanations one could expect to derive from models.

In Section \ref{sec:ocec}, I discuss how the fictional approach can help to clarify the opposition between the ontic and epistemic conceptions of explanation. The focus on idealisations and representations in model building if often taken as an argument in favour of epistemic approaches, that is conceptions that consider explanations to be products of some scientific activity involving various techniques, such as the fictional processes described in this paper. I argue on the contrary that critiques of the ontic conception focusing on idealisations and representations often miss the point: the ontic conception suitably understood does not deny the importance of fictional processes, but puts the focus on the referents of explanatory texts and representations. The fiction view is able to provide a framework in which questions about explanation and questions about the ontic or epistemic expectations of explanations are clearly distinguished.


\section{Fiction and Models}\label{sec:fiction}


In this section, I will (very) briefly summarise the main aspects of the fiction view of models.

Epistemology was of course not the first to use the concept of fiction. This term refers primarily to ``works of fiction'', understood in the purely artistic sense of the term, and it has been the subject of much discussion in aesthetics, for example. Recent interest for fiction in epistemology and philosophy of science often refers to Walton's seminal work, \textit{Mimesis as make-Believe}, which deals mainly with artistic representation, but offers a framework suitable to address epistemological issues.

The idea that science deals with theories or models that simplify reality is not new either, and one could find the premises of such an analysis for instance in Vaihinger's extensive use of the ``as-if'' statement~\parencite{Vaihinger1911eng}. That is an important shift in understanding the explanatory function of models. Models don't just simplify matters by isolating variables, idealising processes or abstracting properties. They are an invitation to think ``as-if'' things were so and so, just as a work of fiction is a prescription to imagine situations, people and places. The simple pendulum is a fiction in the sense that reading its theoretical description, we imagine a point mass oscillating at the end of a non-extendable, massless string, even if we know point masses and massless strings don't exist in reality.

Specifying a model by stipulating principles of generation is of course not the whole story. One could imagine any set of (at least coherent) assumptions and claim that it is a model by a mere \textit{modelling fiat}. But it is not. For a set of hypotheses to constitute a model, it needs to be applicable to the target system: the terms involved in the assumptions need to be interpretable in terms of the target. Most of the time, scientific models come already partially interpreted by a theory. The simple pendulum is not a model of real world pendula because scientists just say so, but because we can assign to each term (e.g. $m$, $l$, $g$) a target-interpretation (respectively mass of the bob, length of the cord, gravitational field magnitude), and the relevance of these variables is inherited from Newtonian mechanics, the theory in which the model takes place.

The model is then studied, investigated and manipulated. New fictional truths (i.e. propositions true in the model) are derived from the principles of generation. In a famous quote, Hacking writes that ``a model in physics is something you hold in your head rather than your hands''~\parencite{Hacking1983}. I think we can understand this quote literally. I take the derivation of new fictional truths as analogous to the manipulation, for instance, of a scale-model or a map: generally conceived as epistemic representations, models are manipulable because we can discover new propositions from postulated ones. It is clear for the example of the pendulum, but finding your way by making inferences from a map works in the exact same way: interpreting lines and color patches as roads and forests enable the map user to determine his position.

According to the fiction view of scientific representation, that is where the representative power of models comes from. It also explains why scientists often talk about abstract models as-if they were real objects, as we talk about Sherlock Holmes as-if he was a real person. Starting from Walton's make-believe theory, Frigg and Nguyen~\parencite*{Frigg2020} have developed the DEKI account of scientific representation, but for our present purpose, only the basic concepts of the make-believe view are necessary.

As a matter of fact, fiction is often thought to clash with truth. In libraries, there is a strict distinction between fiction and non-fiction sections. In everyday language, fiction is associated with imagination, fantasy or lies. If models incorporate such imaginary or false assumptions, it seems impossible to reconcile this view with veritism.

In her book \textit{True Enough}~\parencite*{Elgin2017}, Catherine Elgin establishes this incompatibility on the very first page: 

\myquote{
Modern science is one of humanity’s greatest cognitive achievements. To think that this achievement is a fluke would be mad. So epistemology has the task of accounting for science’s success. A truth-centered, or veritistic, epistemology must treat models, idealizations, and thought experiments as mere heuristics, or forecast their disappearance with the advancement of scientific understanding. Neither approach is plausible. We should not cavalierly assume that the inaccuracy of models and idealizations constitutes an inadequacy; quite the opposite. I suggest that their divergence from truth or representational accuracy fosters their epistemic functioning. When effective, models and idealizations are, I contend, felicitous falsehoods. They are more than heuristics. They are ineliminable and epistemically valuable components of the understanding science supplies.~\parencite[p.1]{Elgin2017}
}

As the make-believe view bets on the central importance of fictional aspects of modelling, truth-based explanations cannot be derived from models at all, it seems, and the conclusion is the same as Elgin's: \textit{Felicitous falsehoods} cannot be removed from the success of science, then \textit{veritism} must be abandoned.

I think there is nonetheless a way to defend a modified form of veritism in the context of the fictional view. Such a defense must proceed in two steps. The first concerns epistemic virtues. Elgin acknowledges the virtue of idealisations, but dismisses the one of truth. On the other side, the veritist claims that the main epistemic virtue is truth, but still, that doesn't mean idealisations cannot also have some kind of value and that the two cannot be articulated in a common framework. The second step, which will be our main concern in the remainder of this article, is to establish what limitations, if any, the fiction view imposes on the scope of model-based explanations.

There is a large literature about epistemic virtues, and various authors have defended that veritism and idealisation may not be as incompatible as we may think. For instance, Nawar argues that ``in grasping and idealizing claim \textit{as an idealizing claim}, if seems that one in facts grasps a truth''~\parencite[][his emphasis]{Nawar2019}. Sullivan and Khalifa~\parencite*{Sullivan2019} admit that idealisation has virtues, but that they are non-epistemic. Idealisations are used for convenience, simplicity or tractability, in this sense they are felicitous falsehoods, but only their true components can provide understanding. In the same vein, Lawler argues that ``falsehoods can play an epistemic enabling role in the process of obtaining understanding but are not elements of the explanations or analyses that constitute the content of understanding''~\parencite{Lawler2019}. Making the process/content of explanation distinction, called the \textit{extraction view}, is interesting for our purpose. Concerning the former, the fiction view enables all the usual virtues granted to idealisations by taking these fictionalising procedures as the central feature of models. Concerning the latter, to be fully adequate in the fiction context, we must clarify in terms of make-believe what exactly the content of the explanations provided by the models is.

Admitting there is a place for the virtue of idealisations even if truth still constitutes the main goal of inquiry in providing explanations, we must now turn to the question of explanations themselves. How can fictional models generate virtuous explanations? How can the false explain the real? This is what will see in subsequent sections.

\section{Counterfactuals at work}\label{sec:counterfactuals}


In this section, I will examine the lessons we can draw from the fiction view in the way models generate explanations. I will also see how my account may provide insights to understand the relation between models and laws via the use of counterfactual inferences.

Remember the fiction view proceeds in two stages: firstly, rules of the game that generate the fictional world are postulated; then, secondly, the model is explored and fictional truths are derived from these principles of generation.

Modulo the applicability of the model, any kind of assumption is \textit{a priori} acceptable, whether it be idealisation, abstraction or distorsion. These are not the whole story. Some of the fictional processes the scientist may use to build the model are not reducible to these simplifying assumptions. As a matter of fact, models sometimes feature impossibilities, assumptions that are incompatible with the theory in which the model takes place.\footnote{I here use the term theory in a very loose sense: a system of concepts and general principles. The question of the theory-model relationship is a debate in its own right and is beyond the scope of this article, but I will briefly sketch a possible answer that naturally arises in the fiction framework in the remainder of the article.} Take the case of the simple pendulum as an example: point masses not only do not exist in reality, but are also impossible according to the Newtonian picture of the world. I take this observation to be a good reason to turn ourselves to a modal conception of models, where notions of possibility are directly implemented in the framework.

Models, like fiction, seem to describe possible worlds, that is worlds that resemble ours in many aspects, but where Sherlock Holmes is a detective, rabbits talk or point masses oscillate without friction. If we acknowledge models may contain impossibilities, the fiction view itself faces a conundrum, as coherence and consistency are often presented as necessary conditions for a work of fiction to be acceptable.\footnote{This is also related to debates around the \textit{willing suspension of disbelief}. Interestingly, Kendall Walton is one of the critics of this approach \parencite[see e.g.,][p.~7]{Walton1978}.} In the context of epistemology, this seems to suggest that some models contain clashing propositions, thus creating a contradictory story and undermining the potential veridic ground of explanations. What is an impossible world, according to the fiction view? The distinction is important, as any account oriented towards truth must be able to offer acceptability criteria. At this point, we also face the modal version of the initial conundrum: how can models be impossible descriptions of real (therefore possible) phenomena?

To resolve the apparent paradox, we must clarify the use of (im)possibilities in models to understand how they fit together. I think the resources of the fiction view are of great interest, here, as it draws our attention to the important distinction between the inside of the model (fictional---intradiegetic\footnote{Literally: inside (intra) the narrative (diegetic).}---propositions) and the outside.

When qualifying assumptions as impossible, it is always with an implicit frame of reference in mind. Something is possible or impossible only according to a set of hypotheses or axioms. In this regard, all the fictional propositions are, by definition, diegetically possible. When we say that a model features impossibilities, it is with regard to the exterior, to the laws, theories or principles we believe to be true in reality. Models are often thought to be interpreted structures that link the theories to the empirical world, the fiction view generalises this idea to any kind of inference vehicles (theoretical or material) and principles of generation (theories, laws or imaginary entities).

Newtonian mechanics is testable only if we build a model that generates, when applicable to a target system, empirical propositions. The theory itself acquires representational or explanatory power via the models that depict concrete situations. Remember the Ian Hacking quote. What does it mean to manipulate a theoretical model? Taking the simple pendulum example, that means plugging values for the free parameters end seeing what comes out. The model does not contain number values, but a network of relations between variables with potential inputs. Manipulating a model is then playing with counterfactual propositions: when applicable to the target, they generate propositions of the form ``had the parameters had such and such values, the system would behave as such and such''. That is how models are empirically testable. As the ``as-if'' clause is characteristic of the fictional process, these models generate ``what-if'' propositions about their targets. Models are sets of propositions arranged in a counterfactual structure and potentially applicable to a target physical system and some, but not all, their principles of generation are derived from a more general theory: the simple pendulum is a Newtonian model because it embodies what are considered to be the Newtonian laws.

So far so good, but does it still make sense to talk about being the model of a theory in this context? As we have seen, models often contain impossibilities as premisses, which makes them incompatible with the theory. If models are believed to serve as intermediaries, they must respect the properties they pretend to exemplify. Models may be caricatures of the world, but not of the theories they are models of.

The analogy with fiction sheds light on the complex relation of models not only with physical systems, but also with theoretical principles. A model is neither a strict exemplification of theoretical principles, nor it is a faithful representation of target systems. Models are strange, sometimes abstract, objects, made of heterogeneous parts like an epistemic Frankenstein's creature.

Yet, the relation of fiction to truth is also a more complicated story, and we sometimes use them to learn about the real world. It is widely acknowledged that fiction is not reducible to falsity. One of the most used examples is Tolstoy's \textit{War and Peace}, which contains lots of accurate details about the Napoleonian wars. This example shows there is a use of outside truths for diegetic purposes, and that a reader could learn about the real world by reading the novel. However, this could hardly be described as genuine knowledge as, in the absence of prior background knowledge, the reader may be as justified to believe in the love story of Pierre and Natasha. In the case of scientific models, we have seen that, because of postulated impossibilities, the same problem arises.

There is, I think, another example more suitable to our epistemic concerns that will illustrate the way the fiction view may solve the problem. Let's consider the fables of La Fontaine. At first glance, they describe anthropomorphic animals in imaginary situations. The characters themselves are less human-like in their attitudes than they are archetypes of certain behaviours. Yet, the fictional world depicted by the fable has the function of providing information on actual human behaviour. This is the role of the final moral. Here, I think, the analogy makes sense: idealised entities are postulated, some of their properties are intended to be interpreted literally while others are not, and the final epistemic goal is to state something true of the exterior of the fiction.

The analogy also seems to suggest an important role assigned to the theories. What makes the moral of a fable a good indication of human behaviour? I claim it is a kind of legal compliance with respect to laws that are supposed to govern human behaviour. In this sense, the fable functions as a model: the fictional entities are embedded in a web of relations and these relations are described by certain laws. It is no threat to the compatibility with the background theory because it is made to describe humans and not animals\footnote{I leave aside the symbolic aspect of the use of certain animals in that particular case.}. The model is a model of that theory because the objects it depicts, even if purely fictional, enter the web of relations described by the theory. In this sense, models are intermediary objects between a theory and target systems.

This is, I think, the main contribution of the fiction view: it leads to a sort of structural \textit{divide et impera} strategy. Models are considered as modal structures, mechanisms used to exemplify a web of counterfactual relations. The modal structure is the skeleton of the model, fictional hypotheses are the flesh that makes the model more tractable or easy to interpret. More importantly: the modal structure exemplified is partially independent from the fictional assumptions. 

Manipulating the simple pendulum, we find a relation for the period that is independent of time, regardless of knowingly false hypotheses. That is this counterfactual structure that makes the work. When a model is empirically adequate, we have to ask ourselves: what makes the predictive job? When explaining with models, the question is: what is doing the explanatory job? Explaining in terms of the validity of the principles of generation (rules of the game) is a no-go (at least for the veritist): they are knowingly false. Then if something is doing the explanation, it is the modal structure itself, inherited from the laws\footnote{My account does not rely on any particular conception, or metaphysics, of laws, I use the term in a minimal sense: I take counterfactuals as the focal point of explanation and laws are known for supporting counterfactuals. Models exemplifying counterfactual dependencies are in this sense inherited from laws.} of the theory the model is a model of. Asking for an explanation of the independence on mass of the period of the pendulum, one may present the simple pendulum model, show how the equation $T = 2\pi \sqrt{l/g}$ is derived and explain why it is applicable to the target pendulum. In this context, idealisations, abstractions and all the fictional processes are epistemically valuable not because they constitute the explanation, but because they help make the model tractable by expliciting its counterfactual structure. As Bokulich~\parencite*[p.1]{Bokulich2016} puts it: ``Fictional models can succeed in offering genuine explanations by correctly capturing relevant patterns of counterfactual dependence and licensing correct inferences.''

Exploring the real via falsehoods still seems a dangerous strategy if we are not able to distinguish between valid and invalid explanations, as there is still some kind of \textit{pessimistic meta-induction} (PMI) threat here. Bokulich takes the example of a non-explanatory fiction: the epicycles model of the Solar system. From a purely fictional-counterfactual point of view, this and the Newtonian-heliocentric models are on a par, but the former fails in capturing the right counterfactual pattern. This is of course circular reasoning if we can't provide a justification for what ``right'' here means. This is what Bokulich calls the ``justificatory step'', but she claims that a general account is impossible, thereby undermining the veritist hopes: 

\myquote{
However, what does it mean to say that a fictional representation is adequate? It has to be more than mere empirical adequacy. Unfortunately, here is where I think abstract philosophical generalizations purporting to hold across all model explanations give out, and one needs to turn to the nitty-gritty details of the science in question. What is to count as an adequate fictional representation is something that has to be negotiated by the relevant scientific community and will depend on the details of the particular science, the nature of the target system, and the purposes for which the scientists are deploying the model~\parencite[p.734]{Bokulich2012}.
}

I think that objection is no fundamental threat to my account for two reasons. Firstly, finding the common ground veridically interpretable for explanations and providing a general acceptability criterion are two different things. In this sense, my account is minimal, as it proposes a necessary condition. I agree that the sufficiency argument may be context dependent. Secondly, the resources provided by the fiction view seem compatible with the general structural arguments put forward by many scientific realists. The response to the PMI-like worries raised by Bokulich may follow the same path.

In the same vein as Bokulich, Potochnik claims that

\myquote{
depicting causal patterns regularly motivates departures from accuracy of any given phenomenon; this is why idealizations are used to represent as-if. Put in these terms, the present idea is that idealizations positively contribute to generating understanding by revealing causal patterns and thereby enabling insights about these patterns that would otherwise be inaccessible to us~\parencite[p.95]{Potochnik2017}.
}

Again, the highlighting role of idealisations is considered as one of the main aspects of models, but the exact role and nature of those \textit{causal patterns} remain unclear. Potochnik writes her view does not rely on any particular metaphysics of causation, but she acknowledges causal patterns must be real: ``How, then, can we tell if understanding is actual and not merely apparent? For this, the causal pattern apparently grasped must be real. Briefly, for a causal pattern to be real, it must be embodied (to some degree or other) in some range of phenomena.''~\parencite[p.115]{Potochnik2017} But still, she dismisses truth as the main epistemic goal of science: 

\myquote{
The clearest illustrations of this are idealizations themselves, which are quite far from the truth but, in the right circumstances, are epistemically acceptable nonetheless. So, in my view, science simply is not after the truth. There are some important ways in which truth still may be involved in the scientific enterprise, but in each case, it is only a means to other ends~\parencite[p.117]{Potochnik2017}.
}

So, when truth is indeed put forward by scientists, it is always with other, more important, goals in mind: understanding of phenomena, which is not truth-oriented. But what would it mean to highlight real causal patterns in a non-truth-centered way? To me, it seems clear that qualifying something as real in a model, even highly idealised, involves some kind of correspondence and, in the end, (at least partial) truth. There is a tension at play, here: accuracy is a requirement of epistemic acceptability, but we must refuse to align it with truth, as idealisation help generate explanations and understanding. The initial puzzle is still unsolved.

Fiction view seems particularly adequate to treat this problem, as it makes clear the distinction between the fictional process by which the world of the model (the description of possibilities potentially not realised) and the counterfactual structure it brings out. The role of the fictional part is to bring counterfactual structure to the front, but the structure itself is immune to fiction, as we have seen. I propose to take this observation as the focal point of our veritist considerations.

One counterargument would be to argue that once fiction enters the picture, it propagates to the entire model. Setting aside approximations and idealisations, when a purely fictional entity like silogen atoms~\parencite{Winsberg2006} are ineliminable parts of a model, even the counterfactual structure makes a truth-oriented interpretation and explanation impossible. Here again, I think fiction solves the conundrum.

From a make-believe point of view, postulating new knowingly-unreal particles like silogen atoms is generating a model in which these impossible particles exist, have properties and interact. The model itself is predictive and has good empirical adequacy, but to fully understand its use in generating explanation, we also must examine how it is applied to the target system. As fictional assumptions, silogen atoms have no interpretations in terms of the target system, they are not strictly applicable. Engaging in the silogen game of make-believe means we accept the assumptions inside the model-world, but not outside. The model is still capable of being veridically evaluated because what is confirmed is not a matter of entities or even physical processes. Silogen atoms are fictional entities that ground a set of properties, properties that feature in laws (quantum mechanics and solid-state physics, say), laws that are exemplified in the model, thus exhibiting a counterfactual structure. Interpreting silogen atoms as existing would be a misuse of the model, just like asserting that the moral of the fable is only valid for anthropomorphic animals.

The veritist base of the model explanation is then to be found in the way the laws are generating the skeleton of the model, and the model is evaluated by manipulating this structure to make it generate empirically testable propositions. Giving an interpretation of the counterfactual dependencies is the ground of any explanation. We are not forced to identify these patterns as causes and consequences. My proposition remains agnostic and offers room for different interpretations and levels of ontological commitments. This process is fiction-blind but empowered by the fictional freedom.


\section{Modality and Explanations}\label{sec:modality}


Let us now turn to the kind of explanation such models can generate.

Verreault-Julien\parencite*{VerreaultJulien2019} argued that models may provide ``how-possibly explanations'', which are propositions of the form $\diamond (p \text{ because } q)$. Facing the issue of how highly idealised models featuring impossibilities may provide such an explanation, Verreault-Julien insists on the importance of making a clear distinction between model-propositions (translated in my account, we could say fictional propositions, i.e. propositions true in the fiction) and world-propositions. ``What model propositions (e.g., unrealistic assumptions) do is to give reasons to believe in the truth of the possibility claim'', he writes. In his view, models may depict impossibilities and still support possibility claims, which are non-fictional world propositions.

With this I agree, only if we consider not propositions of the form $\diamond (p \text{ because } q)$ for given $p$ and $q$, but a counterfactual function that assigns a value of $q$ to a value of $p$. The model is used not to support a set of definite statements about the target, but a counterfactual structure supposed to be embedded in the phenomena we are interested in explaining. These functions also often have a higher arity, as they link several physical values.

More generally, and to avoid any metaphysical commitment to the term ``because'', I think suitably applicable models support claims of the form $\diamond (p \sim q)$, where $\sim$ is a relation between quantities exemplified by the model. The simple pendulum depicts an impossible object but nonetheless supports the modal claim that connects a certain number of quantities applicable to real-world pendula when interpreted as mass, length, etc.

Another support for the need for a clear fictional/world propositions  distinction comes from the fictional process itself. As Verrault-Julien makes clear, it is possible for a model to non-trivially support possibility claims only if we already have $\diamond p$ and $\diamond q$. But of course, these fictional assumptions may be impossible, understood as world propositions, hence the need to distinguish ``diegetic'' from ``extradiegetic'' claims, as I suggested in the previous section.

Sugden suggests that the posited similarities between the model and reality may license inductive inferences \parencites[see e.g.][]{Sugden2000}[p.240]{Sugden2013}. He gives an argument that takes the following form: $p\rightarrow q$ in the model and $p$ and $q$ in the world gives good reason to infer that $p\rightarrow q$ in the world. As in any inductive argument, a similarity between specific observations is posited, and in the case of model-based inference, it is the model-world similarity that supports induction. The second step of Sugden's argument is analogous to what I call applicability, and the focus on the counterfactual structure may support the inferential step by providing ground for the justification of the similarity.

Sugden also takes modality to be an important feature of models: ``So what might increase our confidence in such inferences? I want to suggest that we can have more confidence in them, the greater the extent to which we can understand the relevant model as a description of how the world could be.''~\parencite[p.24]{Sugden2000}. There are many ways the world could be, and Sugden proposes \textit{credibility} as a way to sort them, but in his view, it is not clear how credible worlds (i.e. worlds that could be real) would deal with postulated impossibilities. Are impossible worlds credible?

Elsewhere~\parencite{author2021}, I have defended the view that the inductive framework proposed by Mill~\parencite{Mill1843eng} is transformed into a deductive system when causal laws have highlighted the relevant structure in experiments. Regularities are inductively explained and then serve as the ground for deductively making new predictions. Avoiding the discussion about the nature of causality in Mill's work, I think we can export his view on the notion of laws to consolidate our fictional and counterfactual conception.

I disagree with Sugden when he writes: ``To put this another way, the real world is equivalent to an immensely complicated model: it is the limiting case of the process of replacing the simplifying assumptions of the original model with increasingly realistic specifications.''~\parencite[p.~23]{Sugden2000} because fiction allows for impossibilities that are not only simplifications or idealisations. Natural laws, understood as Mill's inductive generalisations, generate models. Models are then credible only if compatible with these laws, but credibility does not prevent impossibilities.

Laws express sets of relations obtained via inductive reasoning over observed regularities. These counterfactual relations are exemplified in models, and exemplification may involve all sorts of fictional processes such as, but not limited to, idealisations. That is where the felicitous falsehoods draw their epistemic virtues. Postulating impossibilities (i.e. incompatibilities with laws) is the main modelling freedom offered by the fiction view, but as the laws generating the models express a web of counterfactual dependences, only these relations need to be compatible for the model to be acceptable. In Sugden's terms, impossible worlds can be credible, as long as the model is robust through manipulation and the counterfactual structure remains applicable to the modelised phenomena.

There seems to be some kind of ``truth-eligibility'' in fictional statements. Manipulating a fictional model does not only mean deriving new fictional statements from old ones, it also means being able to give an interpretation of those propositions in terms of the target. Using the model adequately is also a matter of knowing what is not to be supposed to be true, or even possible at all. If a simple pendulum user claims that the empirical adequacy of the model supports the existence of point masses, he is obviously not using the model properly, the non-existence of point masses is no argument against the model itself. We may be wrong about some aspects or properties of the depicted entities, even about their existence, but the conservation of the counterfactual structure through manipulation of the model is the focal point of our understanding and explanations. That is how we shed light on the fixed point of counterfactual dependencies the model exemplifies.

Model propositions must not be taken at face value. A model is more like a dynamical entity, a counterfactual engine that generates sets of propositions about a physical system and provides a justification for inferences to the world. They are ``descriptions of how the world could be''~\parencite[p.241]{Sugden2013} equipped with inference rules (inherited from the laws) that guide the fictional reasonings. 


\section{Veritism reloaded?}\label{sec:veritism}


As in many subfields of epistemology and philosophy of science, realist and truth-oriented positions have faced strong arguments from all sides. If, as I claim, veritism of explanation can be retained in the fiction view of models, what limitations does it impose on veritism?

I think a position needs to meet two conditions to be considered as veritist:

\begin{enumerate}
    \item There needs to be some kind of correspondence or similarity between the world and the model at play, and this relation must be, at least partially, the ground of the explanation. In particular, pure empirical adequacy, acceptance by the scientific community or compliance to scientific norms is insufficient if not based on the correspondence relation.
    \item The correspondence must be equipped with a demarcation mechanism: we must be able to link the validity of the model-based explanation with the other epistemic virtues, such as empirical adequacy, and the difference between acceptable and non-acceptable explanation must be constructible in the framework.
\end{enumerate}

Firstly, as I have explained in the previous sections, I take the counterfactual structure exemplified by the model to be the ground of the explanation. Clearly, the model must reproduce and predict empirical data in order to be veridically evaluated. Robustness for a range of input values, i.e. manipulability of the model, offers the ground for inductive inferences. This is where the correspondence relation comes into play: as in Sugden's example of inductive (and abductive) argument, a similarity gives the argument its skeleton. I claim this similarity to be a similarity of counterfactual structure inherited in the model from the laws, understood in the minimal sense of inductive generalisations supporting counterfactual reasonings.

Secondly, the lesson from the fiction view is that the demarcation between valid and invalid explanations is possible, but that we must refrain to interpret veridically anything that is not part of the counterfactual structure in our explanations. For example, using a model of silogen atoms to explain phenomena, we may talk about these atoms as-if they were real, but the veritist ground of the explanation are the quantum-mechanical processes at play in solid-state physics. Explaining with silogen would be a misuse of the silogen fiction.


\section{Fiction and the ontic/epistemic accounts of explanation}
\label{sec:ocec}

In this section, I will show how the fiction-view of models defended in this paper can help clarify issues related to the conceptions of explanation and the OC/EC opposition.

Conceptions of explanation are typically classified into two categories: ontic and epistemic. Commonly conceived, ontic conceptions (OC) consider explanations to be concrete things that exist ``out-there'', independently of any theorising about them. On the other hand, epistemic conceptions (EC) take explanations to be the product of a scientific activity involving various techniques, such as representation of the phenomena to be explained. According to the EC, there is then no explanation if no scientist is doing the explaining.

In \parencite{Bokulich2016}, the focus on the central role of idealisations and fictional processes in modelling is taken to be the sign of a need to move ``beyond the ontic conception''. Her line of argument is quite straightforward and easily phrased in fictional terms. Models feature deliberate falsities and idealisations which are part of their explanatory power. Models considered as fictions have representational power, and these representations are epistemic products build and used by scientists to explain phenomena. In this context, it seems clear that EC is a much more natural way of conceiving explanations.

Differently put, if, as it is claimed in OC, explanations are ``full-bodied things'' \parencite[p.40]{Craver2014}, ``objective features of the world'' \parencite[p.27]{Craver2007} that exist independently of the epistemic goals, arguments and activities performed by scientists, then how could deliberate falsities be explanatory? The recognition of the explanatory power of fictions rules out this possibility, and the fiction-view provides an argument in favour of EC. That, of course, should be no surprise: it is at the very heart of \textit{make-believe} oriented approaches to take models as epistemic products designed and used by certain agents in order to achieve certain goals according to certain norms of evaluation.

Also, as \parencite{Wright2015} notes, the ontic conception, for example in Salmon's phrasing, involves something as an ``exhibition''. If it is deeply unclear to point out what could be an exhibition ``in re'', it is quite natural in the fictional perspective, where exhibition can be understood as a form of model-representation: it is exactly the role of idealisations and fictional hypotheses to exhibit features in models to support the explanation. This is the act of representing which is at the core of model-explanations and in which, as I have emphasised earlier, fiction plays a positive role.

Bokulich~\parencite*{Bokulich2018} takes her analysis a step further where she proposes what she calls the ``eikonic'' conception. In this perspective, explanations also are ``the product of an epistemic activity involving representations of the phenomena to be explained''. Leaving the details aside, the eikonic conception is close to the fiction view discussed in this paper. What I think is more important is the distinction she introduces, as it may help clarify the roots of the OC/EC debate.

One of the key aspects of the eikonic proposition is that it constitutes a conception of explanation and not an account of explanation. An account of explanation is a claim about how explanations work, while a conception is a claim about what explanations are. The problem is that there is an ambiguity in the general treatment of the OC/EC debate: are these conceptions in the general sense, or in the restrained use of Bokulich?

This is an important point because examples traditionally associated with the ontic conception may fall under the epistemic umbrella if seen not as explanations, but as ways of explaining. For example, causal or mechanistic accounts are on the ontic side: it is the electron that hit the screen that explains the presence of a white dot. But one could also argue that causes, mechanisms and unobservable entities are features of idealised models that are not explanations but descriptions of how explanations work, thus constituting an account of explanation. A mere reference to a cause or mechanism is not sufficient to decide whether it is part of an ontic or epistemic conception of explanation. A causal account is compatible with both the ontic and epistemic conceptions depending on what is considered to be the explanation itself.

If, as Bokulich asserts, early literature on explanation fails to draw the account/conception distinction, it becomes necessary to examine if such a distinction helps clarify the ontic/epistemic opposition. Unlike her, I don't think it does, and I claim the concept of fiction I am focusing on in this paper may explain why. That is what I will illustrate in the remainder of this section.

The lack of the aforementioned distinction and the only recent critique of the ontic conception may seem surprising at first glance. The central theses of the ontic conception, namely that explanations are not arguments but things and that explaining doesn't involve representation, have been left uncriticised for a long time, even if they are obviously false. Some proponents of EC, including Bokulich, share this surprise, but on the contrary, we might as well be surprised that such obviously false positions could be attributed to anyone. This type of astonishment could just as easily have been the result of a misunderstanding as of an ill-considered philosophical position.

To shed light on the tension at play in this defence of EC, let us extend the analysis proposed in the eikonic conception. As a conception of explanation, the eikonic position is a claim about what explanations are and it falls under the epistemic umbrella for reasons discussed just before. Also, the eikonic conception is independent of the particular account of explanation one chooses to defend. Bokulich also claims that her view is compatible with scientific realism, just as I claim with the fiction view. Explanations are then considered as epistemic products involving a representational activity via idealised models. The compatibility with different accounts comes from the fact that the model may produce the explanation via different means, e.g. by using covering laws, causes, causal patterns, mechanisms, etc.

If a realist justification is to be found in this context, it must be via some kind of correspondence. I elaborated on the realist requirements and the correspondence with the causal patterns the model exemplifies (or exhibits!) in the previous sections. The question that arises is then the following: how could a purely epistemic explanation be realistically justified? Objective features of the world cannot be deliberately false as models are, but if we look for a justification in the realist sense, it must be grounded on a correspondence with objective features, may it be entities, mechanisms, causes or structures, depending on the flavour of realism one prefers.

In his response to \parencite{Bokulich2016}, Craver develops an argument along these lines. Contrary to many critics of OC, Craver starts by asserting that proponents of the ontic conception simply do not dismiss the explanatory use of texts and arguments:

\myquote{
    When defenders of the ontic view write about explanations as if they are, ``out there”, as they are, independently of what anyone knows or thinks about them, they are expressing realism about the appropriate referents of explanatory texts, not abandoning the idea that scientists use texts to express explanations \parencite{Craver2019}.
}

If proponents of EC reject OC because of the obvious use of epistemic products to express explanations, it is a misunderstanding of what makes OC ontic. Now, if we come back to the fictional or the eikonic view, we may as well claim they fall under the ontic conception. Idealisations and fictional models are ways to extract and convey information, but they do not constitute the explanation. What explains is the correct correspondence with real causal patterns. On that topic, Craver is explicit:

\myquote{
    But conveying explanatory information about X and truly representing the explanation for X are not the same thing. Friends of the ontic conception should say that idealized models are useful for conveying true information about the explanation, but that they are not true representations of the explanation \parencite[p.50]{Craver2014}.
}

And, in the next paragraph, he adds:

\myquote{
    Once these are separated, the problem of idealization is clearly not a problem for theories of scientific explanation; rather it is a problem for philosophical theories of reference. The question at the heart of the problem of idealization is this: What is required for a given explanatory text to convey information about the ontic structure of the world? This is an important question, but it is a question about reference, not a question about explanation. We only invite confusion if we fail to keep these questions distinct \parencite[p.50]{Craver2014}.
}

This is exactly what the fictional view is able to provide: accounting for the representational and idealised component of explanation while justifying realistically these explanations. Fiction makes a clear distinction between the world of the model and what it may refer to. Both OC and EC can make room for various epistemic strategies in conveying the relevant information: idealisations, representations, modelling in terms of causes, mechanisms, laws, etc. In the fictional view I defend, the model is not the explanation, the model is supporting the explanation by making a set of hypothesis manipulable. When asking, in the vein of Bokulich, about the fictional view as a conception of explanation, there are two equally valid answers: it is ontic if the model provides a justified representation of real causal patterns, it is epistemic if the fictional models are considered as tools for predicting and manipulating physical systems with no reference to real entities or causes or mechanisms. In this understanding of ``conception'', the OC/EC debate aligns with the realism/antirealism opposition: what makes the ``electron hit the screen'' explanation ontic is the reference to the real photon and its properties. Following Craver, as conceptions of explanation conflate with reference problems, if Bokulich claims her eikonic view is compatible with realism, it then falls under the ontic umbrella, contrary to her claim.

On the other hand, fiction naturally handles the division of labour between questions about what models refer to if they are to count as explanatory and questions about how explanations work, i.e. the question of \textit{accounts} of explanation. Make-believe and exhibition of causal pattern is a description of how model-based explanations work, while the justification in terms of realist-oriented correspondence provides an ontological understanding. In the same vein, Bokulich's eikonic view is an \textit{account} of explanation, neither ontic nor epistemic, but compatible with realist as well as antirealist views on reference, just as the fiction view is to be considered as an account of explanation, irrespective of questions of reference and OC/EC classification. As I showed earlier, this account is compatible with veritism and, more broadly, with a realist attitude towards science.



\section{Conclusion}\label{sec:conclusion}


Is there a place for veritism of explanation in the fiction view of models? I think it does, but exploring the aspects of models as games of make-believe about credible worlds imposes limits on the kind of veritism one can hope to achieve.

First of all, the fictional view can accomodate the epistemic virtue of idealisations and approximations: they epistemically contribute to explanations and understanding by simplifying matters, and the fictional freedom in the construction of models offers all the strategies to scientists in doing so.

The central conundrum of the fiction/veritist approaches may be clarified by turning our attention to the modal aspects of models. Models are descriptions of credible worlds which we can manipulate to generate, when suitably interpreted, propositions about physical target systems. Manipulating a model means we explore the robustness across a range of input values of the embedded counterfactual structure. Fictional processes, like postulating non-existent entities can help in this exploration, but the counterfactual structure itself is immune to fiction and remains veridically interpretable.

The relation between laws and models also appears clearly in the fictional context. Laws generate models in the sense that they are the expression of the counterfactual dependences the model contains. The structure is the skeleton, fictional hypotheses are the flesh that facilitates reasoning and interpretation. When the model is found to be robust, we can say that the counterfactual structure reproduces the one of the physical system. This is the ground of explanations, and it remains veritist in the sense that it is the agreement between the two structures that strengthens the inductive inferences we draw from the model to build explanations. That also explains why laws are so important for explanations. The observation of regularities needs to be explained, the law is the explanation, and the models make the link between laws and the world by making them manipulable. Models are not the explanation, but we need models in order to express explanations.

Regarding the OC/EC debate, the fiction view puts forward the positive role of representations and idealisations while making room for realist and truth-oriented justification of explanations. The ontic slogan according to which explanations are objective features of the world must be interpreted as a claim about the reference of explanations rather than a claim about the nature of explanations qua arguments or representations. This shows that the opposition at play is less about explanations that it is about the more general problem of reference and realism. The fiction view marks clearly the distinction by making a claim about how explanations work (i.e. it provides an account of explanation) and by providing the conceptual resources to tackle the problem of the reference of idealised claims and models.


\end{artengenv}
