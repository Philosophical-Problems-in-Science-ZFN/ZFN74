\begin{editorialeng}{Abel Peña \& Cory Wright}
	{The nature and norms of scientific explanation: Some preliminaries}
	{The nature and norms of scientific explanation: Some preliminaries}
	{The nature and norms of scientific explanation: Some preliminaries}
		{California State University Long Beach,\\Department of Philosophy}
%	{The nature and norms of scientific explanation: Some preliminaries}
	
	


\lettrine[loversize=0.13,lines=2,lraise=-0.03,nindent=0em,findent=0.2pt]%
{T}{}here are at least two deep and related debates about explanation. Firstly, there is a~debate about its nature. What are explanations? How do they appear? What features do they have? It is a~traditional metaphysical or descriptive debate, with much the same structure as debates over the nature of knowledge, causation, levels, reduction, or other phenomena of interest to philosophers of science. Hence, just as we can ask what knowledge or causation consists in, so too can we ask the same question about explanation; and we can inquire about the nature of scientific explanation more narrowly or even explanation in particular scientific disciplines. Secondly, there is a~debate about the norms of explanation. What distinguishes a~better explanation from a~lesser one? What are their virtues, what do they aim at, and what are the conditions of their success? It is a~traditional axiological or prescriptive debate, and concerns the evaluation of explanatory goodness. Hence, just as one might instead debate how best to reason abductively, or whether conductive and inductive inferences can both be good in the same way, so too do philosophers have serious work to do in articulating the norms of explanation and generating criteria for distinguishing them.

There is reason to believe that these two debates about explanation are ordered by a~dependency or priority relation, and so not equivalently deep; for one cannot separate out the good explanations from the bad if he does not know what explanations are in the first place. Consequently, it seems that a~resolution to the second debate over the norms of explanation depends on the availability of a~working resolution to the first debate over its nature. For example, there are many terrific discussions of the normative force of model-based idealizations, and, in particular, whether some further corrective step of `alethic repair' or de-idealization is required. In these discussions, it is widely assumed without argument that models just are explanations. However, some philosophers distinguish between explanations and what they call `explanatory texts'. An implication of this distinction is that models are instead scientific representations of explanations rather than explanations \textit{per se}, such that discussions of the normative force of model-based idealizations would be discussions of the norms of representation rather than of explanation. For other philosophers, what this reveals is just that the explanation–explanatory-text distinction rests on a~mistake. But does it? Determining the answer involves resolving the first debate about the nature of explanation; an analysis of our concepts is called for before we can turn to normative or practical topics.

Some philosophers demur. In a~spirited attack on `verbal' metaphysics, Chwistek
%\label{ref:RNDDiJy0UqlLW}(1932)
\parencite*[][]{chwistek_tragedia_1932} %
 lambasted the belief that ``through honest and free discourse it is possible to reach the essence of the concepts hidden behind the words uttered in common language, such as good, love, etc.'' 
%\label{ref:RNDSEoHScM3s9}(Chwistek, 2017, p.2).
\parencite[][p.2]{trybus_tragedy_2017}. %
 Chwistek's skeptical attack expressed a~kind of deflationary attitude toward conceptual analysis that strikes at the very heart of the method---an attitude which has since wended its way into the literature on scientific explanation. One sees this attitude among those who believe that, at bottom, science aims instead to solve problems 
%\label{ref:RNDGLSe4NkmLa}(Laudan, 1977; Elliott, 2021; Levenstein and et al., in press).
\parencites{laudan_progress_1977}{elliott_research_2021}{levenstein_problem-ladenness_2024}. %answer 
 For instance, Koertge writes, ``suppose we could all agree on one or more [conceptions] of explanation. What would we do with them? I~am in general dubious […] about the value of asking `What is x?' questions […]'', and ``I suggest that we reverse the order of investigation. We should begin by asking what problems a~good theory about scientific explanation might reasonably be expected to solve'' 
%\label{ref:RNDKLi7L0EvLo}(Koertge, 1992, pp.85–86).
\parencite[][pp.85–86]{koertge_explanation_1992}. %
 Many pluralists and pragmatists, like Mantzavinos 
%\label{ref:RNDYis8PK8Ypo}(2016, p.14)
\parencite*[][p.14]{mantzavinos_explanatory_2016} %
 and Kitcher 
%\label{ref:RND6meMGc2rWl}(2023, p.60)
\parencite*[][p.60]{kitcher_whats_2023} %
 respectively, concur; they insist that the traditional project of conceptual analysis is regressive or sterile: there's nothing to say about what all and only the scientific explanations have in common, and they don't yield action-oriented recipes for improvement of our lot.

These lines of thought often prove to be self-defeating. Even if we can retrain our attention onto the nature of problems instead of explanations, the same questions recur \textit{mutatis mutandis}. What is a~problem? What would count as a~solution? If something is a~scientific problem, what features does it have? And if something has those features, is it a~scientific problem? The deflationary attitude might be redeployed elsewhere; but we should expect the same point to hold true for any substituend that isn't just analytically brute: scientific practices, explanatory games, methodological constraints, etc. Some might want to turn instead toward normative debates. What makes one problem more interesting, or more tractable, or more useful for making progress? But here, again, the debates are not equifundamental. Whatever the goodness or badness of problems (the utility of practices, games, etc.) consists in, one cannot separate out the good ones from the bad if she does not know what problems are in the first place.

Conceptions of explanation provide explicative answers to questions about what explanations are. For example, according to the epistemic conception of explanation (EC), scientific explanations are complexes of representations of entities or phenomena in the physical world. It takes these representations to aim at increased knowledge about the entities in the physical world, and it takes the norms of explanation to be the norms of knowledge
%\label{ref:RNDgVOff9GW1d}(Wright and van Eck, 2018, p.998).
\parencite[][p.998]{wright_ontic_2018}. %
 This conception is often associated---but often too closely---with the groundbreaking work of Hempel 
%\label{ref:RNDPzZD9uXQHu}(1965)
\parencite*[][]{hempel_aspects_1965} %
 as well as the so-called `San Diego School' of explanation from Kitcher, Churchland, Perini, Bechtel, Burnston, and others besides. As Scriven 
%\label{ref:RNDP5jGz62dkj}(1962, p.224)
\parencite*[][p.224]{scriven_explanations_1962} %
 characterized it, ``[a scientific explanation] is a~topically unified communication, the content of which imparts understanding of some scientific phenomena''. According to the ontic conception (OC), however, explanations are instead complexes of the physical entities so represented, which are located among the other spatiotemporal parts of reality and which do not aim at representational norms of goodness. As Forge 
%\label{ref:RNDdF1exMwKXw}(1998, p.76)
\parencite*[][p.76]{anapolitanos_explanation_1998} %
 wrote, ``on [OC], an explanation is actually a~state of affairs in the world''. Following in this vein, Craver 
%\label{ref:RNDiDNNEJ1W4G}(2007, p.27)
\parencite*[][p.27]{craver_explaining_2007} %
 wrote, `[ontic] explanations are not texts; they are full-bodied things. They are facts, not representations. […] There is no question of [ontic] explanations being right or wrong, or good or bad. They just are'. Likewise, Jenkins 
%\label{ref:RNDBjFvtk6Evz}(2008, p.64)
\parencite*[][p.64]{jenkins_romeo_2008} %
 wrote that explanations conceived ontically are `worldly things', that is, ``not the sort of things that are true or false, but rather the sorts of things that take place or obtain, such as facts or events''. This conception is often associated with the work of Salmon 
%\label{ref:RND1Sv409lsjh}(1984),
\parencite*[][]{salmon_scientific_1984}, %
 as well as the so-called `Pittsburgh School' of Salmon, Woodward, Craver, Andersen, and others.

Bokulich
%\label{ref:RNDcUx7lsN04m}(2016, p.263)
\parencite*[][p.263]{bokulich_fiction_2016} %
 introduced a~helpful distinction between views about what explanations are (`conceptions') versus how explanations work (`accounts'). She puts this distinction to work by correctly noting that Salmon endorsed both the ontic conception and the causal account; and, she might have added, Hempel endorsed both the epistemic conception and the nomological account. She is also right to note that one can reject the ontic conception while accepting the causal account, just as one can accept the epistemic conception while rejecting the nomological account.

By using and enforcing the distinction between conceptions and accounts, we stand to gain a~more sophisticated interpretation about the literature. For instance, Salmon
%\label{ref:RNDTLXPXjE9sb}(1984, p.301)
\parencite*[][p.301]{salmon_scientific_1984} %
 distinguished EC and OC from what he called the `modal conception', according to which ``explanations explain by showing that what did happen had to happen, from which it follows that no incompatible alternative could have happened''. But the modal `conception' does not readily specify what explanations are. Indeed, one could accept EC, or could accept OC, and build in these modal commitments about how explanations show off counterfactual necessity. That is, using Bokulich's distinction, one could accept these other conceptions while endorsing the modal account. Similarly, one can accept EC while endorsing the erotetic account, which scientific explanations answer certain kinds of why- and how-questions. To take another instance, Faye 
%\label{ref:RNDa1M5cWTSg9}(1999; 2007)
\parencites*[][]{faye_explanation_1999}[][]{faye_pragmatic-rhetorical_2007} %
 describes what he calls the `pragmatic-rhetorical conception' (PC), according to which explanations are informationally relevant answers that are advanced in the problem context of a~rhetorical situation whenever the speaker intends to solve the problem. But again, deploying Bokulich's distinction, we can now see that PC is either a~conception of explanation, albeit one that's derivative of EC, or else an erotetic account of how these types of representations work in communicative situations, and so not a~genuine competitor.

The aim of this special issue of \textit{Philosophical Problems in Science/Zagadnienia Filozoficzne w~Nauce} (ZFN) is to survey whether or not a~consensus is at hand in these debates and to help settle what it can. The overarching foci are twofold: (i) the nature of scientific explanation, with special attention to the debate between OC and EC, and (ii) the norms of scientific explanation, with special attention to so-called `ontic' (or better, `alethic') norms like truth and referential success and epistemic norms like intelligibility and idealized understanding. It called for advocates of various conceptions to articulate the current state of these debates. Researchers and scholars from around the globe---including Poland, Canada, Korea, The Netherlands, the United States, Greece, Austria, and Belgium---contributed. The special issue also attempts to provide an opening for new work on the norms of explanation, such as truth or model-based accuracy, information compression, abstraction, and generalization.

The first paper in this issue, Panagiotis Karadimas's `Explanation, representation, and information' argues that EC can encompass almost all scientific explanations by conceptualizing them as relations between representations and thus renders OC gratuitous. He arrives at this conclusion by first demonstrating that abstract explanations do not ultimately make up a~distinct category apart from non-abstract ones, and thus EC and OC are not differentially applicable. Karadimas then develops some new objections uniquely faced by OC; these include the dominant role of information transfer (rather than direct observation) in scientific explanation and the fact that the ontic conception doesn't accommodate explanations that involve false propositions. It is concluded, then, that the applicability of the EC is significantly preferable in virtue of its vastly broader scope.

In `Dimensions of explanation' Eric Hochstein rejects the three-fold division of explanation into exclusive communicative, representational, and ontic aspects. Instead, his paper argues that a~scientific explanation can always be analyzed along each of these dimensions. After describing his proposal, Hochstein describes how to dispatch some potential problems. The result is a~means for resolving disputes involving mechanistic explanations.

The topic of mechanistic explanation also serves as the backdrop of both Jinyeong Gim's `The ontic-epistemic debates of explanation revisited: The three-dimensional approach' and Michał Oleksowicz's `Ontic or epistemic conception of explanation: A~misleading distinction?'. Both authors concur with Karadimas that mechanistic explanations are more likely to be epistemic than ontic. Both Gim and Oleksowicz begin with a~survey of how the debates seem to have changed, moving from a~discussion of Hempel and Salmon's works in the last century to the current state of the debates over explanation. Salmon
%\label{ref:RND9C8OgQJdsM}(1989)
\parencite*[][]{salmon_four_1989} %
 attributed OC to Coffa. But was Coffa instead reacting to the thought of Scheffler 
%\label{ref:RNDUDy0miMWKI}(1963)
\parencite*[][]{scheffler_anatomy_1963} %
 on inscriptionalism? Did Coffa's anti-Kantian attachment to Bolzano, and his study of Russellian propositional complexes, influence his understanding of OC, and therefore Salmon's? These deeper historical lines have yet to be excavated.

Gim arrives at a~three-part classification, comprising a~relational dimension of explanatoriness, a~conceptual dimension of the nature of explanation, and a~normative dimension evaluating the goodness of explanations. What Gim calls `dimensions' are different than what Hochstein intends, however. His first dimension (`explanatoriness') is analyzed in terms of form, force, and relevance. This might be mapped into Bokulich's discussion of accounts of how explanations work. The second and third dimensions concern the proper conception of the nature of explanation and its normativity, respectively---basically, what we described as the first and second debates at the outset. By exploring each of these in depth, Gim aims to show that mechanistic explanation need not be ontic and can be epistemic in each dimension.

Mark Povich argues in `A conventionalist account of distinctively mathematical explanation' that this kind of explanation averts a~strong objection to the ontic status of other non-conventionalist accounts of distinctively mathematical explanation (DME). This conception treats the explananda and explanantia of (DMEs), which are mathematical facts, as ontic items and the explanatory relation between these items as likewise ontic. Povich's article begins with an exemplar of a~DME, expounds upon recent ontic conceptions of DME, develops a~conventionalist account of DME and lastly anticipates some proposed challenges. For example, an open question concerns the required conception of facts. On an inflated compositional or Tractarian conception of facts, DMEs seem genuinely ontic; but conventionalism is a~harder sell. On a~deflated propositional or Fregean conception, the ontic nature of DMEs may be called into question. By coupling conventionalism and DMEs, Povich aims to show that a~path forward for the ontic conception of explanation remains open, if only in the mathematical domain.

As Povich's paper shows, the first debate between conceptions like OC and EC has continued to evolve into different areas. One suggestion is that many discussions have recently moved in the direction of normative analyses of `ontic and epistemic constraints' that explanations must satisfy in order to count as good scientific explanations. van Eck
%\label{ref:RNDqU3MHJKnem}(2015; 2018; 2021)
\parencites*[][]{van_eck_reconciling_2015}[][]{wright_ontic_2018}[][]{van_eck_mechanist_2021} %
 has argued that appealing to ontic constraints just unwittingly concedes the debate between EC and OC. Michał Oleksowicz's `Ontic or epistemic conception of explanation: A~misleading distinction?' engages with this evolution: he attempts to provide a~summary of the first debate between OC and EC that can make sense of transitioning interest to the second debate: what's been called the `normative turn'. Oleksowicz, following Illari and others, contends that the debate has shifted to a~consideration of norms and constraints on scientific explanation that differentially benefit various conceptions and accounts. For the New Mechanists, the idea of mechanisms as complex causal systems in the world and of mechanistic explanations as tools for discovering those complex systems are both critical to an understanding of natural phenomena.

Understanding is regulative norm emphasized by many advocates of EC, and resurfaces in Federica Malfatti's review of McCain's \textit{Understanding How Science Explains the World}. Malfatti teases out two competing views of scientific explanation discussed in the book. The first view suggests that explanation mirrors the facts, depicting dependency relations actually holding in the real world; the other view, however, ties explanation to the contingencies of evidential support and evidential standards. Malfatti wrestles with a~possible reconciliation of these positions, and emphasizes the role that scientific realism can play in making such cases clearer and stronger.

Many advocates of both EC and OC are friendly to realism, but are then pressed to say something about idealizations. In `Can fiction and veritism go hand in hand?' Antoine Brandelet considers this familiar tension in the use of models as scientific explanations. On one hand, the thesis that truth is a~necessary condition of explanation (veritism) belies acceptance of model-based explanations as integral to explanatory knowledge. On the other hand, models can be highly idealized---even to the point of gross simplification and distortion. Brandelet co-opts a~fictionalist strategy from responses to the representation problem in modeling to mount a~defense of veritism. He argues that this fictional approach ultimately helps to clarify the distinction between OC and EC, asserting that the former does not deny the importance of such fictional processes as idealizations but, rather, emphasizes the referents of explanatory texts and representations. The debate between the two conceptions, then, is over the additional question of whether or not those referents just are explanations.

Relatedly, Kristian Campbell González Barman's review of \textit{Models and Idealizations in Science: Artifactual and Fictional Approaches}, edited by Cassini \& Redmond, provides brief overviews of the chapters, including those on the topic of de-idealization by Carrillo \& Knuuttila, Cassini, and others, and on fictionalism by Frigg \& Nguyen, Salis, García-Carpintero, and others. However, González Barman takes the additional step of relating several of these to ontic and epistemic concerns. Barman particularly highlights some of the strategies and problems involved with respect to idealizations and de-idealizations in scientific modeling, as well as various fictional interpretations, to draw pertinent lessons for proponents on both sides of the debate.

\enlargethispage{1.5\baselineskip}
In `Mechanisms `all the way down'?', Ioan Muntean reviews of \textit{Mechanisms in Physics and Beyond}, edited by Falkenburg \& Schiemann. The New Mechanists have primarily confined their views to the life sciences, and the various chapters in this book are among the first attempts to apply this doctrine to the lower-level sciences. Muntean critically analyzes arguments adduced for and against implementing mechanistic explanations in physics, where nomological, mathematical, and non-causal explanations play a~much more common role. The review is rounded out by an analysis of contributions dealing with an explanatory framework of levels, the cognitive process of interpreting mechanisms, and the relation of physical systems to computations.

Overall, the papers in this special issue jointly demonstrate the fruitfulness of these debates, and they also lay some additional groundwork for developing the theoretical issues even further. It is important to acknowledge the many referees that generously lent their expertise to provide feedback and recommendations. Finally, thanks to Ning Shao; special appreciation goes to the journal's editors and staff for providing guidance and assistance, and for their dedication to intellectual discussion in philosophy of science.

%\begin{flushright}
%Abel Peña \& Cory Wright\\
%Department of Philosophy\\
%California State University Long Beach
%\end{flushright}

\end{editorialeng}
